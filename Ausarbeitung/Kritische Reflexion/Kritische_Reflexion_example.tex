%Reflektiere über umsetzung der Komponenten, was hat gut funktioniert, was nicht, was war einfach, was war schwer, was war unerwartet, was war nicht so gut.
\section{IoT Device}
Das IoT Device wurde von einer Person als Hauptentwickler und mehreren Unterstützenden Entwicklern erstellt.
    \subsection{Hardware}:
    Geplant war das IoT Device als ein Gerät, das die Luftfeuchtigkeit und Temperatur misst und diese Daten an einen Server sendet.
    Für die Hardware wurde ein fertiger Bausatz von Alibaba gekauft, der alle benötigten Komponenten enthielt. Damit war es möglich, die Hardware schnell zusammenzubauen und zu testen.
    Die Sensoren sind nur mittelmäßig genau, was aber für ein Proof of Concept ausreichend ist.
    \subsection{Software}:
    Die Software wurde in C entwickelt. Dabei wurde CLion mit CMake als IDE verwendet.
    Dadurch kam es zu komplikationen im Setup, da die IDE nicht richtig konfiguriert war. Daraus resultierten verzögerungen die das Projekt belasteten
    Mehr feste Entwickler hätten hier helfen können.

Insgesamt lief die Entwicklung des IoT Devices gut, aber es gab einige Herausforderungen, die bewältigt werden mussten.
    Die Hardware war einfach zusammenzubauen, aber die Software war schwieriger zu implementieren als erwartet.
    Es gab einige Probleme mit der Kommunikation zwischen dem IoT Device und dem Server, die behoben werden mussten.
    Auch die Genauigkeit der Sensoren war nicht so hoch wie erhofft, was die Ergebnisse beeinflusste.
    Dennoch konnte das IoT Device erfolgreich entwickelt und getestet werden, und es erfüllt die Anforderungen des Projekts.
    