% ============= Package Einstellungen & Sonstiges ============= 
\documentclass[a4paper,12pt]{report}
\usepackage[left= 2.5 cm,right = 2.5 cm, bottom = 2.5 cm]{geometry}
\usepackage[onehalfspacing]{setspace}

\usepackage[
pdftitle={Studienarbeit},
pdfsubject={Entwicklung einer smarten Bewässerungslösung mit Web-Anbindung},
pdfauthor={Maximilian Schüller, Fynn Thierling, Justus Siegert, Lukas Maier, Timon Kleinknecht},
pdfkeywords={},	
hidelinks %Links nicht einrahmen
]{hyperref}

\usepackage[utf8]{inputenc}
\usepackage[ngerman]{babel}
\usepackage[T1]{fontenc}

\usepackage{fancyhdr}
\usepackage{color}
\usepackage{csquotes}
%\usepackage{cite}
\usepackage[backend=biber, autocite=inline, style=ieee, natbib=true]{biblatex}
\addbibresource{literatur.bib}
\DefineBibliographyStrings{ngerman}{andothers = {{et\,al\adddot}},}
\usepackage{url}

\usepackage{graphicx} %für Einbindung von Grafiken
\graphicspath{{img/}} %Pfad für Grafiken

\usepackage{pdfpages}

\usepackage{todonotes}

%\usepackage[printonlyused]{acronym}
% Akronymverzeichnis
\usepackage{hyperref}
\usepackage{array}
\usepackage{supertabular}
\usepackage{acro}
\acsetup{make-links}
%Überschrift "Abkürzungsverzeichnis" setzen
\section*{Abkürzungsverzeichnis}
\addcontentsline{toc}{section}{Abkürzungsverzeichnis}
\begin{acronym}[STRIDE]
	%\acro{EXP}{example}	-> Im text verwenden mit \ac{EXP}	
	
\end{acronym}

\usepackage{minted} %für Darstellung von Code
\usepackage{float}
\usepackage[german]{varioref} %für schönere Referenzierung von Abbildungen

\fancyhead[L]{} % Linke Kopfzeile leer lassen

\usepackage{xspace}

% Define a command to ensure consistent space after acronym
%\newcommand{\acspace}{\xspace\hspace{1em}}

\makeatletter
\renewcommand*{\aclabelfont}[1]{\textbf{\acsfont{#1}\acspace}}
\makeatother


\newcommand{\source}[1]{\vspace{1ex}\noindent{\small \textit{Quelle: #1}}}

\newcommand{\initializeBibliography}{
	\ihead{}
	\printbibliography[title=\Literaturverzeichnis] 
	\cleardoublepage
}

\usepackage{enumitem}
\usepackage{amssymb}

\usepackage{listings}
\usepackage{xcolor}  % Optional, für zusätzliche Farbanpassungen
\lstset{ 
	language=Python,                % Programmiersprache
	basicstyle=\ttfamily\scriptsize,     % Grundschriftart (monospace) und -größe
	keywordstyle=\bfseries\color{blue}, % Schlüsselwörter fett und blau
	commentstyle=\itshape\color{green!50!black}, % Kommentare kursiv und grün
	stringstyle=\color{red},        % Strings rot
	numbers=left,                   % Zeilennummern links anzeigen
	numberstyle=\tiny\color{black},  % Stil der Zeilennummern (klein und grau)
	stepnumber=1,                   % Jede Zeile nummerieren
	numbersep=8pt,                  % Abstand der Zeilennummern zum Code
	backgroundcolor=\color{gray!10},  % Hintergrundfarbe des Codes (weiß)
	frame=single,                   % Rahmen um den Code
	rulecolor=\color{black},        % Rahmenfarbe
	captionpos=b,                   % Position des Titels (unten)
	breaklines=true,                % Zeilen umbrechen, wenn sie zu lang sind
	breakatwhitespace=true,         % Zeilenumbrüche nur an Leerzeichen
	showstringspaces=false,         % Leerzeichen in Strings nicht anzeigen
	tabsize=4,                      % Breite eines Tabs
	%escapeinside={(*@}{@*)},        % LaTeX-Befehle innerhalb des Codes
	morekeywords={print,def,class}, % Zusätzliche Schlüsselwörter für Python
	extendedchars=true,             % Unterstützt erweiterte Zeichen (z. B. Umlaute)
}
% Label für Listings ändern
\renewcommand{\lstlistingname}{Beispiel}
\newcommand{\code}[1]{\colorbox{gray!20}{\consola #1}}

\usepackage{titlesec}
\titleformat{\chapter}[hang]
  {\normalfont\huge\bfseries}
  {\thechapter\quad}
  {0pt}
  {} 


% ============= Dokumentbeginn =============
\begin{document}
	
	%Titelseite
	\thispagestyle{empty}
\begin{center}
\begin{tabular}{p{\textwidth}}
		
\begin{center}
	\textbf{\Large{\textsc{
			Entwicklung einer smarten Bewässerungslösung mit Web-Anbindung
	}}}
\end{center}

\vspace{1em}
\vspace{1em}
\vspace{1em}

\begin{center}
	\Large{Studienarbeit}
\end{center}

\vspace{1em}

\begin{center}
	im Rahmen des \\
	\large{\textbf{Bachelor of Science (B.Sc.)}} 
\end{center}

\vspace{1em}

\begin{center}
	des Studiengangs Informatik Cyber Security \\
	der Dualen Hochschule Baden-Württemberg Mannheim
\end{center}

\vspace{1em}
\vspace{1em}

\begin{center}
	vorgelegt von
\end{center}

\begin{center}
	\textbf{Maximilian Schüller, Fynn Thierling, Justus Siegert,\\Lukas Maier, Timon Kleinknecht}
\end{center}

\vspace{1em}
\vspace{1em}

\begin{center}
	\today
\end{center}
\end{tabular}
\end{center}
	
	\cleardoublepage
	\pagenumbering{roman}
	
	%------------- Erklärung der Eigenleistung-----------

	\pagebreak
\hspace{0pt}
\vfill
\begin{center}
    \large{Erklärung der Eigenleistung}
\end{center}
\vspace{1em}
\begin{center}
    \textit{Hiermit erklären wir, dass wir die vorliegende Studienarbeit selbstständig und ohne fremde Hilfe verfasst haben. Wir haben keine anderen als die angegebenen Quellen und Hilfsmittel benutzt. Darüber hinaus erklären wir, dass im Rahmen des Schreibprozesses KI-gestützte Werkzeuge (ChatGPT) zur Umformulierung von Textstellen verwendet wurden. Wir bestätigen hiermit, dass alle verwendeten Quellenangaben korrekt sind und die inhaltliche Verantwortung für die Arbeit uneingeschränkt bei uns liegt.}
\end{center}

\begin{tabular}{>{\centering\arraybackslash}p{0.5\textwidth} >{\centering\arraybackslash}p{0.5\textwidth}}
  \includegraphics[height=2\baselineskip, keepaspectratio]{img/MS_Unterschrift.png}
  &
  \includegraphics[height=2\baselineskip, keepaspectratio]{img/FT_Unterschrift.png}
  \\
  Maximilian Schüller & Fynn Thierling \\
  \includegraphics[height=2\baselineskip, keepaspectratio]{img/JS_Unterschrift.png}
  &
  \includegraphics[height=2\baselineskip, keepaspectratio]{img/LM_Unterschrift.jpg}
  \\
  Justus Siegert & Lukas Maier \\
  \includegraphics[height=2\baselineskip, keepaspectratio]{img/Unterschrift_TK.jpg}
  &
  \\
  Timon Kleinknecht & Mannheim, 15.04.2025
\end{tabular}


\vfill
\hspace{0pt}
\pagebreak

	\newpage

	%------------- Abstract -------------
	% Abstract in English
\section*{Abstract}
\addcontentsline{toc}{section}{Abstract}

\newpage

% Abstract in Deutsch
\section*{Abstrakt}
\addcontentsline{toc}{section}{Abstrakt}


	\newpage
	
	%------------- Inhaltsverzeichnis -------------
	\tableofcontents
	
	%------------- Abkürzungsverzeichnis -------------
	%%Überschrift "Abkürzungsverzeichnis" setzen
\section*{Abkürzungsverzeichnis}
\addcontentsline{toc}{section}{Abkürzungsverzeichnis}
\begin{acronym}[STRIDE]
	%\acro{EXP}{example}	-> Im text verwenden mit \ac{EXP}	
	
\end{acronym}
	\printacronyms[template=supertabular]
    \addcontentsline{toc}{chapter}{Abkürzungen}
    \newpage
	
	%------------- Abbildungsverzeichnis -------------
	\section*{Abbildungsverzeichnis}
	\addcontentsline{toc}{section}{Abbildungsverzeichnis}
	\renewcommand{\listfigurename}{} % Verhindert doppelten großen Titel
	\newpage

	
	%------------- Tabellenverzeichnis -------------
	\section*{\listtablename}
	\addcontentsline{toc}{section}{\listtablename}
	\renewcommand{\listtablename}{} % Verhindert doppelten großen Titel
	\newpage

	
	% pagestyle für restliches Dokument aktivieren
	\pagestyle{fancy}
	\pagenumbering{arabic}
	


	%------------- Einleitung -------------
	\chapter{Einleitung}
	\section{Motivation}
\label{sec:Motivation}

Pflanzen gehören in Deutschland und Europa fest zum Alltag in Wohnung und Garten. Laut einer repräsentativen Umfrage aus dem Jahr 2020 besitzen rund drei Viertel der Bundesbürger:innen (74\,\%) Zimmerpflanzen in ihrem Zuhause; auch auf Balkonen (35\,\%), Terrassen (30\,\%) und Fensterbänken (21\,\%) grünt es, während nur etwa 10\,\% ganz ohne Pflanzen leben\autocite{pflanzenbesitz_de}. Dieses „grüne Zuhause“ liegt im Trend und gewann insbesondere während der COVID-19-Pandemie an Bedeutung – viele Menschen entdeckten 2020 im Home-Office ihre Liebe zu Haus- und Gartenpflanzen neu\autocite{pflanzenbesitz_de}. Entsprechend stieg der Absatz: Der deutsche Markt für Blumen und Zierpflanzen erreichte nach Jahren der Stagnation 2020 ein Rekordvolumen von 9{,}4\,Mrd.\,€\autocite{stihl_gartenbarometer}. Ähnlich hohe Werte zeigen sich europaweit, wo Pflanzen als wichtiger Teil der Wohn- und Lebensqualität gelten. Neben dekorativen Aspekten werden Zimmer- und Gartenpflanzen aufgrund positiver Effekte wie besserer Luftqualität und Stressreduktion geschätzt\autocite{pflanzenbesitz_de}. Die hohe Verbreitung und Wertschätzung von Pflanzen in Privathaushalten bildet den Ausgangspunkt für die Betrachtung, wie ihre Pflege im Alltag unterstützt werden kann.

Allerdings stehen viele Pflanzenbesitzer:innen vor praktischen Herausforderungen bei der Pflege ihres „grünen Mitbewohners“. Im hektischen Alltag wird das Gießen leicht vergessen oder unregelmäßig vorgenommen; umgekehrt gießen unerfahrene Halter oft zu viel aus Sorge um die Pflanze. Studien bestätigen, dass Überwässerung der häufigste Grund für das Eingehen von Zimmerpflanzen ist\autocite{pflanzenpflege_fehler}. Generell erfordert jede Pflanzenart spezifische Kenntnisse zu Wasser- und Nährstoffbedarf, Lichtverhältnissen etc., über die im privaten Umfeld nicht immer ausreichend Wissen vorhanden ist. So gaben in einer Umfrage lediglich 37\,\% der befragten Frauen und 20\,\% der Männer an, einen „grünen Daumen“ zu haben\autocite{pflanzenbesitz_de} – die Mehrheit traut sich die optimale Pflanzenpflege also eher nicht zu. Hinzu kommt, dass während Urlaubs- oder Abwesenheitszeiten oft keine Betreuung für die heimischen Gewächse sichergestellt ist. Tatsächlich vermissten in einer Befragung 26\,\% der Pflanzenhalter:innen ihre Zimmerpflanzen im Urlaub sogar mehr als die Kolleg:innen\autocite{pflanzenbesitz_de}, was die emotionale Bindung und zugleich das Problem der Versorgung in dieser Zeit verdeutlicht. Diese Pflegeherausforderungen führen dazu, dass viele privat gehaltene Pflanzen Schäden nehmen oder vorzeitig absterben.

Die Folgen von falscher oder unregelmäßiger Pflege sind in Zahlen beträchtlich. Hochrechnungen zufolge überlebt ein erheblicher Teil der gekauften Zierpflanzen nicht lange: Etwa 40\,\% der Pflanzen gehen bereits in der Lieferkette zugrunde, und weitere rund 35\,\% sterben später in den Wohnungen der Kundschaft\autocite{pflanzensterben_statistik}. Mit anderen Worten wird fast die Hälfte aller gekauften Haus- und Gartenpflanzen letztlich aufgrund suboptimaler Bedingungen oder Pflegefehler nicht dauerhaft erhalten. Auch Verbraucherumfragen deuten auf dieses Problem hin. Beispielsweise gab über ein Drittel der Hobbygärtner in einer aktuellen Erhebung an, jedes Jahr ein bis zwei Zimmerpflanzen zu verlieren\autocite{pflanzensterben_statistik}. Solche Verluste sind nicht nur emotional enttäuschend für Pflanzenliebhaber, sondern bedeuten auch Ressourcenverschwendung – insbesondere von Wasser, Zeit und Geld. Schätzungen aus den USA zeigen etwa, dass jüngere „Plant Parents“ im Durchschnitt schon mehrere ihrer erworbenen Pflanzen unbeabsichtigt zum Eingehen gebracht haben. Diese Zahlen unterstreichen die Notwendigkeit, neue Wege zu finden, um häufige Pflegefehler zu vermeiden und die Lebensdauer der Pflanzen zu verlängern.

Technologische Lösungen im Sinne von \textit{Smart Gardening} setzen hier an und versprechen Abhilfe. Insbesondere automatische Bewässerungssysteme für den Heimgebrauch bieten die Möglichkeit, den Gießvorgang zu optimieren und zu automatisieren. Solche Systeme kombinieren oft Sensoren (etwa für Bodenfeuchte oder Licht) mit internetfähigen Steuerungen, um den Pflanzen exakt bei Bedarf und in der richtigen Menge Wasser zuzuführen. Erste Ansätze sind bereits auf dem Markt verfügbar – von App-gesteuerten Bewässerungscomputern bis hin zu smarten Pflanzentöpfen mit Selbstbewässerungs-Funktion. Die Akzeptanz solcher \textit{Smart-Home}-Technologien im Garten- und Pflanzenbereich steigt kontinuierlich. Laut dem STIHL-Gartenbarometer 2022 nutzen bereits rund 7\,\% der deutschen Gartenbesitzer smarte Garden-Lösungen, und etwa 30\,\% wünschen sich zukünftig solche automatisierten Helfer\autocite{stihl_gartenbarometer}. Dabei stehen Bewässerungsautomationen an erster Stelle der Wunschliste: 83\,\% der Befragten mit Smart-Gardening-Interesse nennen ein automatisches Bewässerungssystem als besonders gefragte Lösung. Diese Nachfrage spiegelt sich auch in anderen Ländern wider. Beispielsweise glauben in Österreich über 60\,\% der Gartenbesitzer, dass sich der Wasserverbrauch durch automatisierte Bewässerungsanlagen deutlich optimieren lässt. Moderne Systeme können Wetterdaten oder Bodensensoren einbeziehen, um nur dann zu wässern, wenn die Pflanze es wirklich benötigt – eine Technik, die den Pflanzenstress reduziert und zugleich Wasserverschwendung vorbeugt. Aktuelle Untersuchungen zeigen denn auch, dass intelligente Bewässerungssteuerungen den Wasserverbrauch im Garten gegenüber herkömmlichen Timern erheblich senken können (um etwa 20–40\,\% je nach System).

Mehrere übergeordnete Trends begünstigen die Verbreitung von smarten Pflanzenpflege-Systemen. Zum einen führt die Urbanisierung dazu, dass immer mehr Menschen auf kleinem Raum in Städten leben – in Deutschland etwa 78\,\% der Bevölkerung\autocite{urbanisierung_de} – und sich dennoch nach Natur im eigenen Umfeld sehnen. Insbesondere Stadtbewohner ohne Garten kultivieren vermehrt Zimmerpflanzen oder Balkongrün, sind aber beruflich oft stark eingebunden. Eine automatische Bewässerung kann hier den Pflegeaufwand mindern und sicherstellen, dass Pflanzen trotz hektischem Alltag oder Abwesenheiten ausreichend versorgt werden. Zum anderen rückt Nachhaltigkeit in den Fokus: Wassermanagement und effiziente Ressourcennutzung gewinnen an Bedeutung, da die Auswirkungen des Klimawandels – etwa häufigere Sommerdürreperioden – auch private Gärten und Balkone betreffen. In Umfragen äußern fast zwei Drittel der Befragten die Erwartung, dass digitale Technologien im Garten helfen können, den Klimawandel abzuschwächen, und nennen den schonenden Umgang mit Wasser als oberste Priorität. Smart-Bewässerungssysteme erfüllen genau diesen Zweck, indem sie bedarfsgerecht gießen und Überwässerung verhindern. Schließlich trägt auch die allgemeine Verbreitung von \textit{Internet of Things}-Anwendungen im Haushalt dazu bei, dass vernetzte Lösungen immer selbstverständlicher werden. Der europäische Smart-Home-Markt verzeichnet hohe Wachstumsraten und wird 2024 bereits auf über 22\,Mrd.\,US-\$ geschätzt\autocite{iot_trend}. Vernetzte, per App oder Sprache steuerbare Geräte – vom Thermostat bis zur Lichtsteuerung – gehören zunehmend zum Alltag. Diese Entwicklung macht auch vor dem Bereich der Pflanzenpflege nicht Halt: Die Nutzerakzeptanz für digitale Helfer im Haushalt schafft ein günstiges Umfeld für \textit{Smart Gardening}-Innovationen.

Insgesamt ist die Einführung eines smarten Bewässerungssystems im heimischen Umfeld vor dem Hintergrund dieser Fakten sowohl technisch zeitgemäß als auch gesellschaftlich sinnvoll. Die weit verbreitete Haltung von Zimmer- und Gartenpflanzen einerseits und die häufig auftretenden Pflegeprobleme andererseits schaffen ein deutliches Bedürfnis nach Unterstützung. Automatisierte Bewässerungslösungen können hier einen doppelten Nutzen stiften: Sie helfen Pflanzenbesitzer:innen, ihre grünen Schützlinge zuverlässig und fachgerecht zu versorgen, und tragen zugleich zu Nachhaltigkeit und Komfort bei. Indem ein smartes Bewässerungssystem Wasser bedarfsgerecht dosiert und den Pflegeprozess vereinfacht, steigert es die Überlebensrate und Vitalität der Pflanzen und entlastet den Menschen von Routineaufgaben. Die vorliegenden Studien, Statistiken und Trends untermauern somit die Notwendigkeit und den Nutzen eines solchen Systems, das im Folgenden technisch konzipiert und beschrieben wird.

	\newpage
	\section{Zielsetzung}
\label{sec:Zielsetzung}

Ziel dieser Studienarbeit ist die Konzeption, Entwicklung und prototypische Umsetzung eines automatisierten Systems zur Bewässerung von Zimmerpflanzen im privaten Wohnumfeld. Es soll eine lauffähige Gesamtlösung entstehen, die aus einem Mikrocontroller als zentrale Steuereinheit, einer Backend-Infrastruktur zur Datenverarbeitung und -persistierung sowie einem benutzerfreundlichen Frontend zur Visualisierung und Steuerung besteht. Die Realisierung erfolgt im Rahmen eines \textit{Proof of Concept}, der die technische Machbarkeit sowie die Integration der Systemkomponenten demonstriert.
\\
Das zu entwickelnde System erfasst über geeignete Sensorik (z.\,B. Bodenfeuchte, Temperatur, Luftfeuchtigkeit, Lichtintensität) kontinuierlich relevante Umgebungsdaten. Diese Messwerte dienen entweder als Entscheidungsgrundlage für den Nutzer, um über die Benutzeroberfläche manuell eine Bewässerung anzustoßen, oder sie werden vom Mikrocontroller automatisch verarbeitet. Im letzteren Fall wird anhand zuvor definierter Sollwerte eine autonome Steuerung der Bewässerungseinheit realisiert. Vorrangiges Ziel ist die technische Umsetzung des automatischen Betriebsmodus. Die Konzeption und Entwicklung des manuellen Modus sowie die Integration beider Steuerungsarten in das Gesamtsystem erfolgen nachrangig und abhängig von den im Projektverlauf verfügbaren Entwicklungskapazitäten.
\\
Die Bewässerungslösung ist primär für den Einsatz in Innenräumen konzipiert. Dies umfasst insbesondere Haushalte mit Zimmerpflanzen, bei denen typische Pflegeprobleme wie unregelmäßiges Gießen oder Unsicherheit bezüglich des Wasserbedarfs adressiert werden sollen.
\\
Der konkrete Funktionsumfang des Systems wird im Verlauf des Projekts iterativ entwickelt. Eine detaillierte Beschreibung der funktionalen und nicht-funktionalen Anforderungen sowie der Zielsystemeigenschaften erfolgt in Kapitel~\ref{chap:Anforderungen}. Dabei wird angestrebt, etablierte \textit{Best Practices} der Software- und Systementwicklung zu berücksichtigen und – wo sinnvoll und realisierbar – aktuelle Technologien gemäß dem Stand der Technik (\textit{State of the Art}) zu verwenden. Gleichzeitig wird die technische Umsetzung unter Berücksichtigung der konzeptionellen Natur als \textit{Proof of Concept} gewichtet, sodass pragmatische Abwägungen hinsichtlich Komplexität, Aufwand und Ressourcen erfolgen.
\\
Insgesamt dient die Arbeit dem Ziel, ein funktional überzeugendes Demonstrationssystem zu realisieren, das eine fundierte Grundlage für weiterführende Entwicklungen, Evaluationen oder mögliche Produktivsetzungen bietet.

	\newpage
	\section{Ziel der Arbeit}
\label{sec:Ziel der Arbeit}


	\newpage
	
	% ------------- Hauptteil -------------

	\chapter{Theoretische Grundlagen}
	\section{Scrum und das Agile Manifest}
Scrum basiert auf den Grundsätzen des Agilen Manifests. Das Agile Manifest ist eine Sammlung von Priorisierungsprinzipien, die im Jahr 2001 von 17 Experten und Vertretern unterschiedlicher agiler Vorgehensweisen in Snowbird, USA, entwickelt wurde. Diese Gruppe, oft als die \glqq Snowbird 17\grqq{} bezeichnet, erkannte frühzeitig die Notwendigkeit einer neuen Ära der Softwareentwicklung. \cite{Drumond} Das Agile Manifest umfasst im Original nur 68 Wörter, aber aus diesem knappen Text wurden allgemeingültige Prinzipien abgeleitet, die bis heute die Grundlage für agile Methoden wie Scrum bilden.

Das Manifest lautet sinngemäß ins Deutsche übersetzt: 

\vspace{1em}
\glqq Wir erschließen bessere Wege, Software zu entwickeln, indem wir es selbst tun und anderen dabei helfen. Durch diese Tätigkeit haben wir diese Werte zu schätzen gelernt:

\noindent \textbf{Individuen und Interaktionen}
\vspace{-1em}
\begin{flushright}
mehr als Prozesse und Werkzeuge,
\end{flushright}
\vspace{-1em}
\textbf{Funktionierende Software}
\vspace{-1em}
\begin{flushright}
mehr als umfassende Dokumentation,
\end{flushright}
\vspace{-1em}
\textbf{Zusammenarbeit mit dem Kunden}
\vspace{-1em}
\begin{flushright}
mehr als Vertragsverhandlung,
\end{flushright}
\vspace{-1em}
\textbf{Reagieren auf Veränderung}
\vspace{-1em}
\begin{flushright}
mehr als das Befolgen eines Plans.
\end{flushright}

Das heißt, obwohl wir die Werte auf der rechten Seite wichtig finden, schätzen wir die Werte auf der linken Seite höher ein.\grqq{} \cite{Snowbird2001}

\subsection{Interpretation des Agilen Manifests in der Praxis}
Das Agile Manifest betont die Wichtigkeit von Praxisnähe in der agilen Entwicklung. Es reicht nicht aus, nur theoretische Konzepte zu entwickeln; vielmehr müssen praktische Erfahrungen gesammelt und in den Entwicklungsprozess eingebracht werden. Ein zentrales Prinzip ist, dass die Individuen und ihre Interaktionen im Vordergrund stehen. Dies bedeutet, dass ein Vorgehen gefunden werden muss, das eine effektive Kommunikation und Interaktion aller Beteiligten ermöglicht. Prozesse und Werkzeuge sollten an die Bedürfnisse der Menschen angepasst werden, nicht umgekehrt.

Darüber hinaus wird die Bedeutung funktionierender Software hervorgehoben. Der Fortschritt eines Projekts wird anhand der tatsächlich funktionierenden Software gemessen, nicht anhand umfangreicher Dokumentationen. Es ist entscheidend, dass die entwickelte Software in regelmäßigen Abständen gezeigt und von den Anwendern beurteilt wird. Dies stellt sicher, dass das Projekt auf dem richtigen Weg bleibt und die Anforderungen der Nutzer erfüllt.

Da Softwareentwicklung ein dynamischer Prozess ist, entstehen oft neue Herausforderungen oder Anforderungen, selbst bei sorgfältiger Planung. Daher ist es unerlässlich, flexibel auf Veränderungen zu reagieren. Der Erfolg eines Projekts wird daran gemessen, wie gut es sich an neue Gegebenheiten anpasst und ob daraus ein Lerneffekt resultiert, der das Projekt voranbringt.

Auch wenn das Manifest die Bedeutung von Prozessen, Dokumentation, Verträgen und Plänen anerkennt, stellt es klar, dass diese Aspekte im Vergleich zu den übergeordneten Prinzipien von geringerer Priorität sind. Sie haben jedoch weiterhin ihre Daseinsberechtigung und müssen in einem angemessenen Maße berücksichtigt werden. \cite{Wolf2011}

\subsection{Zusammenhang zwischen Scrum und dem Agilen Manifest}
Scrum operationalisiert die Prinzipien des Agilen Manifests in einem strukturierten Rahmenwerk. Die regelmäßigen Sprints und die damit verbundenen Meetings – wie das Sprint Planning, Daily Stand-ups, Sprint Reviews und Retrospektiven – fördern die Kommunikation und die Interaktion zwischen den Teammitgliedern und den Stakeholdern. Durch die iterative Natur von Scrum wird sichergestellt, dass funktionierende Software frühzeitig und kontinuierlich geliefert wird, wodurch die Kundenzufriedenheit gesteigert wird.

Die enge Zusammenarbeit mit dem Kunden, die in Scrum durch die Rolle des Product Owners verkörpert wird, gewährleistet, dass das Entwicklungsteam ständig auf die sich ändernden Anforderungen reagieren kann. Diese Flexibilität ist ein direkter Ausdruck des Wertes \glqq Reagieren auf Veränderung mehr als das Befolgen eines Plans\grqq, der im Agilen Manifest verankert ist.


\subsection{Anwendung von Scrum im Sensora-Projekt}
Aus den Grundsätzen des Agilen Manifests ergeben sich spezifische Herangehensweisen für das Sensora-Projekt:
\begin{description}
    \item[Zwischenergebnisse:] Es wird eine hohe Frequenz bei der Präsentation von Zwischenergebnissen angestrebt. Diese regelmäßigen Präsentationen bieten eine hervorragende Gelegenheit, um mit den Stakeholdern in den Dialog zu treten, Feedback zu sammeln und Verbesserungen zu identifizieren. Zudem fungieren diese Präsentationen als Indikatoren für den Projektfortschritt, wodurch erkennbar wird, ob das Projekt planmäßig voranschreitet oder ob Maßnahmen zur Kurskorrektur erforderlich sind.
    \item[Kommunikation:] Alle technischen Entscheidungen werden in enger Abstimmung mit allen Entwicklern getroffen. Dabei werden die Meinungen und Bedenken der beteiligten Personen berücksichtigt, um sicherzustellen, dass realistische Lösungen verfolgt werden. Durch diesen intensiven Austausch wird verhindert, dass Zeit und Ressourcen in ineffiziente oder unpraktikable Lösungen investiert werden. Gleichzeitig wird sichergestellt, dass das kollektive Wissen genutzt wird und potenzielle Probleme frühzeitig erkannt werden.
    \item[Lessons Learned:] Im Verlauf der Entwicklung entstehen neue Erkenntnisse, die zu neuen Möglichkeiten führen. Diese Lernfortschritte, sowohl auf fachlicher als auch auf technischer Ebene, werden genutzt, um das Projekt kontinuierlich weiterzuentwickeln und anzupassen. Die Fähigkeit, aus Erfahrungen zu lernen und diese in den Entwicklungsprozess einzubeziehen, ist ein zentraler Bestandteil des agilen Vorgehens.
\end{description}

Zusammengefasst projiziert Scrum die Werte und Prinzipien des Agilen Manifests auf einen praxisnahen und strukturierten Entwicklungsprozess, der es Teams ermöglicht, effizient und flexibel auf die Herausforderungen der Softwareentwicklung zu reagieren. Durch die Integration dieser Prinzipien in das Sensora-Projekt wird sichergestellt, dass das Produkt den Anforderungen gerecht wird und gleichzeitig eine hohe Qualität und Benutzerfreundlichkeit erreicht.

	
\section{REST APIs: Grundlagen und Best Practices}
\ac{rest} \ac{api} ist ein Architekturstil für verteilte Systeme, insbesondere für Webanwendungen. Er wurde erstmals im Jahr 2000 von Roy Thomas Fielding in seiner Dissertation eingeführt. \ac{rest} definiert eine Reihe von Prinzipien, die die Interaktion zwischen Clients und Servern in einem verteilten System standardisieren und vereinfachen sollen. Obwohl es keine offizielle Spezifikation wie einen \ac{rfc} oder eine ISO-Norm für \ac{rest} gibt, hat sich der Architekturansatz in der Praxis durchgesetzt und bildet die Grundlage für viele der heutigen Web-\acp{api}.

\ac{rest} basiert auf dem Prinzip, dass ein Webdienst über eine standardisierte Schnittstelle (\ac{api}) ansprechbar ist, bei der die Kommunikation zwischen Client und Server zustandslos ist. Das bedeutet, dass jede Anfrage vollständig ist und keine Informationen über den vorherigen Zustand benötigt werden. Diese Eigenschaft macht \ac{rest}-\acp{api} besonders skalierbar und flexibel.

\subsection{Warum REST?}

\ac{rest} hat sich gegenüber anderen Architekturansätzen wie \ac{soap} aus mehreren Gründen durchgesetzt. \ac{rest}-\acp{api} sind leichter zu implementieren und zu nutzen, da sie auf den bestehenden HTTP-Standards aufbauen. \ac{rest} nutzt die standardmäßigen HTTP-Verben (GET, POST, PUT, DELETE), um \ac{crud}-Operationen auf Ressourcen durchzuführen. Diese Einfachheit und die Nutzung bewährter Webstandards machen \ac{rest}-\acp{api} besonders attraktiv für Webanwendungen und mobile Apps, wo schnelle Entwicklung und hohe Performance entscheidend sind.

Ein weiterer Vorteil von \ac{rest} ist seine Flexibilität und die Möglichkeit zur Integration in eine Vielzahl von Plattformen und Programmiersprachen. Da \ac{rest}-\acp{api} auf dem HTTP-Protokoll basieren, können sie in nahezu jeder Umgebung eingesetzt werden, die HTTP unterstützt, was zu einer breiten Akzeptanz und Nutzung geführt hat.

\subsection{Grundlagen einer REST API}

Obwohl es kein formales Regelwerk für \ac{rest} gibt, haben sich in der Entwicklergemeinschaft einige Best Practices etabliert, die eine \ac{rest}-\ac{api} als gut definieren. Diese Best Practices sind weitgehend anerkannt und werden häufig in der Praxis angewendet:
\begin{description}
    \item[\ac{json} als Standardformat verwenden:] \ac{rest}-\acp{api} sollten \ac{json} als Standardformat für die Datenübertragung verwenden, da \ac{json} leichtgewichtig, gut lesbar und in den meisten Programmiersprachen nativ unterstützt wird.
    \item[Verwendung von Substantiven in Endpunktpfaden:] Endpunktpfade sollten Substantive anstelle von Verben verwenden, um die Ressource zu definieren, auf die die Operation angewendet wird. Beispielsweise sollte der Endpunkt \code{/users} anstelle von \code{/getUsers} verwendet werden.
    \item[Logische Verschachtelung von Endpunkten:] Endpunkte sollten logisch verschachtelt sein, um die Hierarchie der Daten widerzuspiegeln. Zum Beispiel könnte ein Endpunkt für die Bestellungen eines Benutzers \code{/users/{userId}/orders} lauten.
    \item[Fehlerbehandlung und Standard-HTTP-Fehlercodes:] Eine gute \ac{rest}-\ac{api} sollte standardisierte HTTP-Fehlercodes verwenden, um dem Client klare Rückmeldungen über den Status der Anfrage zu geben. Beispielsweise steht der Fehlercode 404 für \glqq Nicht gefunden \grqq{} und 500 für \glqq Interner Serverfehler \grqq.
    \item[Filtern, Sortieren und Paginierung:] \ac{rest}-\acp{api} sollten die Möglichkeit bieten, Ergebnisse zu filtern, zu sortieren und zu paginieren, um die Rückgabemenge zu steuern und die Effizienz zu erhöhen.
    \item[Sicherheitspraktiken:] \ac{rest}-\acp{api} sollten sichere Authentifizierungs- und Autorisierungsmechanismen verwenden, wie OAuth2 oder \ac{jwt}, um sicherzustellen, dass nur berechtigte Benutzer auf die Ressourcen zugreifen können.
    \item[Daten-Caching:] Um die Leistung zu verbessern, sollten \ac{rest}-\acp{api} Daten zwischenspeichern, wo es sinnvoll ist. Dies kann die Ladezeiten reduzieren und die Last auf dem Server verringern.
    \item[\ac{api}-Versionierung:] Eine gute \ac{rest}-\ac{api} sollte versioniert werden, um Änderungen und Verbesserungen an der \ac{api} zu ermöglichen, ohne bestehende Clients zu beeinträchtigen.
\end{description}

Diese Best Practices bilden die Grundlage für die Entwicklung robuster und skalierbarer \ac{rest}-\acp{api}. Sie wurden von der Entwicklergemeinschaft, beispielsweise auf Plattformen wie StackOverflow \cite{JohnAuYeung2020}, breit akzeptiert und weiter verfeinert.

\subsection{Vertiefende Empfehlungen und Designansätze}
Neben diesen grundlegenden Prinzipien bietet das Buch \glqq\ac{rest} \ac{api} Design Rulebook\grqq{} von Mark Masse \cite{Masse2011} eine weitergehende Sammlung von Designregeln. Diese basieren auf den ursprünglichen Prinzipien von Fielding und wurden im Laufe der Zeit durch die praktische Erfahrung ergänzt und weiterentwickelt. Masse betont beispielsweise, dass eine \ac{rest}-\ac{api} entworfen und nicht einfach nur codiert wird. Der Entwurfsprozess sollte klar strukturierte Ressourcen und deren Beziehungen beinhalten, um eine konsistente und verständliche \ac{api} zu schaffen.

Darüber hinaus wird empfohlen, eine \ac{rest}-\ac{api} grafisch zu visualisieren, um Entwicklern und Nutzern der \ac{api} eine klare Vorstellung von den verfügbaren Endpunkten und deren Beziehungen zu geben. Diese Visualisierung erleichtert das Verständnis der \ac{api} und fördert die Konsistenz in der Implementierung.

\subsection{Umsetzung für Sensora}
Da Sensora von vielen verschiedenen Softwarelösungen der Sensora-Community genutzt werden soll, ist eine gut durchdachte \ac{rest}-\ac{api} erforderlich, die den oben genannten Best Practices folgt. Die \ac{api} wird gemäß dem OpenAPI-Standard dokumentiert, was allen Entwicklern detaillierte Informationen über die Funktionsweise und Möglichkeiten bietet. Diese Standardisierung erleichtert die Implementierung und fördert die Konsistenz in der Nutzung der \ac{api}.

Zur Visualisierung und Dokumentation der \ac{api} wird Swagger verwendet, ein weit verbreitetes Tool, das es ermöglicht, die API grafisch darzustellen und interaktive Dokumentationen zu erstellen. Dies stellt sicher, dass Entwickler schnell und einfach auf die notwendigen Informationen zugreifen können, um die Sensora-\ac{api} effizient in ihre Anwendungen zu integrieren.

Die sorgfältige Umsetzung der \ac{rest}-Prinzipien in der Sensora-\ac{api} wird dazu beitragen, eine robuste, flexible und benutzerfreundliche Schnittstelle zu schaffen, die den Produktanforderungen gerecht wird und eine breite Akzeptanz unter den Entwicklern der Sensora-Community findet.

	
\section{Designprinzipien und -muster}
Die Entwicklung von Sensora basiert auf einer klar definierten und bewährten Architektur, die die spezifischen Anforderungen an eine flexible, skalierbare und sichere Benachrichtigungsplattform erfüllt. Im Rahmen dieser Entwicklung wurden verschiedene Designprinzipien und -muster berücksichtigt, die sicherstellen, dass das System nicht nur leistungsfähig, sondern auch leicht wartbar und zukunftssicher ist.

\subsection{Prinzipiengeleitetes Design}
Sensora wurde unter strikter Beachtung der Coding-Guidlines entwickelt, einer Reihe von spezifischen Richtlinien, auf die sich die Entwickler geeinigt haben. Diese Guidlines werden an oberster Stelle während des gesamten Entwicklungsprozesses befolgt, um eine einheitlich hohe Qualität in der Architektur und im Code zu gewährleisten. Die \ac{api} wurde gemäß den Best Practices gestaltet, um eine nahtlose Integration in die Softwarelandschaft des Gesamtprodukts zu gewährleisten. Diese Integration wurde durch die strikte Einhaltung der Prinzipien zur \ac{api}-Entwicklung, einschließlich der Nutzung von \ac{rest} für synchrone und asynchronen Kommunikation, ermöglicht. Zudem wurden Sicherheitsprinzipien, insbesondere das Zero Trust-Modell, konsequent umgesetzt, wodurch jede Anfrage eines Clients neu authentifiziert wird.

\subsection{Coding-Guidlines}
Die Coding-Guidlines stellen ein umfassendes Regelwerk dar, das die Grundlage für die Softwareentwicklung bildet. Diese Prinzipien sind mehr als nur Richtlinien; sie sind integraler Bestandteil der Projektkultur und stellen sicher, dass alle Entwicklungsprojekte nach denselben hohen Standards durchgeführt werden. Die Coding-Guidlines decken eine Vielzahl von Aspekten ab, von der Architektur über die Server-Nutzung bis hin zur Sicherheit und Best Practices für spezifische Programmiersprachen wie Python oder Rust.

Ein zentrales Prinzip ist das der losen Kopplung. Es fordert, dass Anwendungen nur über sprachunabhängige Protokolle miteinander kommunizieren. Dies fördert die Modularität und erleichtert die Wartung sowie die Integration neuer Systeme. In Sensora wird dies durch die strikte Trennung der Module und die Nutzung von klar definierten Schnittstellen erreicht. Jedes Modul ist eigenständig und kommuniziert über definierte \acp{api}, was die Austauschbarkeit und Wiederverwendbarkeit der Komponenten sicherstellt.

Das Zero Trust-Sicherheitsmodell ist ein weiteres kritisches Element der Coding-Guidlines. Es besagt, dass keine Entität – weder Benutzer noch Gerät oder Netzwerk – automatisch vertraut wird, unabhängig davon, ob sie sich innerhalb oder außerhalb des Netzwerks befindet. Stattdessen wird jede Anfrage verifiziert. Authentifizierung und Autorisierung erfolgen kontinuierlich. Zugriffe werden auf das Minimum beschränkt, basierend auf dem Prinzip der geringsten Privilegien. Diese Sicherheitsanforderung wurde in Sensora durch die Integration von \acp{jwt} umgesetzt, sodass jede Anfrage an einen Service strengen Authentifizierung unterliegt.

Abgerundet werden die Coding-Guidlines durch eine starke Fokussierung auf Best Practices und Clean Code. Diese umfassen spezifische Regeln für die Nutzung von Rust, wie beispielsweise die Bevorzugung von etablierten Bibliotheken wie \code{acti\_web}, die Verwendung von \code{match}-Ausdrücken statt verschachtelter \code{if}-Bedingungen und die Anwendung des \code{?}-Operators zur Verbesserung der Lesbarkeit und Sicherheit des Codes. Diese Praktiken tragen dazu bei, dass der Code von Sensora nicht nur funktional, sondern auch wartbar und erweiterbar ist.

Durch die strikte Beachtung der Coding-Guidlines bei der Entwicklung von Sensora konnte ein System geschaffen werden, das nicht nur den hohen technischen Anforderungen entspricht, sondern auch die langfristigen Ziele in Bezug auf Nachhaltigkeit, Sicherheit und Effizienz unterstützt. Diese Prinzipien sind somit ein wesentlicher Bestandteil der Architektur und des Designs von Sensora und bilden das Fundament für alle getroffenen Entscheidungen während der Entwicklung.

\subsection{Event-Driven Architecture}
Ein zentrales architektonisches Muster, das bei der Entwicklung von Sensora bewusst gewählt wurde, ist die Event-Driven Architecture. Dieses Muster eignet sich besonders gut für die Verarbeitung von Messungen, die asynchron generiert und verteilt werden müssen. Die Entscheidung für eine Event-Driven Architecture ermöglicht es, auf Ereignisse in Echtzeit zu reagieren und Daten effizient zu verteilen, ohne die Performance des Systems zu beeinträchtigen. Solace, das als Messaging-System innerhalb von Sensora eingesetzt wird, spielt hierbei eine entscheidende Rolle, indem es stabile und zuverlässige asynchrone Kommunikation sicherstellt.

\subsection{Herausforderungen und Lösungen}
Eine der zentralen Herausforderungen bei der Implementierung der Event-Driven Architecture war die asynchrone Benachrichtigung der Clients. Diese Herausforderung wurde erfolgreich durch die Integration von Solace gemeistert, das als stabiles und zuverlässiges Messaging-System fungiert. Ein weiteres potenzielles Problem, nämlich die Handhabung von Zugriffskollisionen bei gleichzeitigen Datenbankzugriffen, wurde durch die Verwendung von PostgreSQL adressiert. PostgreSQL bietet ein fortschrittliches Session-Management, das automatisch Kollisionen bei gleichzeitigen Zugriffen verwaltet, sodass Sensora selbst keine zusätzlichen Mechanismen zur Kollisionsvermeidung implementieren musste.

\subsection{Zusammenfassung}
Die Entwicklung von Sensora basiert auf einer durchdachten Kombination aus bewährten Designprinzipien und modernen Architekturmustern. Durch die konsequente Anwendung der Coding-Guidlines und die Nutzung einer Event-Driven Architecture wurde ein System geschaffen, das sowohl leistungsfähig als auch flexibel ist. Die Berücksichtigung von Best Practices und die Umsetzung eines strengen Sicherheitsmodells gewährleisten, dass Sensora nicht nur den aktuellen Anforderungen gerecht wird, sondern auch zukunftssicher und erweiterbar bleibt.

	%%jegliches benötigtes theoretisches Wissen kann hier THemenweise mit Literaturrecherche dargestellt werden
\section{IoT Devices}
 Ein IoT Device ist das und das und das.
    \subsection{Home Automation}
    Home Automation ist das und das und das.
    \subsection{Smart Home}
    Smart Home ist das und das und das.
	\newpage
	\chapter{Anforderungen}
	%%definiere und analysiere hier die Anforderungen entweder an das gesamte Projekt oder eine Komponente des Projektes
\section{Anforderungen an das IoT Device}
Die Anforderungen an das IoT Device sind wie folgt: blablabla.

%hier geht es darum tatsächliche Kritierein für die Auswahl der Technologien zu definieren, die dann in der nächsten Sektion verwendet werden
\section{Anforderungsanalyse für das IoT Device}
Die Anforderungen an das IoT Device bedeuten, dass folgende Features durch das Device geleistet werden müssen:
    \subsection{Anforderung 1}
    Das IoT Device muss in der Lage sein xyz zu tun.
    \subsection{Anforderung 2}
    Das IoT Device muss die Anwendung ABC in seiner Software vorsehen/ermöglichen.

	\newpage
	\chapter{Auswahl der Technologien}
	%%Auswahl der Technologien für eine Kompüonente des Projektes, basierend auf den Anforderungen, die in der vorherigen Sektion definiert wurden (Zentral für die Arbeit)
\section{IoT Device}
Das IoT Device ist ein zentraler Bestandteil des Projektes. Es ist dafür verantwortlich, die Daten von den Sensoren zu sammeln und an die Cloud zu übertragen. In diesem Abschnitt werden die Technologien ausgewählt, die für das IoT Device verwendet werden sollen. Die Auswahl erfolgt auf Basis der Anforderungen, die in der vorherigen Sektion definiert wurden.
\subsection{Verfügbare Technologien}
Die folgenden Technologien sind verfügbar.
\subsection{State of the Art Technologie}
Die Technologie 1 ist State of the Art und wird oft für IoT Devices verwendet.
\subsection{Vergleich von Technologien}
Die Technologien erfüllen die Anforderungen wie folgt:
\begin{itemize}
    \item Technologie 1: erfüllt die Anforderungen A, B und C.
    \item Technologie 2: erfüllt die Anforderungen A, B und D.
    \item Technologie 3: erfüllt die Anforderungen A, C und D.
\end{itemize}
\subsection{Auswahl der Technologie}
Die Technologie 1 wird ausgewählt, da sie die Anforderungen A, B und C erfüllt. Die Technologie 2 wird nicht ausgewählt, da sie die Anforderungen A, B und D erfüllt, aber nicht so gut wie Technologie 1. Die Technologie 3 wird ebenfalls nicht ausgewählt, da sie die Anforderungen A, C und D erfüllt, aber nicht so gut wie Technologie 1.
%Alternativ: Die technologie wird trotz ihre Status als State of the Art nicht ausgewählt, da sie die Anforderungen nicht ausreichend erfüllt.
	\newpage
	\chapter{Umsetzung}
	%%Hier wird alles beschrieben und erklärt, was während in der Praxis passiert ist und gemacht wurde.
\section{Umsetzung des IoT Devices}
Das IoT Device wurde von einer Person als Hauptentwickler und mehreren Unterstützenden Entwicklern erstellt.
    \subsection{Hardware}:
    Für die Hardware wurde ein fertiger Bausatz von Alibaba gekauft, der alle benötigten Komponenten enthielt.
    Darunter ein ESP32, sensoren und eine Pumpe.
    Die Hardware wurde zusammengebaut und testweise in Betrieb genommen, yada yada yada.
    \subsection{Software}:
    Die SOftware wurde in C Entwickelt.
    Es wurden die Komponenten x, y und z implementiert.
    Dabei lief dies gut und das nicht so gut.
   
	\newpage
	
	% ------------- Ende Hautpteil -------------
	
	\chapter{Kritische Reflexion}
	%%Reflektiere über umsetzung der Komponenten, was hat gut funktioniert, was nicht, was war einfach, was war schwer, was war unerwartet, was war nicht so gut.
\section{IoT Device}
Das IoT Device wurde von einer Person als Hauptentwickler und mehreren Unterstützenden Entwicklern erstellt.
    \subsection{Hardware}:
    Geplant war das IoT Device als ein Gerät, das die Luftfeuchtigkeit und Temperatur misst und diese Daten an einen Server sendet.
    Für die Hardware wurde ein fertiger Bausatz von Alibaba gekauft, der alle benötigten Komponenten enthielt. Damit war es möglich, die Hardware schnell zusammenzubauen und zu testen.
    Die Sensoren sind nur mittelmäßig genau, was aber für ein Proof of Concept ausreichend ist.
    \subsection{Software}:
    Die Software wurde in C entwickelt. Dabei wurde CLion mit CMake als IDE verwendet.
    Dadurch kam es zu komplikationen im Setup, da die IDE nicht richtig konfiguriert war. Daraus resultierten verzögerungen die das Projekt belasteten
    Mehr feste Entwickler hätten hier helfen können.

Insgesamt lief die Entwicklung des IoT Devices gut, aber es gab einige Herausforderungen, die bewältigt werden mussten.
    Die Hardware war einfach zusammenzubauen, aber die Software war schwieriger zu implementieren als erwartet.
    Es gab einige Probleme mit der Kommunikation zwischen dem IoT Device und dem Server, die behoben werden mussten.
    Auch die Genauigkeit der Sensoren war nicht so hoch wie erhofft, was die Ergebnisse beeinflusste.
    Dennoch konnte das IoT Device erfolgreich entwickelt und getestet werden, und es erfüllt die Anforderungen des Projekts.
    
	\newpage 
	\chapter{Ausblick}
	%% Hier bisschen Tagträumen was in Zukunft noch so kommen könnte.
\section{IoT Device}
Das IoT Device könnte in Zukunft durch bessere Sensoren und eine bessere Software weiter verbessert werden.
Mehr Modularität könnte auch helfen, das IoT Device einfacher zu erweitern und anzupassen.
Das IoT Device könnte auch in Zukunft durch eine bessere Benutzeroberfläche und eine bessere Integration in andere Systeme weiter verbessert werden.
	\newpage
	
	%------------- Literaturverzeichnis -------------
	\newpage
	\pagestyle{fancy}
	\fancyhead[R]{LITERATUR}

	\chapter*{Literaturverzeichnis}
	\addcontentsline{toc}{chapter}{Literaturverzeichnis}
	\printbibliography
	\newpage

	%------------- Anhang -------------
	\cleardoublepage
	\fancyhead[R]{ANHANG}

	\chapter*{Anhang}
	\addcontentsline{toc}{chapter}{Anhang}

	% Hier Anhänge einfügen:
	%\input{./Anhang/sensorwerte}
	%\input{./Anhang/schaltplan}

\newpage

	
\end{document}