% ============= Package Einstellungen & Sonstiges ============= 
\documentclass[a4paper,12pt]{article}
\usepackage[left= 2.5 cm,right = 2.5 cm, bottom = 2.5 cm]{geometry}
\usepackage[onehalfspacing]{setspace}

\usepackage[
pdftitle={Srtudienarbeit},
pdfsubject={Entwicklung einer smarten Bewässerungslösung mit Web-Anbindung},
pdfauthor={Maximilian Schüller, Fynn Thierling, Justus Siegert, Lukas Maier, Timon Kleinknecht},
pdfkeywords={},	
hidelinks %Links nicht einrahmen
]{hyperref}

\usepackage[utf8]{inputenc}
\usepackage[ngerman]{babel}
\usepackage[T1]{fontenc}

\usepackage{fancyhdr}
\usepackage{color}
\usepackage{csquotes}
%\usepackage{cite}
\usepackage[backend=biber, autocite=inline, style=ieee, natbib=true]{biblatex}
\addbibresource{literatur.bib}
\DefineBibliographyStrings{ngerman}{andothers = {{et\,al\adddot}},}
\usepackage{url}

\usepackage{graphicx} %für Einbindung von Grafiken
\graphicspath{{img/}} %Pfad für Grafiken

\usepackage{pdfpages}

\usepackage{todonotes}

\usepackage[printonlyused]{acronym}

\usepackage{minted} %für Darstellung von Code
\usepackage{float}
\usepackage[german]{varioref} %für schönere Referenzierung von Abbildungen

\fancyhead[L]{} % Linke Kopfzeile leer lassen

\usepackage{xspace}

% Define a command to ensure consistent space after acronym
\newcommand{\acspace}{\xspace\hspace{1em}}

\makeatletter
\renewcommand*{\aclabelfont}[1]{\textbf{\acsfont{#1}\acspace}}
\makeatother

\newcommand{\source}[1]{\vspace{1ex}\noindent{\small \textit{Quelle: #1}}}

\newcommand{\initializeBibliography}{
	\ihead{}
	\printbibliography[title=\Literaturverzeichnis] 
	\cleardoublepage
}

\usepackage{enumitem}
\usepackage{amssymb}

\usepackage{listings}
\usepackage{xcolor}  % Optional, für zusätzliche Farbanpassungen
\lstset{ 
	language=Python,                % Programmiersprache
	basicstyle=\ttfamily\scriptsize,     % Grundschriftart (monospace) und -größe
	keywordstyle=\bfseries\color{blue}, % Schlüsselwörter fett und blau
	commentstyle=\itshape\color{green!50!black}, % Kommentare kursiv und grün
	stringstyle=\color{red},        % Strings rot
	numbers=left,                   % Zeilennummern links anzeigen
	numberstyle=\tiny\color{black},  % Stil der Zeilennummern (klein und grau)
	stepnumber=1,                   % Jede Zeile nummerieren
	numbersep=8pt,                  % Abstand der Zeilennummern zum Code
	backgroundcolor=\color{gray!10},  % Hintergrundfarbe des Codes (weiß)
	frame=single,                   % Rahmen um den Code
	rulecolor=\color{black},        % Rahmenfarbe
	captionpos=b,                   % Position des Titels (unten)
	breaklines=true,                % Zeilen umbrechen, wenn sie zu lang sind
	breakatwhitespace=true,         % Zeilenumbrüche nur an Leerzeichen
	showstringspaces=false,         % Leerzeichen in Strings nicht anzeigen
	tabsize=4,                      % Breite eines Tabs
	%escapeinside={(*@}{@*)},        % LaTeX-Befehle innerhalb des Codes
	morekeywords={print,def,class}, % Zusätzliche Schlüsselwörter für Python
	extendedchars=true,             % Unterstützt erweiterte Zeichen (z. B. Umlaute)
}
% Label für Listings ändern
\renewcommand{\lstlistingname}{Beispiel}


% ============= Dokumentbeginn =============
\begin{document}
	
	%Titelseite
	\thispagestyle{empty}
\begin{center}
\begin{tabular}{p{\textwidth}}
		
\begin{center}
	\textbf{\Large{\textsc{
			Entwicklung einer smarten Bewässerungslösung mit Web-Anbindung
	}}}
\end{center}

\vspace{1em}
\vspace{1em}
\vspace{1em}

\begin{center}
	\Large{Studienarbeit}
\end{center}

\vspace{1em}

\begin{center}
	im Rahmen des \\
	\large{\textbf{Bachelor of Science (B.Sc.)}} 
\end{center}

\vspace{1em}

\begin{center}
	des Studiengangs Informatik Cyber Security \\
	der Dualen Hochschule Baden-Württemberg Mannheim
\end{center}

\vspace{1em}
\vspace{1em}

\begin{center}
	vorgelegt von
\end{center}

\begin{center}
	\textbf{Maximilian Schüller, Fynn Thierling, Justus Siegert,\\Lukas Maier, Timon Kleinknecht}
\end{center}

\vspace{1em}
\vspace{1em}

\begin{center}
	\today
\end{center}
\end{tabular}
\end{center}
	
	\cleardoublepage
	\pagenumbering{roman}
	
	%Überschrift "Abkürzungsverzeichnis" setzen
\section*{Abkürzungsverzeichnis}
\addcontentsline{toc}{section}{Abkürzungsverzeichnis}
%\begin{acronym}[STRIDE]
	%\acro{EXP}{example}	-> Im text verwenden mit \ac{EXP}	
    
	
%\end{acronym}

	
	%Inhaltsverzeichnis
	\tableofcontents
	
	
	
	%Verzeichnis aller Abbildungen
	%\addcontentsline{toc}{section}{\listfigurename}
	%\listoffigures
	
	%Verzeichnis aller Tabellen
	%\listoftables
	
	\newpage
	
	% pagestyle für restliches Dokument aktivieren
	\pagestyle{fancy}
	\pagenumbering{arabic}
	
	\input{einleitung}
	\newpage
	
	% ------------- Hauptteil -------------

	\input{Teil 1}
	\newpage
	\input{Teil 2 ...}
	\newpage
	
	% ------------- Ende Hautpteil -------------
	
	\input{schluss}
	
	%\newpage
	%\thispagestyle{fancy} % fancyhdr aktivieren
	%\fancyhead[R]{LITERATUR}

	%\newpage
	%\thispagestyle{fancy} % fancyhdr aktivieren
	%\fancyhead[R]{ANHANG}	
	
	\newpage
	%Literaturverzeichnis
	%\bibliographystyle{unsrtdin}
	%\bibliography{literatur}
	\printbibliography
	
\end{document}