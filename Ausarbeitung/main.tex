% ============= Package Einstellungen & Sonstiges ============= 
\documentclass[a4paper,12pt]{report}
\usepackage[left= 2.5 cm,right = 2.5 cm, bottom = 2.5 cm]{geometry}
\usepackage[onehalfspacing]{setspace}

\usepackage[
pdftitle={Studienarbeit},
pdfsubject={Entwicklung einer smarten Bewässerungslösung mit Web-Anbindung},
pdfauthor={Maximilian Schüller, Fynn Thierling, Justus Siegert, Lukas Maier, Timon Kleinknecht},
pdfkeywords={},	
hidelinks %Links nicht einrahmen
]{hyperref}

\usepackage[utf8]{inputenc}
\usepackage[ngerman]{babel}
\usepackage[T1]{fontenc}

\usepackage{fancyhdr}
\usepackage{color}
\usepackage{csquotes}
%\usepackage{cite}
\usepackage[backend=biber, autocite=inline, style=ieee, natbib=true]{biblatex}
\addbibresource{literatur.bib}
\DefineBibliographyStrings{ngerman}{andothers = {{et\,al\adddot}},}
\usepackage{url}

\usepackage{graphicx} %für Einbindung von Grafiken
\graphicspath{{img/}} %Pfad für Grafiken

\usepackage{pdfpages}

\usepackage{todonotes}

\usepackage[printonlyused]{acronym}

\usepackage{minted} %für Darstellung von Code
\usepackage{float}
\usepackage[german]{varioref} %für schönere Referenzierung von Abbildungen

\fancyhead[L]{} % Linke Kopfzeile leer lassen

\usepackage{xspace}

% Define a command to ensure consistent space after acronym
\newcommand{\acspace}{\xspace\hspace{1em}}

\makeatletter
\renewcommand*{\aclabelfont}[1]{\textbf{\acsfont{#1}\acspace}}
\makeatother


\newcommand{\source}[1]{\vspace{1ex}\noindent{\small \textit{Quelle: #1}}}

\newcommand{\initializeBibliography}{
	\ihead{}
	\printbibliography[title=\Literaturverzeichnis] 
	\cleardoublepage
}

\usepackage{enumitem}
\usepackage{amssymb}

\usepackage{listings}
\usepackage{xcolor}  % Optional, für zusätzliche Farbanpassungen
\lstset{ 
	language=Python,                % Programmiersprache
	basicstyle=\ttfamily\scriptsize,     % Grundschriftart (monospace) und -größe
	keywordstyle=\bfseries\color{blue}, % Schlüsselwörter fett und blau
	commentstyle=\itshape\color{green!50!black}, % Kommentare kursiv und grün
	stringstyle=\color{red},        % Strings rot
	numbers=left,                   % Zeilennummern links anzeigen
	numberstyle=\tiny\color{black},  % Stil der Zeilennummern (klein und grau)
	stepnumber=1,                   % Jede Zeile nummerieren
	numbersep=8pt,                  % Abstand der Zeilennummern zum Code
	backgroundcolor=\color{gray!10},  % Hintergrundfarbe des Codes (weiß)
	frame=single,                   % Rahmen um den Code
	rulecolor=\color{black},        % Rahmenfarbe
	captionpos=b,                   % Position des Titels (unten)
	breaklines=true,                % Zeilen umbrechen, wenn sie zu lang sind
	breakatwhitespace=true,         % Zeilenumbrüche nur an Leerzeichen
	showstringspaces=false,         % Leerzeichen in Strings nicht anzeigen
	tabsize=4,                      % Breite eines Tabs
	%escapeinside={(*@}{@*)},        % LaTeX-Befehle innerhalb des Codes
	morekeywords={print,def,class}, % Zusätzliche Schlüsselwörter für Python
	extendedchars=true,             % Unterstützt erweiterte Zeichen (z. B. Umlaute)
}
% Label für Listings ändern
\renewcommand{\lstlistingname}{Beispiel}

\usepackage{titlesec}
\titleformat{\chapter}[hang]
  {\normalfont\huge\bfseries}
  {\thechapter\quad}
  {0pt}
  {} 


% ============= Dokumentbeginn =============
\begin{document}
	
	%Titelseite
	\thispagestyle{empty}
\begin{center}
\begin{tabular}{p{\textwidth}}
		
\begin{center}
	\textbf{\Large{\textsc{
			Entwicklung einer smarten Bewässerungslösung mit Web-Anbindung
	}}}
\end{center}

\vspace{1em}
\vspace{1em}
\vspace{1em}

\begin{center}
	\Large{Studienarbeit}
\end{center}

\vspace{1em}

\begin{center}
	im Rahmen des \\
	\large{\textbf{Bachelor of Science (B.Sc.)}} 
\end{center}

\vspace{1em}

\begin{center}
	des Studiengangs Informatik Cyber Security \\
	der Dualen Hochschule Baden-Württemberg Mannheim
\end{center}

\vspace{1em}
\vspace{1em}

\begin{center}
	vorgelegt von
\end{center}

\begin{center}
	\textbf{Maximilian Schüller, Fynn Thierling, Justus Siegert,\\Lukas Maier, Timon Kleinknecht}
\end{center}

\vspace{1em}
\vspace{1em}

\begin{center}
	\today
\end{center}
\end{tabular}
\end{center}
	
	\cleardoublepage
	\pagenumbering{roman}
	
	%------------- Erklärung der Eigenleistung-----------

	\pagebreak
\hspace{0pt}
\vfill
\begin{center}
    \large{Erklärung der Eigenleistung}
\end{center}
\vspace{1em}
\begin{center}
    \textit{Hiermit erklären wir, dass wir die vorliegende Studienarbeit selbstständig und ohne fremde Hilfe verfasst haben. Wir haben keine anderen als die angegebenen Quellen und Hilfsmittel benutzt. Darüber hinaus erklären wir, dass im Rahmen des Schreibprozesses KI-gestützte Werkzeuge (ChatGPT) zur Umformulierung von Textstellen verwendet wurden. Wir bestätigen hiermit, dass alle verwendeten Quellenangaben korrekt sind und die inhaltliche Verantwortung für die Arbeit uneingeschränkt bei uns liegt.}
\end{center}

\begin{tabular}{>{\centering\arraybackslash}p{0.5\textwidth} >{\centering\arraybackslash}p{0.5\textwidth}}
  \includegraphics[height=2\baselineskip, keepaspectratio]{img/MS_Unterschrift.png}
  &
  \includegraphics[height=2\baselineskip, keepaspectratio]{img/FT_Unterschrift.png}
  \\
  Maximilian Schüller & Fynn Thierling \\
  \includegraphics[height=2\baselineskip, keepaspectratio]{img/JS_Unterschrift.png}
  &
  \includegraphics[height=2\baselineskip, keepaspectratio]{img/LM_Unterschrift.jpg}
  \\
  Justus Siegert & Lukas Maier \\
  \includegraphics[height=2\baselineskip, keepaspectratio]{img/Unterschrift_TK.jpg}
  &
  \\
  Timon Kleinknecht & Mannheim, 15.04.2025
\end{tabular}


\vfill
\hspace{0pt}
\pagebreak

	\newpage

	%------------- Abstract -------------
	% Abstract in English
\section*{Abstract}
\addcontentsline{toc}{section}{Abstract}

\newpage

% Abstract in Deutsch
\section*{Abstrakt}
\addcontentsline{toc}{section}{Abstrakt}


	\newpage
	
	%------------- Inhaltsverzeichnis -------------
	\tableofcontents
	
	%------------- Abkürzungsverzeichnis -------------
	%Überschrift "Abkürzungsverzeichnis" setzen
\section*{Abkürzungsverzeichnis}
\addcontentsline{toc}{section}{Abkürzungsverzeichnis}
\begin{acronym}[STRIDE]
	%\acro{EXP}{example}	-> Im text verwenden mit \ac{EXP}	
	
\end{acronym}
	\newpage
	
	%------------- Abbildungsverzeichnis -------------
	\section*{Abbildungsverzeichnis}
	\addcontentsline{toc}{section}{Abbildungsverzeichnis}
	\renewcommand{\listfigurename}{} % Verhindert doppelten großen Titel
	\newpage

	
	%------------- Tabellenverzeichnis -------------
	\section*{\listtablename}
	\addcontentsline{toc}{section}{\listtablename}
	\renewcommand{\listtablename}{} % Verhindert doppelten großen Titel
	\newpage

	
	% pagestyle für restliches Dokument aktivieren
	\pagestyle{fancy}
	\pagenumbering{arabic}
	


	%------------- Einleitung -------------
	\chapter{Einleitung}
	\section{Motivation}
\label{sec:Motivation}

Pflanzen gehören in Deutschland und Europa fest zum Alltag in Wohnung und Garten. Laut einer repräsentativen Umfrage aus dem Jahr 2020 besitzen rund drei Viertel der Bundesbürger:innen (74\,\%) Zimmerpflanzen in ihrem Zuhause; auch auf Balkonen (35\,\%), Terrassen (30\,\%) und Fensterbänken (21\,\%) grünt es, während nur etwa 10\,\% ganz ohne Pflanzen leben\autocite{pflanzenbesitz_de}. Dieses „grüne Zuhause“ liegt im Trend und gewann insbesondere während der COVID-19-Pandemie an Bedeutung – viele Menschen entdeckten 2020 im Home-Office ihre Liebe zu Haus- und Gartenpflanzen neu\autocite{pflanzenbesitz_de}. Entsprechend stieg der Absatz: Der deutsche Markt für Blumen und Zierpflanzen erreichte nach Jahren der Stagnation 2020 ein Rekordvolumen von 9{,}4\,Mrd.\,€\autocite{stihl_gartenbarometer}. Ähnlich hohe Werte zeigen sich europaweit, wo Pflanzen als wichtiger Teil der Wohn- und Lebensqualität gelten. Neben dekorativen Aspekten werden Zimmer- und Gartenpflanzen aufgrund positiver Effekte wie besserer Luftqualität und Stressreduktion geschätzt\autocite{pflanzenbesitz_de}. Die hohe Verbreitung und Wertschätzung von Pflanzen in Privathaushalten bildet den Ausgangspunkt für die Betrachtung, wie ihre Pflege im Alltag unterstützt werden kann.

Allerdings stehen viele Pflanzenbesitzer:innen vor praktischen Herausforderungen bei der Pflege ihres „grünen Mitbewohners“. Im hektischen Alltag wird das Gießen leicht vergessen oder unregelmäßig vorgenommen; umgekehrt gießen unerfahrene Halter oft zu viel aus Sorge um die Pflanze. Studien bestätigen, dass Überwässerung der häufigste Grund für das Eingehen von Zimmerpflanzen ist\autocite{pflanzenpflege_fehler}. Generell erfordert jede Pflanzenart spezifische Kenntnisse zu Wasser- und Nährstoffbedarf, Lichtverhältnissen etc., über die im privaten Umfeld nicht immer ausreichend Wissen vorhanden ist. So gaben in einer Umfrage lediglich 37\,\% der befragten Frauen und 20\,\% der Männer an, einen „grünen Daumen“ zu haben\autocite{pflanzenbesitz_de} – die Mehrheit traut sich die optimale Pflanzenpflege also eher nicht zu. Hinzu kommt, dass während Urlaubs- oder Abwesenheitszeiten oft keine Betreuung für die heimischen Gewächse sichergestellt ist. Tatsächlich vermissten in einer Befragung 26\,\% der Pflanzenhalter:innen ihre Zimmerpflanzen im Urlaub sogar mehr als die Kolleg:innen\autocite{pflanzenbesitz_de}, was die emotionale Bindung und zugleich das Problem der Versorgung in dieser Zeit verdeutlicht. Diese Pflegeherausforderungen führen dazu, dass viele privat gehaltene Pflanzen Schäden nehmen oder vorzeitig absterben.

Die Folgen von falscher oder unregelmäßiger Pflege sind in Zahlen beträchtlich. Hochrechnungen zufolge überlebt ein erheblicher Teil der gekauften Zierpflanzen nicht lange: Etwa 40\,\% der Pflanzen gehen bereits in der Lieferkette zugrunde, und weitere rund 35\,\% sterben später in den Wohnungen der Kundschaft\autocite{pflanzensterben_statistik}. Mit anderen Worten wird fast die Hälfte aller gekauften Haus- und Gartenpflanzen letztlich aufgrund suboptimaler Bedingungen oder Pflegefehler nicht dauerhaft erhalten. Auch Verbraucherumfragen deuten auf dieses Problem hin. Beispielsweise gab über ein Drittel der Hobbygärtner in einer aktuellen Erhebung an, jedes Jahr ein bis zwei Zimmerpflanzen zu verlieren\autocite{pflanzensterben_statistik}. Solche Verluste sind nicht nur emotional enttäuschend für Pflanzenliebhaber, sondern bedeuten auch Ressourcenverschwendung – insbesondere von Wasser, Zeit und Geld. Schätzungen aus den USA zeigen etwa, dass jüngere „Plant Parents“ im Durchschnitt schon mehrere ihrer erworbenen Pflanzen unbeabsichtigt zum Eingehen gebracht haben. Diese Zahlen unterstreichen die Notwendigkeit, neue Wege zu finden, um häufige Pflegefehler zu vermeiden und die Lebensdauer der Pflanzen zu verlängern.

Technologische Lösungen im Sinne von \textit{Smart Gardening} setzen hier an und versprechen Abhilfe. Insbesondere automatische Bewässerungssysteme für den Heimgebrauch bieten die Möglichkeit, den Gießvorgang zu optimieren und zu automatisieren. Solche Systeme kombinieren oft Sensoren (etwa für Bodenfeuchte oder Licht) mit internetfähigen Steuerungen, um den Pflanzen exakt bei Bedarf und in der richtigen Menge Wasser zuzuführen. Erste Ansätze sind bereits auf dem Markt verfügbar – von App-gesteuerten Bewässerungscomputern bis hin zu smarten Pflanzentöpfen mit Selbstbewässerungs-Funktion. Die Akzeptanz solcher \textit{Smart-Home}-Technologien im Garten- und Pflanzenbereich steigt kontinuierlich. Laut dem STIHL-Gartenbarometer 2022 nutzen bereits rund 7\,\% der deutschen Gartenbesitzer smarte Garden-Lösungen, und etwa 30\,\% wünschen sich zukünftig solche automatisierten Helfer\autocite{stihl_gartenbarometer}. Dabei stehen Bewässerungsautomationen an erster Stelle der Wunschliste: 83\,\% der Befragten mit Smart-Gardening-Interesse nennen ein automatisches Bewässerungssystem als besonders gefragte Lösung. Diese Nachfrage spiegelt sich auch in anderen Ländern wider. Beispielsweise glauben in Österreich über 60\,\% der Gartenbesitzer, dass sich der Wasserverbrauch durch automatisierte Bewässerungsanlagen deutlich optimieren lässt. Moderne Systeme können Wetterdaten oder Bodensensoren einbeziehen, um nur dann zu wässern, wenn die Pflanze es wirklich benötigt – eine Technik, die den Pflanzenstress reduziert und zugleich Wasserverschwendung vorbeugt. Aktuelle Untersuchungen zeigen denn auch, dass intelligente Bewässerungssteuerungen den Wasserverbrauch im Garten gegenüber herkömmlichen Timern erheblich senken können (um etwa 20–40\,\% je nach System).

Mehrere übergeordnete Trends begünstigen die Verbreitung von smarten Pflanzenpflege-Systemen. Zum einen führt die Urbanisierung dazu, dass immer mehr Menschen auf kleinem Raum in Städten leben – in Deutschland etwa 78\,\% der Bevölkerung\autocite{urbanisierung_de} – und sich dennoch nach Natur im eigenen Umfeld sehnen. Insbesondere Stadtbewohner ohne Garten kultivieren vermehrt Zimmerpflanzen oder Balkongrün, sind aber beruflich oft stark eingebunden. Eine automatische Bewässerung kann hier den Pflegeaufwand mindern und sicherstellen, dass Pflanzen trotz hektischem Alltag oder Abwesenheiten ausreichend versorgt werden. Zum anderen rückt Nachhaltigkeit in den Fokus: Wassermanagement und effiziente Ressourcennutzung gewinnen an Bedeutung, da die Auswirkungen des Klimawandels – etwa häufigere Sommerdürreperioden – auch private Gärten und Balkone betreffen. In Umfragen äußern fast zwei Drittel der Befragten die Erwartung, dass digitale Technologien im Garten helfen können, den Klimawandel abzuschwächen, und nennen den schonenden Umgang mit Wasser als oberste Priorität. Smart-Bewässerungssysteme erfüllen genau diesen Zweck, indem sie bedarfsgerecht gießen und Überwässerung verhindern. Schließlich trägt auch die allgemeine Verbreitung von \textit{Internet of Things}-Anwendungen im Haushalt dazu bei, dass vernetzte Lösungen immer selbstverständlicher werden. Der europäische Smart-Home-Markt verzeichnet hohe Wachstumsraten und wird 2024 bereits auf über 22\,Mrd.\,US-\$ geschätzt\autocite{iot_trend}. Vernetzte, per App oder Sprache steuerbare Geräte – vom Thermostat bis zur Lichtsteuerung – gehören zunehmend zum Alltag. Diese Entwicklung macht auch vor dem Bereich der Pflanzenpflege nicht Halt: Die Nutzerakzeptanz für digitale Helfer im Haushalt schafft ein günstiges Umfeld für \textit{Smart Gardening}-Innovationen.

Insgesamt ist die Einführung eines smarten Bewässerungssystems im heimischen Umfeld vor dem Hintergrund dieser Fakten sowohl technisch zeitgemäß als auch gesellschaftlich sinnvoll. Die weit verbreitete Haltung von Zimmer- und Gartenpflanzen einerseits und die häufig auftretenden Pflegeprobleme andererseits schaffen ein deutliches Bedürfnis nach Unterstützung. Automatisierte Bewässerungslösungen können hier einen doppelten Nutzen stiften: Sie helfen Pflanzenbesitzer:innen, ihre grünen Schützlinge zuverlässig und fachgerecht zu versorgen, und tragen zugleich zu Nachhaltigkeit und Komfort bei. Indem ein smartes Bewässerungssystem Wasser bedarfsgerecht dosiert und den Pflegeprozess vereinfacht, steigert es die Überlebensrate und Vitalität der Pflanzen und entlastet den Menschen von Routineaufgaben. Die vorliegenden Studien, Statistiken und Trends untermauern somit die Notwendigkeit und den Nutzen eines solchen Systems, das im Folgenden technisch konzipiert und beschrieben wird.

	\newpage
	\section{Zielsetzung}
\label{sec:Zielsetzung}

Ziel dieser Studienarbeit ist die Konzeption, Entwicklung und prototypische Umsetzung eines automatisierten Systems zur Bewässerung von Zimmerpflanzen im privaten Wohnumfeld. Es soll eine lauffähige Gesamtlösung entstehen, die aus einem Mikrocontroller als zentrale Steuereinheit, einer Backend-Infrastruktur zur Datenverarbeitung und -persistierung sowie einem benutzerfreundlichen Frontend zur Visualisierung und Steuerung besteht. Die Realisierung erfolgt im Rahmen eines \textit{Proof of Concept}, der die technische Machbarkeit sowie die Integration der Systemkomponenten demonstriert.
\\
Das zu entwickelnde System erfasst über geeignete Sensorik (z.\,B. Bodenfeuchte, Temperatur, Luftfeuchtigkeit, Lichtintensität) kontinuierlich relevante Umgebungsdaten. Diese Messwerte dienen entweder als Entscheidungsgrundlage für den Nutzer, um über die Benutzeroberfläche manuell eine Bewässerung anzustoßen, oder sie werden vom Mikrocontroller automatisch verarbeitet. Im letzteren Fall wird anhand zuvor definierter Sollwerte eine autonome Steuerung der Bewässerungseinheit realisiert. Vorrangiges Ziel ist die technische Umsetzung des automatischen Betriebsmodus. Die Konzeption und Entwicklung des manuellen Modus sowie die Integration beider Steuerungsarten in das Gesamtsystem erfolgen nachrangig und abhängig von den im Projektverlauf verfügbaren Entwicklungskapazitäten.
\\
Die Bewässerungslösung ist primär für den Einsatz in Innenräumen konzipiert. Dies umfasst insbesondere Haushalte mit Zimmerpflanzen, bei denen typische Pflegeprobleme wie unregelmäßiges Gießen oder Unsicherheit bezüglich des Wasserbedarfs adressiert werden sollen.
\\
Der konkrete Funktionsumfang des Systems wird im Verlauf des Projekts iterativ entwickelt. Eine detaillierte Beschreibung der funktionalen und nicht-funktionalen Anforderungen sowie der Zielsystemeigenschaften erfolgt in Kapitel~\ref{chap:Anforderungen}. Dabei wird angestrebt, etablierte \textit{Best Practices} der Software- und Systementwicklung zu berücksichtigen und – wo sinnvoll und realisierbar – aktuelle Technologien gemäß dem Stand der Technik (\textit{State of the Art}) zu verwenden. Gleichzeitig wird die technische Umsetzung unter Berücksichtigung der konzeptionellen Natur als \textit{Proof of Concept} gewichtet, sodass pragmatische Abwägungen hinsichtlich Komplexität, Aufwand und Ressourcen erfolgen.
\\
Insgesamt dient die Arbeit dem Ziel, ein funktional überzeugendes Demonstrationssystem zu realisieren, das eine fundierte Grundlage für weiterführende Entwicklungen, Evaluationen oder mögliche Produktivsetzungen bietet.

	\newpage
	\section{Ziel der Arbeit}
\label{sec:Ziel der Arbeit}


	\newpage
	
	% ------------- Hauptteil -------------

	\chapter{Theoretische Grundlagen}
	
\section{Solace PubSub+ Event Broker}
Solace PubSub+ ist ein Enterprise-Messaging-Broker, der als zentrales Vermittlungssystem in einer ereignisgesteuerten Architektur dient. Er ermöglicht die asynchrone Kommunikation zwischen verteilten Anwendungen über ein publish/subscribe-basiertes Paradigma. Nachfolgend werden Aufbau und Hauptmerkmale dieses Brokers erläutert – insbesondere die Nutzung von Topics und Queues, die Unterstützung des MQTT-Protokolls, Mechanismen zur Zugriffskontrolle (ACLs) sowie die Konfiguration über die SEMP-API. Abschließend erfolgt ein Vergleich mit anderen Messaging-Lösungen.

\subsection{Aufbau eines Message Brokers und Solace-Architektur}

Ein Message Broker wie Solace PubSub+ fungiert als Vermittler zwischen Sendern und Empfängern von Nachrichten. Publisher schicken Nachrichten an den Broker, welcher diese anhand von Metadaten (meist Topics) an interessierte Subscriber weiterleitet. Dadurch sind Publisher und Subscriber entkoppelt – weder muss der Publisher die Empfänger kennen, noch der Subscriber die Quelle der Nachricht \cite{Eugster2003}. Diese Entkopplung erhöht die Skalierbarkeit und Flexibilität des Systems erheblich. Der Broker übernimmt Verantwortung für das Routing, Filtering und ggf. Persistieren der Nachrichten.

Solace PubSub+ implementiert dieses Prinzip in einer \textit{Event Broker}-Architektur, die auf hohe Durchsatzraten und viele parallele Verbindungen ausgelegt ist. Eine Besonderheit von Solace ist die Unterstützung mehrerer Protokolle innerhalb desselben Brokers. So können Clients über unterschiedliche offene Standards wie AMQP 1.0, MQTT 3.1.1/5.0, REST (HTTP) oder JMS 1.1 mit dem Broker kommunizieren, ohne Gateway oder Übersetzer \cite{SolaceProtocols}. Diese Multi-Protokoll-Fähigkeit erlaubt z.B., dass IoT-Geräte via MQTT Daten publizieren, während Backend-Systeme dieselben Daten über JMS oder WebSockets abonnieren – der Broker überbrückt dabei transparent die Protokolle.

Intern organisiert Solace den Nachrichtenraum in sogenannten \textit{Message VPNs} (Virtual Private Networks). Ein Message VPN ist ein logisch isolierter Kontext innerhalb eines Brokers, der Mandanten-Trennung ermöglicht. Alle folgenden Objekte wie Topics, Queues, Clients und ACL-Profile sind immer innerhalb eines bestimmten VPN definiert. Der Broker kann somit mehrere VPNs parallel betreiben, die wie separate virtuelle Broker fungieren.

Abbildung \ref{fig:broker-arch} zeigt schematisch die grundlegende Funktionsweise eines Pub/Sub-Brokers. Publisher senden Ereignisse an den Broker, der diese basierend auf Topic-Filtern an die verbundenen Subscriber verteilt. Solace unterscheidet hierbei zwei Qualitäten von Nachrichtenlieferung: zum einen flüchtige \textit{Direct Messages}, die nicht persistiert werden (QoS 0/1, “non-persistent”), zum anderen \textit{Guaranteed Messages}, die auf dem Broker gespeichert und bei Bedarf erneut zugestellt werden (QoS 1/2, “persistent”). Erstere bieten maximale Geschwindigkeit und werden nur an aktuell verbundene Empfänger zugestellt; Letztere werden in sogenannten \textit{Queues} vorgehalten, bis ein Empfänger sie abruft, wodurch keine Nachrichten verloren gehen \cite{SolaceDirectGuaranteed}. Durch diese Zweiteilung kann ein Event Broker sowohl hochvolumige Echtzeitdaten (z.B. Marktdaten-Feeds) effizient verteilen, als auch kritische Nachrichten zuverlässig zustellen, selbst wenn Empfänger vorübergehend offline sind.

\begin{figure}[h]
\centering
\begin{tikzpicture}[node distance=1.8cm, every node/.style={font=\small}]
  \node[draw, rounded corners, fill=blue!10, minimum width=2.5cm, minimum height=1cm] (broker) {Solace Broker};
  \node[left=4cm of broker, draw, fill=green!20, minimum width=2.2cm, minimum height=0.8cm] (pub1) {Publisher 1};
  \node[below=1.2cm of pub1, draw, fill=green!20, minimum width=2.2cm, minimum height=0.8cm] (pub2) {Publisher 2};
  \node[right=4.2cm of broker, draw, fill=orange!20, minimum width=2.5cm, minimum height=0.8cm] (sub1) {Subscriber 1};
  \node[below=1.2cm of sub1, draw, fill=orange!20, minimum width=2.5cm, minimum height=0.8cm] (sub2) {Subscriber 2 (Queue)};
  \node[above=1.0cm of broker, font=\small\bfseries] at (broker.north) {Topic: sensora/v1/send/controller123};

  \draw[->, thick] (pub1) -- node[above] {Publish} (broker);
  \draw[->, thick] (pub2) -- (broker);
  \draw[->, thick] (broker) -- node[above] {Subscribe (live)} (sub1);
  \draw[->, thick] (broker) -- node[below] {Persisted (Queue)} (sub2);
\end{tikzpicture}
\caption{Schematische Darstellung eines Publish/Subscribe Event Brokers mit Solace.}
\label{fig:broker-arch}
\end{figure}

\subsection{Topics und Queues in Solace PubSub+}

Solace PubSub+ verwendet ein hierarchisches Topic-System zur Adressierung von Nachrichten. Topics sind benannte Kanäle, die typischerweise durch mittels Schrägstrich / getrennte Hierarchieebenen strukturiert sind (z.B. Sensor/Temperatur/Aussen). Publisher veröffentlichen Nachrichten auf einem bestimmten Topic, und Subscriber können sich auf Topics (bzw. Muster davon) abonnieren. Eine Stärke von Solace ist, dass Topics \textit{dynamisch} und ohne Vorab-Konfiguration genutzt werden können – der Broker akzeptiert beliebige Topic-Strings und leitet Nachrichten entsprechend weiter. Für die Filterung unterstützen Topics Platzhalter: Das Zeichen * steht für genau ein Hierarchie-Level (ein Wort zwischen den Trennzeichen), während > als Wildcard für beliebig viele nachfolgende Hierarchieebenen fungiert. Beispielsweise würde ein Subscriber auf Topic Sensor/Temperatur/* alle Temperaturwerte aller Sensoren erhalten, während Sensor/> sämtliche Nachrichten im Topic-Pfad beginnend mit Sensor matcht. Diese flexiblen Wildcards erlauben eine inhaltliche Filterung der Ereignisse bereits auf Broker-Seite.

Neben der direkten Verteilung über Topics bietet Solace die Möglichkeit, Topics mit \textit{Queues} zu verknüpfen. Eine \textit{Queue} ist ein benannter, persistentierender Endpunkt auf dem Broker, der Nachrichten speichern kann. Queues dienen primär der Realisierung von Lastverteilung und Persistenz (Point-to-Point-Messaging): Subscriber können eine exklusive Verbindung zu einer Queue aufbauen und erhalten die dort gespeicherten Nachrichten in der Reihenfolge ihres Eingangs. Solace verknüpft nun das Topic- und Queue-Konzept dadurch, dass man einer Queue eine oder mehrere Topic-Abonnements geben kann. Die Queue fungiert damit als langlebiger Subscriber auf die entsprechenden Topics. Publiziert ein Publisher eine Nachricht auf einem solchen Topic mit persistenter Delivery-Mode (Persistent Message), so wird die Nachricht in der zugeordneten Queue abgelegt und steht dort zur Abholung bereit. Dieses Konzept erlaubt es, die Pub/Sub-Semantik mit der Zuverlässigkeit von Queues zu kombinieren. So kann man etwa erreichen, dass eine bestimmte Ereignisart (Topic) an mehrere unabhängige Dienste zugestellt wird, indem man pro Dienst eine eigene Queue mit demselben Topic-Subscription definiert – die Nachricht wird vom Broker dupliziert und landet in allen relevanten Queues (Fan-Out auf persistente Empfänger). Im Solace-System sind Queues standardmäßig langlebig und können Nachrichten auch über Broker-Neustarts hinweg speichern; optional können sie als \textit{exklusiv} (nur ein Verbraucher gleichzeitig) oder \textit{non-exklusiv} (mehrere konkurrierende Verbraucher) konfiguriert werden. \cite{SolaceTopicWildcard}.

Im Vergleich zu anderen Messaging-Brokern vereinheitlicht Solace damit das sonst oft getrennte Modell von Topics (für Pub/Sub) und Queues (für Punkt-zu-Punkt) zu einem flexiblen Konstrukt. Während z.B. in RabbitMQ Nachrichten immer an Exchanges gesendet und dann per Binding-Key auf Queues verteilt werden müssen, kann in Solace direkt auf einem Topic publiziert werden, das entweder von Clients abonniert oder von Queues “belauscht” wird \cite{SolaceTopicWildcard}. Auch JMS unterscheidet zwischen Topic und Queue als verschiedenen Destination-Typen; Solace als JMS-Provider mappt jedoch JMS-Topics und -Queues intern auf das gleiche einheitliche Event-Model, was Verwaltung und Integration erleichtert.

\subsection{MQTT und topic-basiertes Routing}

Das \textit{Message Queuing Telemetry Transport} (MQTT) Protokoll ist ein offener Standard für Leichtgewichts-Messaging, der vor allem im IoT-Umfeld verbreitet ist. Solace PubSub+ unterstützt MQTT in den Versionen 3.1.1 und 5.0 gemäß dem OASIS-Standard \cite{SolaceMQTT}. MQTT-Clients können sich mit dem Broker verbinden und Nachrichten auf MQTT-Topics publizieren bzw. abonnieren. Ein wesentliches Konzept von MQTT ist das topic-basierte Routing – analog zum oben beschriebenen Topic-System. MQTT-Themenpfade sind ebenfalls hierarchisch mit / aufgebaut; als Wildcards dienen + (ein Level) und \# (mehrere Level). Solace setzt diese MQTT-Semantik um und übersetzt sie intern auf das eigene Topic-Format. Beispielsweise entspricht ein MQTT-Subscription auf sensors/+/humidity einem Solace-Topicfilter sensors/*/humidity. Dadurch können MQTT-Publisher und -Subscriber nahtlos mit anderen Solace-Clients interagieren. So ist es möglich, dass ein Gerät via MQTT auf sensors/room1/humidity sendet, während ein Java-Anwendung über JMS das Topic sensors/\> abonniert und somit die Nachricht erhält – der Broker übernimmt die Protokollkonvertierung und das Routing \cite{SolaceProtocols}.

MQTT sieht per Spezifikation keine expliziten Queues vor; wenn ein MQTT-Subscriber offline geht (und QoS 1 oder 2 verwendet), puffert der Broker die Nachrichten in einer Session-Queue, die beim Wiederverbinden ausgeliefert werden. Solace erweitert diese Möglichkeiten insofern, als dass man persistente Weiterleitung an benannte Queues nutzen kann, indem man MQTT-Themen mit Solace-Queues verknüpft (wie zuvor beschrieben). Rein mit MQTT-Mitteln kann ein Device also nur mittels Durable Session (Clean Session = false) Nachrichten nachliefern lassen, während Solace auch langfristige Persistenz über MQTT hinaus bietet. Allerdings gelten Protokolleinschränkungen: z.B. unterstützt MQTT v3.1.1 keine Bearbeitung von auf einer Queue gespeicherten Nachrichten, da das Konzept im Protokoll fehlt. Daher wird in Szenarien mit Bedarf an fortgeschrittenem Queueing ggf. ein anderes Protokoll (AMQP 1.0 oder Solaces eigenes API/SMF) verwendet \cite{SolaceDaprMQTT}. Nichtsdestotrotz unterstreicht die MQTT-Fähigkeit von Solace die Flexibilität des Brokers, heterogene Clients in ein gemeinsames Topic-basiertes Kommunikationsgeflecht einzubinden.

\subsection{Access Control Lists (ACLs)}

In einem Multi-User-Messaging-System ist feingranulares Zugriffsmanagement essenziell, um Sicherheit und Mandantentrennung zu gewährleisten. Solace PubSub+ bietet hierzu \textit{Access Control List (ACL) Profiles} an, mit denen reguliert wird, welche Aktionen ein Client ausführen darf. Ein ACL-Profil bestimmt erstens, ob sich ein Client überhaupt mit dem Broker/Message-VPN verbinden darf, und zweitens, welche Topics er publizieren oder abonnieren darf \cite{SolaceACL}. Konkret bestehen ACL-Profile aus Regeln, die Verbindungsversuche sowie Topic-Zugriffe kontrollieren. Für jede dieser Kategorien (Connect, Publish-Topic, Subscribe-Topic, sowie Subscribe auf Shared-Queues) lässt sich ein Standardverhalten definieren – entweder \textit{allow} (erlauben) oder \textit{deny} (standardmäßig verweigern) – und es können Ausnahmeregeln (\textit{exceptions}) für spezifische Topics oder Topic-Muster hinzugefügt werden.

Jedem Client, der sich authentifiziert, wird ein ACL-Profil zugewiesen (entweder direkt am Client-Username konfiguriert oder – falls LDAP-Authentifizierung genutzt wird – über Gruppenzuordnung). Nach erfolgreicher Authentisierung prüft der Broker sämtliche Aktionen des Clients gegen die ACL des zugewiesenen Profils. Beispielsweise kann ein ACL-Profil erlauben, dass der Client nur auf Topics unterhalb von ClientA/> publizieren darf und alle anderen Topics geblockt werden. Versucht der Client eine Nachricht auf ein nicht erlaubtes Topic zu senden oder eine unberechtigte Subscription anzulegen, verweigert der Broker die Operation. Auf diese Weise können unterschiedliche Clients auf unterschiedliche Topic-Räume beschränkt werden, was im IoT-Umfeld z.B. dazu genutzt wird, dass jedes Gerät nur auf seinem eigenen Topic-Präfix senden/empfangen darf \cite{AvnetPSK}. ACL-Profile sind damit ein zentrales Sicherheitsmerkmal, um unerlaubten Datenzugriff zu verhindern und eine saubere Mandantentrennung innerhalb eines Brokers (bspw. zwischen verschiedenen Anwendungen im selben Message VPN) sicherzustellen.

\subsection{Konfiguration über die SEMP-API}

Die Administration eines Solace-Brokers kann neben grafischen Tools und dem CLI auch vollständig automatisiert über eine REST-basierte Managementschnittstelle erfolgen – die \textit{Solace Element Management Protocol} API in Version 2 (\textbf{SEMP v2}). SEMP v2 stellt eine REST/HTTP-JSON API bereit, mit der sich der Broker programmatisch konfigurieren und überwachen lässt \cite{SolaceSEMP}. Über definierte Endpunkte können Management-Clients z.B. Queues oder Topic-Endpunkte anlegen, ACL-Profile definieren, den Status von Clients und VPNs abfragen oder Verwaltungsaktionen (wie das Leeren einer Queue) durchführen. Die SEMP-API ist in drei Bereiche gegliedert: \begin{itemize}\setlength\itemsep{0pt} \item \textbf{Config}: Änderung und Abfrage der Konfigurationsobjekte \\(z.B. \verb|POST /SEMP/v2/config/msgVpns/{vpn}/queues| zum Anlegen einer Queue). \item \textbf{Monitor}: Lesezugriff auf betrieblichen Status und Metriken (z.B. Statistiken einer Queue, Anzahl der Clients etc.). \item \textbf{Action}: Auslösen von Aktionen, die keinen dauerhaften Konfigurationszustand ändern (z.B. \textit{Purge} einer Queue, Failover umschalten). \end{itemize} SEMP v2 ist via OpenAPI-Spezifikation beschrieben und ermöglicht so die automatisierte Generierung von Client-Bibliotheken \cite{SolaceSEMP}. In der Praxis kann somit z.B. ein Auth-Service mittels SEMP beim Deployment sicherstellen, dass benötigte Topics, Queues und ACLs im Broker eingerichtet sind, ohne manuell die Admin-Oberfläche bedienen zu müssen. Die Verwendung einer Management-API erleichtert die Integration des Brokers in Infrastructure-as-Code und DevOps-Prozesse. Insbesondere in Cloud-Umgebungen mit dynamischer Skalierung kann der Broker so automatisiert auf neue Services oder geänderte Sicherheitsrichtlinien konfiguriert werden.

\subsection{Vergleich mit anderen Messaging-Systemen}
Solace PubSub+ steht in Konkurrenz zu einer Reihe von Messaging-Systemen, die teils unterschiedliche Architekturen aufweisen. Im Folgenden werden einige charakteristische Unterschiede kurz herausgestellt:
\textbf{RabbitMQ}: RabbitMQ ist ein verbreiteter Open-Source Message Broker, der das AMQP 0.9.1 Protokoll verwendet. Ähnlich wie Solace unterstützt RabbitMQ Publish/Subscribe und Queueing, jedoch meist nur über AMQP (Multi-Protokoll-Fähigkeit ist bei RabbitMQ nur über Plugins begrenzt möglich). RabbitMQ nutzt ein Exchange-Queue-Modell: Publisher senden an Exchanges, welche die Nachrichten anhand von Routing-Keys an Queues binden. Solace hingegen erlaubt das direkte Publizieren auf Topics, die intern sowohl von Abonnements als auch von Queues “gelesen” werden können \cite{SolaceTopicWildcard}. In Solace ist somit das Routing-Konzept etwas einfacher, da kein separater Exchange-Schritt sichtbar ist. Bezüglich Performance skaliert Solace durch seine optimierte Architektur und optional spezialisierte Hardware-Appliances auf sehr hohe Durchsätze, während RabbitMQ für viele Anwendungsfälle ausreichend, aber in extremen Szenarien früher an Grenzen stoßen kann. RabbitMQ bietet dafür eine große Flexibilität durch Plugins (etwa für unterschiedliche Auth-Backends, Shovel/Federation für verteilte Setups usw.), wohingegen Solace viele Funktionen nativ integriert (etwa eingebaute Hochverfügbarkeit, Replay-Funktionalität, etc.).
\textbf{Apache Kafka}: Kafka verfolgt einen anderen Ansatz als klassische Message Broker. Es ist eigentlich eine verteilte Commit-Log-Plattform, die primär für Streaming von Datenströmen und Event Sourcing konzipiert ist. Im Gegensatz zu Solace, das Nachrichten direkt an Subscriber weiterleitet oder zwischenpuffert, schreibt Kafka alle Nachrichten fortlaufend in Partitionen auf Disk. Consumer lesen diese logischen Partitionen in eigenem Tempo und behalten Zustände wie den Lese-Offset. Dadurch eignet sich Kafka hervorragend für das Replaying vergangener Events und das Verteilen großer Datenmengen auf viele Konsumenten, hat aber keinen Broker, der aktiv zu Clients zustellt – die Clients holen vielmehr die Daten. Solace dagegen ist auf Live-Zustellung optimiert: Nachrichten werden typischerweise verworfen, wenn kein Abnehmer da ist (außer es existiert eine persistente Queue). Man könnte vereinfacht sagen, Kafka speichert Streams dauerhaft (Standard ist Tage bis unbegrenzt) und garantiert dadurch Zustellung an spätere Konsumenten, während Solace klassischere Messaging-Semantiken bietet, wo persistente Nachrichten gezielt zugestellt und nach Bestätigung gelöscht werden (ähnlich JMS). Für use-cases im Auth-Service-Umfeld (kurzlebige Ereignisse, Befehle, Responses) ist ein Broker wie Solace passender; Kafka spielt seine Stärken bei Analyse- und Loggingdaten aus \cite{Kreps2011}.
\textbf{JMS Broker (ActiveMQ/IBM MQ) und Andere}: Solace kann auch als JMS-Provider dienen und vergleichbare Funktionalität wie ActiveMQ, Artemis oder IBM MQ bereitstellen. Gegenüber vielen JMS-Brokern punktet Solace mit integrierter Mehrprotokollfähigkeit und höherer Performance. IBM MQ etwa ist stark auf Transaktionssicherheit und Mainframe-Integration ausgelegt, während Solace auf niedrige Latenz und flexible Verteilung optimiert ist. Moderne Cloud-Messaging-Services (AWS SNS/SQS, Azure Service Bus) trennen oft Pub/Sub und Queuing in separate Services – Solace bietet beides in einem. Insgesamt ist Solace als \glqq Enterprise Event Broker\grqq{} positioniert, der in verteilten Architekturen (z.B. Microservices, IoT, Hybrid-Cloud) eine zentrale Kommunikationsdrehscheibe bildet und durch Features wie \textit{Dynamic Message Routing} (für geclusterte Broker-Netze, Event Mesh) und eingebaute Hochverfügbarkeit auszeichnet.
Zusammenfassend lässt sich festhalten, dass Solace PubSub+ aufgrund seines architektonischen Designs (vereinheitlichtes Topic/Queue-Model, Multi-Protokoll-Unterstützung, hohe Performance) eine besondere Rolle unter den Messaging-Systemen einnimmt. Die theoretischen Grundlagen dieses Brokers – Entkopplung durch Pub/Sub, Hierarchie-Topics, etc. – bilden die Basis für den praktischen Einsatz im Rahmen des Auth-Service, wo Solace die Nachrichtenkommunikation absichert und steuert.


\section{Sicherheitsrelevante Technologien im Auth-Service}

Im Auth-Service kommen mehrere kryptographische und sicherheitstechnische Verfahren zum Einsatz, um die Authentifizierung und Kommunikation abzusichern. In diesem Abschnitt werden die folgenden Konzepte erläutert:
\textbf{HMAC} (Hash-based Message Authentication Code) – ein Verfahren zur Gewährleistung von Nachrichtenintegrität und -authentizität.
\textbf{Fernet-Verschlüsselung} – ein symmetrisches Authenticated-Encryption-Schema zur sicheren Ausgabe von Tokens.
\textbf{Tokenisierung mit PSK und Challenge-Response} – das Zusammenspiel von vorab geteilten Schlüsseln (Pre-Shared Keys), Challenge-Response-Authentifizierung und der Verwendung von Tokens im Authentifizierungsprozess.
Ziel ist es, die kryptographischen Grundlagen, Funktionsweisen und Sicherheitseigenschaften dieser Techniken darzustellen, um das Verständnis für ihre Umsetzung im Auth-Service zu schaffen.

\subsection{HMAC: Hash-basierter Message Authentication Code}

Ein \textit{Message Authentication Code} (MAC) ist eine Prüfsumme, die unter Verwendung eines geheimen Schlüssels berechnet wird und es Empfängern ermöglicht, die Integrität und Authentizität einer Nachricht zu verifizieren. Der Sender berechnet einen MAC-Wert über die Nachricht und sendet ihn mit; der Empfänger kann mit dem geteilten Geheimnis denselben MAC berechnen und vergleichen. Stimmen die Werte überein, ist sichergestellt, dass die Nachricht unverändert ist und vom legitimen Sender stammt (solange der Schlüssel geheim bleibt).

HMAC (\textit{Keyed-Hash Message Authentication Code}) ist ein spezielles, sehr weit verbreitetes MAC-Verfahren, das auf kryptographischen Hashfunktionen basiert \cite{RFC2104}. Formal lässt sich HMAC definieren als: 

$\mathbf{HMAC_{k}}(m)=H((K⊕opad)∥H((K⊕ipad)∥m))$, wobei $H$ eine kryptographische Hashfunktion (etwa SHA-256) und $K$ ein geheimer Schlüssel ist (typischerweise auf Blockgröße von $H$ aufgefüllt). Die Konstanten $opad$ und $ipad$ sind bitmasken (\textit{outer/inner padding}), die den Schlüssel vor dem Hashen modifizieren. Durch diese zweistufige Konstruktion (Hash innerhalb eines Hashs) werden bestimmte Angriffe auf einfachere MAC-Konstruktionen (wie \glqq Schlüssel+Nachricht hashen\grqq) verhindert. Intuitiv berechnet HMAC zunächst einen Hash der Kombination aus Schlüssel und Nachricht und hasht dieses Ergebnis nochmals zusammen mit dem Schlüssel.

HMAC wurde 1996 von Bellare, Canetti und Krawczyk entwickelt und analysiert. In RFC 2104 ist HMAC als generisches Konstrukt spezifiziert, das mit jeder iterativen Hashfunktion (MD5, SHA-1, SHA-2 etc.) genutzt werden kann \cite{RFC2104}. So verwendet HMAC-SHA256 beispielsweise die Secure Hash Algorithm 256-bit Funktion als $H$. Die Sicherheit von HMAC hängt von den Eigenschaften der zugrundeliegenden Hashfunktion ab, insbesondere deren Kollisionsresistenz und Unberechenbarkeit \cite{RFC2104}. Unter typischen Annahmen lässt sich zeigen, dass HMAC annähernd ein \textit{pseudorandom function (PRF)} ist, solange $H$ keine gravierenden Schwächen aufweist – informell bedeutet dies, dass ohne Kenntnis des Schlüssels kein effizienter Weg bekannt ist, um den korrekten MAC für eine neue Nachricht zu erzeugen oder eine gültige Nachricht+MAC zu modifizieren. HMAC ist zudem resistent gegen sogenannte \textit{Längenextensierungs-Angriffe}, die bei einfachen Hash-MACs auftreten könnten, weil durch die Verwendung von $opad$/$ipad$ die innere Struktur des Hashs gesichert wird.

Aufgrund seiner starken Sicherheitseigenschaften und Einfachheit ist HMAC in vielen Protokollen und Standards fest verankert. Das US-amerikanische NIST standardisierte HMAC in FIPS 198-1 \cite{FIPS198}. In TLS und IPsec dient HMAC (mit SHA-256 oder SHA-384) als Nachrichten-Authentifizierungsmechanismus, um die Integrität von übertragenen Daten sicherzustellen. Auch viele Anwendungsschichten nutzen HMAC: Beispielsweise beruht der HOTP-Algorithmus (HMAC-based One-Time Password) für Einmalpasswörter auf einem HMAC über einen Zähler \cite{RFC4226}, und JSON Web Tokens (JWT) können mit einem HMAC-SHA256 (\texttt{HS256}) Signaturteil versehen sein, um Tokenintegrität zu gewährleisten.

Im Kontext des Auth-Service wird HMAC insbesondere im Rahmen eines \textit{Challenge-Response}-Verfahrens eingesetzt. Hierbei teilen sich der Client (z.B. ein IoT-Gerät) und der Server einen geheimen Schlüssel $K$ (siehe \glqq Pre-Shared Key\grqq{} unten). Zur Authentifizierung sendet der Server dem Client eine Zufallsherausforderung (Nonce) – typischerweise eine ausreichend lange zufällige Bitfolge $N$. Der Client berechnet daraufhin den HMAC-Wert über diese Challenge:
R=HMACk(N). Dieser Response-Wert $R$ wird an den Server zurückgesendet. Da der Server ebenfalls $K$ kennt, kann er seinerseits $\mathrm{HMAC}_K(N)$ berechnen und prüfen, ob das Ergebnis mit dem empfangenen $R$ übereinstimmt. Ist dies der Fall, beweist der Client damit, dass er den geheimen Schlüssel besitzt (denn nur mit $K$ lässt sich der korrekte MAC erzeugen). Gleichzeitig wird die Integrität der Challenge gewährleistet – eine Manipulation von $N$ unterwegs würde zu einem falschen MAC führen. Wichtig ist, dass für jede neue Authentifizierung eine frische, unvorhersehbare Challenge $N$ gewählt wird, um Wiederholungsangriffe (Replay) zu verhindern. Dieses einfache Challenge-Response-Schema bietet eine zuverlässige Einweg-Authentifizierung des Clients gegenüber dem Server, ohne dass der Schlüssel selbst jemals übertragen wird. Die Verwendung von HMAC dafür ist ideal, da HMAC effizient berechenbar ist (auch auf eingeschränkter Hardware) und kryptographisch stark genug, um nicht geraten oder invertiert werden zu können.

Zusammengefasst liefert HMAC also einen grundlegenden Baustein für die Nachrichtenauthentifizierung: mathematisch basiert es auf Hashfunktionen und verknüpft einen geheimen Schlüssel mit der Nachricht zu einem Prüftoken. Seine Sicherheit wurde analytisch untermauert \cite{Krawczyk1997} und in Praxisstandards übernommen, was es zu einem vertrauenswürdigen Mechanismus für Integritätsschutz macht. Im Auth-Service bildet HMAC den Kern des Authentifizierungsnachweises im Challenge-Response-Protokoll.


\subsection{Fernet-Verschlüsselung: Authenticated Encryption für Tokens}

Neben der Überprüfung von Nachrichtenintegrität (via MAC) ist oft auch Vertraulichkeit und Schutz vor Manipulation bei gespeicherten oder übertragenen Tokens erforderlich. \textit{Fernet} ist ein hochrangiges kryptographisches Schema, das symmetrische Verschlüsselung mit Integritätsschutz kombiniert und dadurch sog. \textit{Authenticated Encryption} bereitstellt. Es wurde ursprünglich als Teil der Python-Kryptographiebibliothek entwickelt, liegt aber als offene Spezifikation vor und ist in mehreren Sprachen implementiert \cite{cryptographyFernet}. Fernet garantiert, dass ein verschlüsselter Datenträger (Token) weder gelesen noch unbemerkt verändert werden kann, ohne den geheimen Schlüssel zu besitzen \cite{cryptographyFernet}.

Technisch basiert Fernet auf anerkannten Standard-Primitiven: \begin{itemize}\setlength\itemsep{0pt} \item \textbf{Verschlüsselung}: AES im CBC-Modus mit 128-Bit-Schlüssel und PKCS#7 Padding wird verwendet, um die eigentlichen Nutzdaten zu chiffrieren. \item \textbf{Integrität/Authentizität}: Ein HMAC über die verschlüsselten Daten (inkl. Header) mit SHA-256 stellt sicher, dass Änderungen erkannt werden. \item \textbf{Zufall}: Für jede Verschlüsselung wird ein frischer 128-Bit Initialisierungsvektor (IV) per Kryptozufall erzeugt. \end{itemize} Diese Details sind in der Fernet-Spezifikation festgelegt \cite{FernetSpec}. Ein Fernet-Schlüssel ist 32 Byte (256 Bit) lang und wird intern in zwei Hälften aufgeteilt: 16 Byte dienen als AES-Schlüssel, 16 Byte als HMAC-Schlüssel \cite{FernetSpec}. Bei der Token-Erstellung wird folgendermaßen verfahren \cite{FernetSpec}:
\begin{enumerate}
    \item Es wird ein 64-Bit-Zeitstempel (Sekunden seit Unix-Epoche) als \textit{Timestamp} festgehalten, der den Erstellungszeitpunkt markiert.
    \item Ein zufälliger 16-Byte IV wird generiert.
    \item Die Klartextnachricht wird mit AES-128-CBC unter Verwendung des IV und des Verschlüsselungsschlüssels verschlüsselt (nachdem sie mit PKCS7 auf Blocklänge aufgefüllt wurde).
    \item Über den sogenannten Token-Header (bestehend aus einer 8-Bit Versionskennung, dem Timestamp und dem IV) \textit{und} den Ciphertext wird ein HMAC-SHA256 mit dem HMAC-Schlüssel berechnet
    \item  Der finale Fernet-Token besteht aus der Konkatenation von Version, Timestamp, IV, Ciphertext und HMAC. Dieser Byte-String wird zuletzt URL-sicher Base64-kodiert, um ihn als ASCII-Token darstellbar zu machen. Das Format eines Fernet-Tokens lässt sich schematisch so darstellen: 


\[
\underbrace{\texttt{0x80}}_{\text{Version}} 
\quad
\underbrace{T_{\text{stamp}}}_{\text{64-Bit-Zeit}} 
\quad
\underbrace{IV}_{\text{128-Bit}} 
\quad
\underbrace{C}_{\text{Ciphertext}} 
\quad
\underbrace{\mathrm{HMAC}}_{\text{256-Bit}},
\]
,wobei $C$ variable Länge (Vielfaches von 128 Bit) hat, abhängig von der Klartextlänge \cite{FernetSpec}. Die Versionsnummer ist aktuell immer 0x80 (Version 1 der Spezifikation). Der HMAC deckt den gesamten Token ohne das HMAC-Feld ab, d.h. $HMAC = \mathrm{HMAC}{K{\text{HMAC}}}(\text{Version} \parallel T_{\text{Stamp}} \parallel IV \parallel C)$ \cite{FernetSpec}. Dadurch wird sichergestellt, dass sowohl Header als auch Ciphertext geschützt sind – eine Veränderung irgendeines Bits im Token führt zu einem HMAC-Mismatch beim Entschlüsseln.
\end{enumerate}

Fernet ist somit ein klar strukturierter Fall von \textit{Encrypt-then-MAC}: Zuerst werden Daten symmetrisch verschlüsselt, dann wird über das Chiffrat ein MAC gebildet. Diese Konstruktion gilt als sicher (im Sinne von IND-CCA2, also Wahrung von Vertraulichkeit auch gegen adaptive Angreifer), da ein Empfänger immer erst den MAC prüft, bevor er entschlüsselt. Ohne Kenntnis des Schlüssels kann ein Angreifer weder den Ciphertext sinnvoll entschlüsseln noch einen gültigen neuen Token erzeugen, da ihm sowohl der AES- als auch der MAC-Schlüssel fehlen.

Die Verwendung von Standardbausteinen (AES, HMAC) stellt sicher, dass Fernet auf bekannten Sicherheitsannahmen beruht. AES-128 ist ein symmetrischer Blockcipher, der als praktisch sicher gilt; CBC-Modus ist sicher, solange jedes IV nur einmalig verwendet wird (was durch die zufällige Wahl gegeben ist). HMAC-SHA256 ist, wie oben erläutert, ein starker MAC. Die Kombination ergibt einen symmetrischen Authenticated-Encryption-Algorithmus, der konzeptionell ähnlich ist zu anderen AEADs (Authenticated Encryption with Associated Data) wie AES-GCM – allerdings mit dem Unterschied, dass Fernet die Komponenten explizit trennt und kein zusätzliches Associated Data Feature hat. Ein Vorteil von Fernet ist die Einfachheit der Schnittstelle: Entwickler erhalten eine einzige Routine zum Verschlüsseln und Entschlüsseln von Bytestrings, wodurch Implementierungsfehler (wie Vergessen der MAC-Prüfung oder unsichere IV-Wahl) vermieden werden. Die Fernet-Implementierung in der Python-\texttt{cryptography}-Bibliothek z.B. stellt sicher, dass bei Aufruf von decrypt(token) automatisch der HMAC geprüft und bei Scheitern eine Exception geworfen wird, sodass kein Entwickler versehentlich mit unverifizierten Daten weiterarbeitet \cite{cryptographyFernet}.

Für den Auth-Service ist besonders relevant, dass Fernet-Tokens einen eingebetteten Zeitstempel besitzen. Dies erlaubt es, die Gültigkeit von Tokens zeitlich zu begrenzen: Beim Entschlüsseln kann der Server den im Token codierten Erstellungszeitpunkt mit der aktuellen Zeit vergleichen und ein Token ggf. als abgelaufen verwerfen, wenn es älter als eine definierte TTL (time-to-live) ist. Somit kann ein vom Auth-Service ausgestelltes Token z.B. nur für eine Stunde gültig sein, danach muss sich der Client neu authentifizieren. Fernet selbst erzwingt keine Ablaufprüfung, bietet aber die Information (Timestamp) dafür an.

Ein weiteres Merkmal ist die feste Größe des Schlüssels (256 Bit), was ausreichend Sicherheit bietet. Sollte es nötig sein, Schlüssel zu rotieren, unterstützt Fernet dies durch das Konzept von \textit{Multi-Fernet}, bei dem mehrere Schlüssel (z.B. alter und neuer) parallel akzeptiert werden können \cite{cryptographyFernet}. Im Normalfall wird jedoch ein einzelner, sicher verwahrter Key vom Auth-Service verwendet, um alle Token zu signieren/verschlüsseln.

Zusammengefasst stellt Fernet also ein einfach zu handhabendes, aber robustes Verfahren dar, um aus beliebigen sensitiven Daten ein sicheres Token zu erstellen. Es gewährleistet Vertraulichkeit, Integrität und Authentizität der Daten in einem Schritt. Im Kontext von Authentication-Tokens bedeutet dies: Ein Fernet-Token kann z.B. Benutzer- oder Gerätedaten enthalten (wie eine Identifikationsnummer, Berechtigungen, Zeitstempel), und nur der Auth-Service kann diese Daten wieder entschlüsseln und verifizieren. Ein Client, der ein solches Token erhält, kann dessen Inhalt nicht ohne Weiteres auslesen oder verändern – er nutzt es lediglich als \glqq Ausweis\grqq{}, den er bei weiteren Anfragen vorzeigt.

\subsection{Tokenisierung, Pre-Shared Key und Challenge-Response im Zusammenspiel}

Nachdem nun HMAC als Authentifizierungsnachweis und Fernet als Tokenisierungsmechanismus beschrieben wurden, soll abschließend beleuchtet werden, wie diese Verfahren im Auth-Service zusammenwirken. Der Authentifizierungsablauf lässt sich vereinfacht in folgende Phasen unterteilen: Initialer Schlüsselaustausch (\textit{Pre-Shared Key}), Challenge-Response-Authentisierung und Ausgabe eines Tokens (\textit{Tokenisierung}).

Pre-Shared Key (PSK): Zu Beginn wird angenommen, dass Client und Server einen gemeinsamen geheimen Schlüssel besitzen, der vorgängig sicher verteilt wurde. Ein solcher vorab geteilter Schlüssel (PSK) dient als Identitätsnachweis des Clients. PSK-Verfahren sind im IoT-Bereich üblich, wenn kein komplexes PKI-System für Zertifikate verfügbar oder gewünscht ist \cite{AvnetPSK}. Der PSK muss sicher auf dem Gerät hinterlegt und auf Serverseite gespeichert werden. Die Verwaltung mehrerer PSKs (für viele Geräte) erfordert Sorgfalt: Jeder PSK sollte einzigartig pro Client sein, und die Verteilung (Provisionierung) der PSKs an die Geräte muss über einen sicheren Kanal erfolgen. Im Vergleich zu Zertifikaten ist PSK-basierte Authentifizierung einfach und ressourcenschonend, skaliert aber weniger gut für sehr große Umgebungen, da der Server jeden PSK individuell managen muss und bei Kompromittierung eines einzelnen PSK kein zentrales Sperren (wie bei Zertifikat-CRLs) möglich ist \cite{AvnetPSK}. Nichtsdestotrotz bildet der PSK in unserer Betrachtung die vertrauliche Grundlage, auf der sich Client und Server gegenseitig Vertrauen schenken.

Challenge-Response-Authentifizierung: Um nun einen Client gegenüber dem Server zu authentisieren, ohne den PSK im Klartext zu übertragen, wird das Challenge-Response-Verfahren eingesetzt. Der Ablauf ist in Abbildung \ref{fig:auth-flow} dargestellt. Zunächst initiiert der Client eine Anmeldung beim Auth-Service. Der Server generiert daraufhin einen kryptographisch starken Zufallswert als \textit{Challenge} $N$ und schickt diesen an den Client (Schritt 1 in Abb.~\ref{fig:auth-flow}). Der Client berechnet mit dem PSK $K$ eine Antwort $R = \mathrm{HMAC}_K(N)$ (Schritt 2) und sendet $R$ zurück an den Server (Schritt 3). Der Server wiederum verifiziert $R$, indem er selbst $\mathrm{HMAC}_K(N)$ ausrechnet und mit dem empfangenen Wert vergleicht (Schritt 4). Ist die Prüfung erfolgreich, gilt der Client als authentifiziert. Andernfalls wird die Authentifizierung abgelehnt. Dieses Protokoll stellt sicher, dass nur ein Client mit Kenntnis des korrekten PSK die Challenge beantworten kann. Ein Angreifer, der die Kommunikation abhört, sieht nur die zufällige Challenge und den Response-Wert – beide nützen ihm ohne $K$ nichts. Selbst ein \textit{Replay}-Angriff (erneutes Senden einer alten gültigen Response) würde fehlschlagen, da der Server für jede Session eine neue Challenge wählt und alte Antworten somit ungültig sind. Wichtig ist, dass der PSK niemals das Gerät verlässt; es wird stets nur ein HMAC damit gebildet. Optional könnte man dieses Schema zur \textit{mutual authentication} erweitern, indem auch der Server sich gegenüber dem Client ausweist – etwa durch eine zweite Challenge in Gegenrichtung oder durch Ableiten gemeinsamer Session-Parameter. In vielen einfachen Anwendungen genügt jedoch die einseitige Authentifizierung des Clients.

\begin{figure}[h]
\centering
\begin{tikzpicture}[node distance=1.4cm, >=latex, every node/.style={font=\small}]
  \node (client) at (0,0) {\textbf{Client}};
  \node (server) at (6,0) {\textbf{Auth-Service}};

  % vertikale gestrichelte Linien
  \draw[dashed] (0,-0.3) -- (0,-4);
  \draw[dashed] (6,-0.3) -- (6,-4);

  % Nachrichten
  \draw[->] (0,-0.5) -- (6,-0.5) node[midway, above] {1. \textit{Login-Anfrage}};
  \draw[->] (6,-1.2) -- (0,-1.2) node[midway, below] {2. \textit{Challenge (Nonce)}};
  \draw[->] (0,-2.2) -- (6,-2.2) node[midway, above] {3. \textit{Response: HMAC(Nonce)}};
  \node at (7.2,-2.8) {4. \textit{Validierung}};
  \draw[->] (6,-3.5) -- (0,-3.5) node[midway, below] {5. \textit{Fernet-Token}};
\end{tikzpicture}
\caption{Challenge-Response-Authentifizierung mit Pre-Shared Key und Fernet-Token.}
\label{fig:auth-flow}
\end{figure}

Tokenisierung und Verwendung des Fernet-Tokens: Nach erfolgreich bestandener Challenge-Response möchte der Server dem Client eine Art \glqq Ticket\grqq{} ausstellen, das für nachfolgende Anfragen genutzt werden kann, damit der Client nicht bei jeder Kommunikation erneut eine Challenge-Response durchführen muss. Hier kommen Tokens ins Spiel. Ein \textit{Token} ist im Wesentlichen ein Datenelement, das die erfolgte Authentifizierung repräsentiert und vom Client bei weiteren Requests mitgeschickt werden kann (\glqq bearer token\grqq{} Prinzip). Im Auth-Service wird ein solcher Token nach der Authentifizierung generiert und dem Client übermittelt (Schritt 5 in Abb.~\ref{fig:auth-flow}).

Wichtig ist, dass dieser Token sicher gestaltet ist – andernfalls könnte ein Angreifer ihn abfangen und unberechtigt benutzen (\textit{Token Hijacking}). Durch die Verwendung von Fernet stellt der Auth-Service sicher, dass der Token verschlüsselt und signiert ist. Konkret kann der Server z.B. eine Token-Payload erstellen, die folgende Informationen enthält: die Client-ID oder Kennung, ggf. Autorisierungsinformationen (Rollen/Rechte) und einen Zeitstempel bzw. Gültigkeitsdauer. Diese Payload wird dann mit dem Fernet-Schlüssel des Servers verschlüsselt, wodurch der Token entsteht. Da nur der Server den Fernet-Schlüssel besitzt, kann auch nur er den Token später wieder entschlüsseln und validieren. Selbst der Client kennt den Inhalt des Tokens nicht (es sei denn, man gibt absichtlich gewisse Felder in Klartext, was hier nicht der Fall ist).

Beim Erhalt eines Tokens speichert der Client diesen meist lokal (z.B. im Speicher oder in einem sicheren Element) und fügt ihn bei künftigen API-Aufrufen an den Auth-Service oder andere geschützte Dienste in der Kommunikationsarchitektur hinzu (oft als Teil eines Headers oder Nachricht, analog zu einem Session-Cookie). Der Server seinerseits kann eingehende Tokens überprüfen: Er entschlüsselt mit dem Fernet-Key die Payload und prüft die Gültigkeit. Dabei wird automatisch die Integrität via HMAC im Fernet gewährleistet – ist der Token manipuliert oder von einem Dritten erzeugt worden, schlägt die HMAC-Prüfung fehl und der Token wird verworfen. Zusätzlich kann der Server den Timestamp im Token ansehen und beurteilen, ob der Token noch innerhalb der erlaubten Lebensdauer liegt. Ist der Token abgelaufen, fordert der Server vom Client eine neue Authentifizierung (Challenge-Response) oder verweigert die Anfrage.
Das Zusammenspiel dieser Komponenten schafft einen robusten Authentifizierungsmechanismus:
\begin{enumerate}
    \item Der PSK liefert ein geteiltes Geheimnis als Vertrauensbasis.
    \item Die Challenge-Response-Interaktion mit HMAC beweist die Kenntnis des PSK ohne ihn preiszugeben, schützt gegen Replay und ermöglicht einmalige Authentifizierungsvorgänge.
    \item Der Fernet-verschlüsselte Token fungiert als kurzfristiger Authentifizierungsnachweis, der vom Client für wiederholte Zugriffe verwendet werden kann, ohne jedes Mal den PSK bemühen zu müssen.
\end{enumerate}


Ein großer Vorteil dieser Tokenisierung ist die Entkopplung der weiteren Kommunikation vom ursprünglichen geheimen Schlüssel. Nach der Ausstellung des Tokens muss der Client den PSK für die Gültigkeitsdauer des Tokens nicht erneut verwenden oder preisgeben. Selbst wenn ein Token kompromittiert würde (etwa durch Diebstahl), würde das zugrunde liegende PSK dadurch nicht unmittelbar bekannt. Allerdings könnte ein gestohlener gültiger Token von einem Angreifer verwendet werden (\textit{Replay Attack} mit Token). Daher ist es essenziell, die Tokens ausreichend kurzlebig zu machen und die Übertragung derselben z.B. durch Transportverschlüsselung (TLS) zu schützen.
In summe ermöglicht die Kombination aus Challenge-Response und Token eine effiziente Authentifizierung: Der rechenintensivere Teil (HMAC-Berechnung) und ein Roundtrip erfolgen nur bei der Anmeldung, während für jede weitere Anfrage der Token als Authentifizierungsbeweis dient. Der Auth-Service kann damit gegenüber nachgelagerten Diensten oder bei späteren Verbindungen schnell die Identität des Clients prüfen, indem er den Token entschlüsselt, ohne erneut eine Nutzer-Eingabe oder ein komplexes Handshake-Protokoll zu durchlaufen. Dieses Muster findet sich in vielen Systemen wieder – man denke etwa an Web-Logins, bei denen nach einmaliger Passworteingabe ein Session-Cookie ausgestellt wird.

Abschließend sei erwähnt, dass Pre-Shared-Keys und symmetrische Tokens vor allem in Szenarien zum Einsatz kommen, wo eine Ende-zu-Ende-Vertrauensstellung zwischen genau zwei Parteien besteht (hier: Gerät und Auth-Service) und die Verwaltungsaufwände für PSKs tragbar sind. In großen verteilten Systemen könnte man alternativ auch Public-Key-Verfahren einsetzen (z.B. Client-Zertifikate und JWTs mit digitaler Signatur), die andere Vor- und Nachteile bieten. Für die Zwecke des betrachteten Auth-Service – vermutlich die Absicherung einer beschränkten Anzahl bekannter Devices – liefert die beschriebene Kombination jedoch einen gut geeigneten Kompromiss aus Sicherheit und Einfachheit. Alle verwendeten Bausteine (HMAC, Fernet/AES, Challenge-Response) gelten als kryptographisch solide. Somit bilden sie eine zuverlässige Grundlage, um im Auth-Service Vertraulichkeit, Integrität und Authentifizierung sicherzustellen und gegen gängige Angriffe (Abhören, Manipulation, Replay) gewappnet zu sein.


	\newpage
	\chapter{Anforderungen}
 	%test
	\section{Anforderungen an die entwickelten Schnittstellen-Services im Projekt Sensora}

Die Hauptaufgabe der Schnittstellen-Services lag in der Konzeption und Realisierung mehrerer Dienste, die als Vermittlungskomponenten zwischen Sensor-Controllern, Benutzerschnittstellen, dem zentralen Datenspeicher sowie einem Nachrichtenübertragungsmechanismus fungieren. Diese Services sind nicht Bestandteil der Steuerlogik auf Hardwareebene, sondern unterstützen den bidirektionalen Informationsaustausch und die sichere Verwaltung verteilter Geräteinstanzen.

Ziel dieses Kapitels ist es, die funktionalen und nicht-funktionalen Anforderungen an diese Schnittstellenkomponenten technologieoffen zu definieren. Die konkrete Wahl der eingesetzten Technologien sowie deren theoretische Fundierung erfolgt erst in den nachfolgenden Kapiteln.

\subsection{Allgemeine Anforderungen an alle Schnittstellen-Services}

Unabhängig von ihrer konkreten Aufgabe müssen sämtliche entwickelten Services folgende generelle Anforderungen erfüllen, die sich aus dem Aufbau des Gesamtsystems sowie den Entwicklungsprinzipien eines modernen verteilten Softwaresystems ergeben:

\begin{itemize}
  \item \textbf{Modularität:} Die Komponenten sollen entkoppelt und unabhängig voneinander betreibbar sein, sodass sie einzeln aktualisiert, getestet und ersetzt werden können.

  \item \textbf{Plattformunabhängigkeit:} Die Services müssen innerhalb einer containerisierten Umgebung lauffähig sein und dürfen keine plattformspezifischen Abhängigkeiten voraussetzen.

  \item \textbf{Fehlertoleranz:} Die Komponenten müssen so gestaltet sein, dass bei Ausfall abhängiger Systeme (z.\,B. Netzwerk, Datenbank) keine kritischen Fehler entstehen. Entsprechende Retry-Mechanismen und Fehlerprotokollierung sind vorzusehen.

  \item \textbf{Datensicherheit:} Sensible Informationen dürfen zu keinem Zeitpunkt im Klartext übertragen oder ungeschützt gespeichert werden. Eine sichere Authentifizierung und Zugriffsbeschränkung ist auf allen öffentlich erreichbaren Schnittstellen erforderlich.

  \item \textbf{Skalierbarkeit und Erweiterbarkeit:} Die Architektur der Services soll so beschaffen sein, dass zusätzliche Sensoren, Controller oder Benutzer ohne grundlegende Systemänderungen hinzugefügt werden können.

  \item \textbf{Zuverlässigkeit bei der Kommunikation:} Die Kommunikation zwischen Services sowie mit externen Geräten soll gegen Nachrichtenverlust abgesichert sein, insbesondere bei systemkritischen Operationen wie Steuerbefehlen oder Datenspeicherung.
\end{itemize}

\section{Anforderungsanalyse für die entwickelten Services}

Im Folgenden werden die spezifischen Anforderungen an die einzelnen Services beschrieben, wobei sowohl funktionale Anforderungen (Was soll der Service leisten?) als auch systemische Rahmenbedingungen betrachtet werden.

\subsection{Registrierungs- und Authentifizierungsservice (Auth-Service)}

Der Auth-Service bildet die sicherheitsrelevante Schnittstelle zur Einbindung verteilter Steuergeräte (Controller) in das System. Er ist verantwortlich für die Validierung, Registrierung und individuelle Konfiguration dieser Geräte.

\subsubsection{Funktionale Anforderungen}
\begin{itemize}
  \item Der Service muss es ermöglichen, neue Geräteinstanzen kontrolliert durch autorisierte Verwaltungsprozesse zu registrieren. Eine unbeaufsichtigte Selbstregistrierung ist auszuschließen.
  
  \item Für jede registrierte Geräteinstanz ist ein eindeutiger Identifikator sowie ein zugehöriges Zugriffsprofil zu erzeugen. Dieses Profil muss zur differenzierten Rechtevergabe geeignet sein.
  
  \item Die Authentifizierung eines Geräts gegenüber dem System muss über ein sicheres Verfahren erfolgen, das keine langfristige Speicherung von Klartextgeheimnissen erfordert und gleichzeitig eine manipulationssichere Verifikation ermöglicht.

  \item Nach erfolgreicher Authentifizierung soll eine Kommunikationsfähigkeit zwischen Gerät und System ermöglicht werden, die sich explizit auf festgelegte Kommunikationskanäle beschränkt.

  \item Der Service muss relevante Metadaten persistieren, sodass die zugehörige Gerätelogik (z.\,B. Sensor- oder Aktorzuordnung) systemweit nachvollzogen werden kann.
\end{itemize}

\subsubsection{Nicht-funktionale Anforderungen}
\begin{itemize}
  \item Die Registrierungslogik darf nur durch explizit autorisierte Systeme oder Nutzer aufrufbar sein.
  \item Die Kommunikation mit dem Service muss verschlüsselt erfolgen.
  \item Eine versehentliche Mehrfachregistrierung desselben Geräts ist zu erkennen und abzuweisen.
\end{itemize}

\subsection{E-Mail-Verifikationsservice (Mail-Service)}

Der Mail-Service übernimmt die Kommunikation mit Nutzenden zur Verifizierung neu angelegter Benutzerkonten. Seine primäre Aufgabe ist es, die Zustellung von zeitlich begrenzten Bestätigungslinks zu ermöglichen.

\subsubsection{Funktionale Anforderungen}
\begin{itemize}
  \item Der Service muss über eine Schnittstelle ansprechbar sein, über die Verifizierungsanfragen gestellt werden können.
  \item Nach Validierung der Anfrage ist ein einmalig nutzbarer Link zu generieren und über einen gängigen E-Mail-Dienst an den Empfänger zu übermitteln.
  \item Bei Aufruf des Verifikationslinks muss der Status des entsprechenden Benutzerkontos im System aktualisiert werden.
\end{itemize}

\subsubsection{Nicht-funktionale Anforderungen}
\begin{itemize}
  \item Nur autorisierte Systeme dürfen Anfragen zur Mailverifikation stellen.
  \item Die Verifizierung darf nur erfolgen, wenn die Kombination aus Benutzername und E-Mail-Adresse im System bekannt ist.
  \item Ein nicht eingelöster Verifizierungslink muss nach Ablauf einer definierten Zeitspanne seine Gültigkeit verlieren.
\end{itemize}

\subsection{Datenschreibdienst für Messdaten (Database Writer)}

Dieser Dienst verarbeitet eingehende Datenströme von Sensoren und persistiert die extrahierten Informationen strukturiert in einem relationalen Datensystem.

\subsubsection{Funktionale Anforderungen}
\begin{itemize}
  \item Der Dienst muss kontinuierlich eingehende Daten von Messgeräten empfangen, analysieren und in geeigneter Form speichern.
  \item Die Identität der messenden Einheit muss eindeutig ermittelbar sein. Ist die Einheit im System nicht bekannt, so muss diese bei Bedarf dynamisch registriert werden können.
  \item Die Speicherung darf nur erfolgen, wenn eine vollständige logische Zuordnung der Sensordaten zu einer Pflanze bzw. zum zugehörigen Anwendungsfall gegeben ist.
  \item Es müssen regelmäßig alle bekannten Sensoren auf Aktivität geprüft werden. Wird über einen definierten Zeitraum keine neue Messung empfangen, ist dies systemintern als Fehlverhalten zu markieren.
\end{itemize}

\subsubsection{Nicht-funktionale Anforderungen}
\begin{itemize}
  \item Der Dienst muss bei Netzwerkausfällen oder vorübergehenden Störungen in der Verbindung zur Datenbank stabil bleiben.
  \item Wiederholte Nachrichtenübertragungen dürfen nicht zu doppelten Datensätzen führen.
  \item Die Datenverarbeitung muss nachvollziehbar protokolliert werden, um Fehler oder Auffälligkeiten zu diagnostizieren.
\end{itemize}

\subsection{Service zur Steuerung von Zielwerten (Setpoint API)}

Über diesen Dienst können Zielwerte (Sollwerte) für bestimmte Sensoren oder Aktoren von außen gesetzt werden. Ziel ist es, die Regelgrößen im System dynamisch anpassen zu können.

\subsubsection{Funktionale Anforderungen}
\begin{itemize}
  \item Der Dienst muss eine externe Schnittstelle bereitstellen, über die Zielwerte für bestimmte Komponenten adressiert werden können.
  \item Der Dienst muss in der Lage sein, die übermittelten Steuerinformationen an die richtige Geräteinstanz weiterzuleiten.
  \item Die übertragenen Datenpakete müssen die Zuordnung zur Zielkomponente sowie den anvisierten Wert enthalten.
  \item Eine Dokumentation der Schnittstelle muss vorliegen, um die Integration in andere Komponenten zu ermöglichen.
\end{itemize}

\subsubsection{Nicht-funktionale Anforderungen}
\begin{itemize}
  \item Der Dienst darf keine gespeicherten Zustände über Zielwerte vorhalten.
  \item Die Kommunikation muss so gestaltet sein, dass sie eine zuverlässige Zustellung ermöglicht.
  \item Fehlkonfigurationen oder fehlerhafte Eingaben müssen zu validierbaren Fehlermeldungen führen.
\end{itemize}

\subsection{Konfigurationsdienst für Kommunikationsinfrastruktur (Solace Init)}

Dieser Service ist verantwortlich für die initiale Einrichtung der Kommunikationsinfrastruktur im System, insbesondere im Hinblick auf Messaging-Komponenten wie Queues oder themenbasierte Weiterleitungspfade.

\subsubsection{Funktionale Anforderungen}
\begin{itemize}
  \item Der Dienst muss eine maschinenlesbare Konfiguration interpretieren können, in der Kommunikationskanäle, Routingregeln und Zugriffspfade definiert sind.
  \item Auf Basis dieser Konfiguration muss die zugrunde liegende Infrastruktur um definierte Ressourcen ergänzt werden.
  \item Bereits existierende Konfigurationselemente dürfen dabei nicht überschrieben oder dupliziert werden.
\end{itemize}

\subsubsection{Nicht-funktionale Anforderungen}
\begin{itemize}
  \item Der Dienst muss so gestaltet sein, dass er mehrfach ausgeführt werden kann, ohne die Konsistenz der Kommunikationsstruktur zu gefährden.
  \item Fehlerhafte Konfigurationseinträge sind zu erkennen, protokollieren und dürfen die Verarbeitung nicht abbrechen.
\end{itemize}

	\newpage
	\chapter{Auswahl der Technologien}
	\section{Auswahl der Technologien f\"ur die entwickelten \\ Schnittstellen-Services}

Die im Rahmen des Projekts \textit{Sensora} entwickelten Schnittstellen-Services \"ubernehmen unterschiedliche Aufgaben innerhalb der Systemarchitektur, stellen jedoch alle zentrale Bausteine zur Kommunikations- und Steuerungsebene dar. Ihre Implementierung erfordert eine wohl\"uberlegte Auswahl geeigneter Technologien, die sowohl den funktionalen Anforderungen als auch den architekturellen, sicherheitstechnischen und betrieblichen Rahmenbedingungen gerecht werden.

In diesem Kapitel werden die getroffenen Technologieentscheidungen f\"ur jeden entwickelten Service einzeln erl\"autert, mit vergleichbaren Alternativen kontextualisiert und unter R\"uckbezug auf die definierten Anforderungen begr\"undet.

\subsection{Technologieauswahl f\"ur den Authentifizierungs- und Registrierungsdienst (Auth-Service)}

\subsubsection*{Zielsetzung und Kontext}

Der Auth-Service ist eine der sicherheitskritischsten Komponenten des Sensora-Systems. Er ist daf\"ur verantwortlich, neue IoT-Controller kontrolliert ins System aufzunehmen, ihnen eindeutige Kommunikationsidentit\"aten zuzuweisen und die Authentizit\"at dieser Ger\"ate eindeutig zu \"uberpr\"ufen. Zudem verwaltet er die Registrierung von Controllern in der zentralen Datenbank und erzeugt differenzierte Zugriffskan\"ale innerhalb der Messaging-Infrastruktur.

\subsubsection*{Verwendete Programmiersprache: Python}

F\"ur die Umsetzung des Auth-Service wurde die Programmiersprache \textbf{Python} gew\"ahlt. Python bietet eine sehr hohe Ausdrucksst\"arke bei gleichzeitig niedriger Komplexit\"at in der Syntax\cite{python_flask_prototyping}. Dies erlaubt eine fokussierte Umsetzung sicherheitskritischer Logik mit hoher Lesbarkeit und reduziertem Fehlerpotenzial. Die Sprache bietet native Unterst\"utzung f\"ur REST-APIs (via Flask), HMAC-Berechnungen (via \texttt{hashlib}) und JSON-Verarbeitung, was sie ideal f\"ur die Umsetzung des Auth-Service macht.

Im Vergleich zu Alternativen wie Java oder Go zeigt Python zwar leistungstechnische Schw\"achen bei hochfrequenten Systemen, bietet daf\"ur jedoch wesentlich h\"ohere Entwicklungsproduktiv\"itat \textendash{} ein entscheidender Vorteil im Rahmen eines begrenzten studentischen Projektzeitraums.

\paragraph*{Alternative Bewertung:}

Java bietet mit Spring Security zwar eine \"au\ss{}erst robuste Infrastruktur f\"ur Authentifizierungsmechanismen, ist jedoch deutlich komplexer im Deployment und ben\"otigt mehr Konfigurationsaufwand. Go bietet starke Performance und native Concurrency-Modelle, jedoch eine im Vergleich zu Python eingeschr\"ankte Bibliothekslandschaft f\"ur Security-Workflows. 

\paragraph*{Begr\"undung der Auswahl:}

Angesichts der Priorisierung von Entwicklungszeit, Lesbarkeit, Testbarkeit und Verf\"ugbarkeit passender Security-Tools wurde Python als die geeignetste Sprache f\"ur den Auth-Service identifiziert.

\subsubsection*{Authentifizierungsmechanismus: HMAC-basiertes Challenge-Response-Verfahren}

F\"ur die Authentifizierung der IoT-Ger\"ate wurde ein leichtgewichtiges Challenge-Response-Verfahren auf Basis von HMAC (Hash-based Message Authentication Code) implementiert. Diese Entscheidung basiert auf den Anforderungen an ein sicheres, serverseitig validierbares Verfahren ohne Notwendigkeit, Klartext-Zugangsdaten zu speichern oder zu \"ubertragen.

\paragraph*{Alternative Technologien:}
\begin{itemize}
  \item \textbf{OAuth 2.0:} Standard f\"ur Benutzer-Authentifizierung, jedoch komplex in der Implementierung f\"ur Maschinen-zu-Maschinen-Kommunikation.
  \item \textbf{JWT (JSON Web Token):} Geeignet f\"ur Token-basierte Sessions, jedoch problematisch hinsichtlich Zustandslosigkeit bei Ger\"aten mit hohem Sicherheitsanspruch.
  \item \textbf{Client-Zertifikate:} Sehr sicher, aber schwergewichtig in der Verwaltung f\"ur dynamisch zu registrierende IoT-Ger\"ate.
\end{itemize}

\paragraph*{Begr\"undung der Auswahl:}

Das HMAC-Verfahren erm\"oglicht die Validierung eines vorab generierten Ger\"ateschl\"ussels (Token), ohne diesen selbst senden zu m\"ussen. Durch serverseitige Generierung einer Challenge, die vom Ger\"at korrekt beantwortet werden muss, wird ein sicheres, manipulationsresistentes Protokoll etabliert, das gleichzeitig ressourcenschonend und einfach zu implementieren ist. Die Entscheidung orientiert sich damit an Best Practices f\"ur Ger\"ateauthentifizierung in Embedded-Umgebungen \cite{rfc2104_hmac}.

\subsubsection*{Kommunikationsmodell: REST-basierte API}

Der Service stellt seine Funktionalit\"at \"uber HTTP/REST-Endpunkte zur Verf\"ugung. Die Entscheidung f\"ur ein REST-Modell basiert auf der Notwendigkeit, den Dienst sowohl von Web-Frontends als auch von anderen Services (z.\,B. Mail-Service oder Solace-Konfiguration) ansprechbar zu machen.

\paragraph*{Alternative Technologien:}
\begin{itemize}
  \item \textbf{gRPC:} Hohe Effizienz, jedoch schwerer in Browser-Umgebungen integrierbar.
  \item \textbf{MQTT:} Bietet nur Publish/Subscribe, nicht geeignet f\"ur Request/Response-Workflows mit stateless APIs.
\end{itemize}

\paragraph*{Begr\"undung:}

REST bietet mit seiner stateless Architektur und breiten Toolunterst\"utzung (z.\,B. Swagger, Postman, curl) die optimale Basis f\"ur verteilte Systeme, insbesondere f\"ur administrative Operationen wie die Controller-Registrierung.

\subsubsection*{Interner Konfigurationsspeicher: JSON-Datei}

Die Informationen \"uber aktive Challenges und bereits registrierte Ger\"ate werden zus\"atzlich zur Datenbank in einer strukturierten JSON-Datei gespeichert. Dies erm\"oglicht schnelle Zugriffe auf tempor\"are Daten ohne Overhead einer persistierten Transaktion. Dieser pragmatische Kompromiss ist im Kontext studentischer Prototypen vertretbar.

\subsection{Technologieauswahl f\"ur den E-Mail-Verifikationsdienst (Mail-Service)}

\subsubsection*{Zielsetzung und Kontext}

Der Mail-Service stellt eine sicherheitsrelevante Verbindung zwischen Benutzerschnittstellen und Backend dar. Seine prim\"are Aufgabe ist die Versendung von E-Mail-\\Verifizierungslinks nach der Benutzerregistrierung. Damit fungiert er als zentrale Instanz zur initialen Verifikation von Benutzeridentit\"aten. Neben der Kommunikation mit einem SMTP-Server stellt der Dienst auch eine kontrollierte API zur Entgegennahme von Verifizierungsanfragen bereit.

\subsubsection*{Verwendete Programmiersprache: Python}

Die Entscheidung f\"ur Python basiert auf \"ahnlichen Argumenten wie beim Auth-Service. Python bietet durch Bibliotheken wie \texttt{smtplib} und \texttt{email.mime} einfache Schnittstellen zur Realisierung von SMTP-Kommunikation. Zus\"atzlich erlaubt die Verwendung von Flask eine unkomplizierte REST-Anbindung mit geringen Komplexit\"atsh\"urden. Im Rahmen des Projekts erm\"oglichte dies eine schnelle, wartbare und lesbare Implementierung des Dienstes.

\paragraph*{Vergleich mit Alternativen:}

Java (z.\,B. mit Spring Boot Mail) h\"atte eine robustere Infrastruktur geboten, w\"are jedoch mit erheblich h\"oherem Konfigurationsaufwand verbunden gewesen. Node.js wiederum bietet durch Pakete wie \texttt{nodemailer} eine gute Grundlage, ist jedoch im Team hinsichtlich Erfahrung weniger etabliert gewesen.

\paragraph*{Begr\"undung der Auswahl:}

Im Hinblick auf Entwicklungszeit, Lesbarkeit, Bibliotheksunterst\"utzung und Teamkompetenz wurde Python als pragmatische und effektive L\"osung gew\"ahlt.

\subsubsection*{E-Mail-Kommunikationsprotokoll: SMTP \"uber TLS}

F\"ur den Versand von Verifizierungsnachrichten wurde das Simple Mail Transfer Protocol (SMTP) in Kombination mit einer Transport Layer Security (TLS)-Verbindung eingesetzt. Diese Kombination bietet einen etablierten Standard f\"ur ausgehende Mail-Kommunikation mit Basisverschl\"usselung.\cite{smtp_tls}

\paragraph*{Alternativen:}

\begin{itemize}
  \item \textbf{REST-basierte Mailservices (z.\,B. SendGrid, Mailgun):} Bieten einfache APIs und statistische Auswertung, erfordern aber Drittanbieterkonten und externe Infrastruktur.
  \item \textbf{SMTP ohne Verschl\"usselung:} Unsicher und nicht datenschutzkonform.
\end{itemize}

\paragraph*{Begr\"undung:}

Die Entscheidung f\"ur SMTP \"uber TLS wurde getroffen, da ein Gmail für eine nicht zu Hohe Menge an Mails dies kostenlos ermöglicht und bereits zur Verf\"ugung stand und keine Drittanbieterintegration gew\"unscht war. Gleichzeitig konnten Sicherheitsanforderungen gewahrt bleiben.

\subsubsection*{Zugriffsschutz: Pre-Shared Key (PSK)}

Um zu verhindern, dass Dritte beliebige Verifizierungsanfragen senden, wurde der REST-Endpunkt des Mail-Service mit einem Pre-Shared Key abgesichert. Nur Systeme, die diesen kennen, k\"onnen autorisiert E-Mails ausl\"osen.

\paragraph*{Alternative Schutzmechanismen:}
\begin{itemize}
  \item \textbf{OAuth 2.0 oder API Tokens:} Sicher, aber unn\"otig komplex f\"ur einen geschlossenen Service.
  \item \textbf{Keine Authentifizierung:} Sicherheitsrisiko.
\end{itemize}

\paragraph*{Begr\"undung:}

Der Einsatz eines PSK ist ein einfacher, aber effektiver Schutzmechanismus, der in einem geschlossenen Systemumfeld wie Sensora praktikabel und ausreichend sicher ist.

\subsection{Technologieauswahl f\"ur den Datenpersistenzdienst (Database Writer)}

\subsubsection*{Zielsetzung und Kontext}

Der Database Writer nimmt eine zentrale Rolle in der persistenznahen Verarbeitung eingehender Sensordaten ein. Seine Hauptaufgabe besteht darin, kontinuierlich Datenpakete aus der Messaging-Infrastruktur entgegenzunehmen, diese auszuwerten und in strukturierter Form in das zugrunde liegende Datenbanksystem zu \"ubertragen. Eine besondere Herausforderung besteht dabei in der Notwendigkeit, sowohl hohe Verf\"ugbarkeit als auch Integrit\"at der Messdaten zu gew\"ahrleisten, selbst bei tempor\"aren Netzwerkproblemen oder inkonsistenten Eingangsnachrichten.

\subsubsection*{Programmiersprache: Python}

Die Implementierung des Database Writers erfolgte in Python. Die Entscheidung basiert auf der Kombination aus vorhandener Expertise im Projektteam, umfangreicher Bibliotheksunterstützung für JSON-Verarbeitung, Messaging-Systeme und Datenbankzugriffe sowie der leichten Wartbarkeit der Servicestruktur. In Python stehen mit Bibliotheken wie \texttt{paho-mqtt}\cite{python_mqtt}, \texttt{psycopg2} und \texttt{json} sofort einsatzfähige und stabile Werkzeuge für alle Teilaufgaben zur Verfügung.

\paragraph*{Vergleich mit Alternativen:}

Eine Implementierung in Go w\"are performant und speichereffizient, jedoch mit deutlich h\"oherem Aufwand bei der Bibliotheksintegration verbunden gewesen. Java h\"atte ebenfalls umfassende JDBC-basierte Anbindungen an relationale Datenbanken geboten, ist aber insbesondere f\"ur prototypische Implementierungen \"uberdimensioniert und aufw\"andig in Bezug auf Konfiguration und Deployment.

\paragraph*{Begr\"undung:}

Die Wahl von Python stellt einen optimalen Kompromiss zwischen Funktionalit\"at, Einfachheit und Flexibilit\"at dar. Dar\"uber hinaus konnte die gemeinsame Sprache mit den \"ubrigen Schnittstellen-Services genutzt werden, was die Homogenit\"at und Wartbarkeit der Gesamtplattform verbessert.

\subsubsection*{Messaging-Empfang: MQTT}

Der Database Writer konsumiert eingehende Nachrichten \"uber das MQTT-Protokoll, das vom zentralen Messaging-System vermittelt wird. MQTT wurde ausgew\"ahlt, da es mit seinem Publish/Subscribe-Paradigma und dem minimalen Protokoll-Overhead ideal auf intermittierende und latenzempfindliche Kommunikationsbeziehungen zwischen Sensoren und Auswertungssystemen zugeschnitten ist.\cite{mqtt_overview}

\paragraph*{Alternative Protokolle:}

\begin{itemize}
  \item \textbf{AMQP:} Industriestandard f\"ur Messaging, jedoch schwergewichtiger und nicht speziell f\"ur IoT optimiert.
  \item \textbf{HTTP Push:} Einfach zu implementieren, aber ungeeignet f\"ur kontinuierliche, bidirektionale Datenstr\"ome.
  \item \textbf{WebSockets:} Echtzeitf\"ahig, jedoch komplexer in der Integration mit klassischen Persistenzsystemen.
\end{itemize}

\paragraph*{Begr\"undung:}

MQTT bietet ein exzellentes Gleichgewicht zwischen Zuverl\"assigkeit, Ressourceneffizienz und Verbreitung im IoT-Bereich. Die Persistenzfunktionen des zugrunde liegenden Messaging-Brokers stellen zudem sicher, dass keine Daten durch kurzzeitige Netzwerkausf\"alle verloren gehen.

\subsubsection*{Fehlerbehandlung und Wiederholungsmechanismen}

Im Database Writer wurden explizite Retry-Mechanismen f\"ur die Datenbankanbindung integriert, um Datenverluste bei vor\"ubergehender Nichtverf\"ugbarkeit zu verhindern. Dies erfolgt \"uber ein warteschlangenbasiertes Zwischenspeichern nicht erfolgreicher Schreiboperationen, die periodisch erneut angesto\"ossen werden.

\paragraph*{Alternative Ans\"atze:}

\begin{itemize}
  \item \textbf{Fire-and-Forget-Ansatz:} Einfach, aber inakzeptabel bei Sicherheits- oder Zuverl\"assigkeitsanforderungen.
  \item \textbf{Transaktionsbasierte Protokolle:} W\"aren robuster, jedoch mit hohem Implementierungsaufwand verbunden.
\end{itemize}

\paragraph*{Begr\"undung:}

Die gew\"ahlte Methode stellt sicher, dass Datenverlust unter realistischen Bedingungen nahezu ausgeschlossen ist, ohne den Entwicklungsaufwand \"uber Geb\"uhr zu erh\"ohen. Die Wiederholungslogik kann bei Bedarf skalierbar erweitert werden.

\subsection{Technologieauswahl f\"ur die Zielwert-Schnittstelle (Setpoint API)}

\subsubsection*{Zielsetzung und Kontext}

Die Setpoint API erm\"oglicht die gezielte \"Ubertragung von Sollwerten an Steuercontroller, welche wiederum f\"ur die Regelung der Wasserzufuhr einzelner Pflanzen verantwortlich sind. Damit bildet sie eine Br\"ucke zwischen Benutzeranwendungen bzw. Systemkomponenten mit Steuerlogik und der dezentralen Aktorik des Systems. Die wesentliche Herausforderung besteht darin, eine flexible, zielgerichtete Kommunikation mit hoher Ausfallsicherheit und Ger\"atespezifit\"at zu gew\"ahrleisten.

\subsubsection*{Programmiersprache und Framework: Python mit Flask und Flasgger}

Die Implementierung erfolgte in Python, da bereits andere Systemkomponenten mit dieser Sprache realisiert wurden und eine hohe Wiederverwendbarkeit von Code sowie Konsistenz bei der Konfiguration gew\"ahrleistet werden sollte. Flask wurde als Web-Framework verwendet, da es eine schlanke Struktur mitbringt, sich hervorragend f\"ur RESTful-Services eignet und geringe Anforderungen an die Serverinfrastruktur stellt. Erg\"anzt wurde dies durch die Nutzung von Flasgger zur Generierung einer automatisierten Swagger-\\Dokumentation der REST-Endpunkte.\cite{flasgger_docs}

\paragraph*{Vergleich mit Alternativen:}

Alternativ w\"are FastAPI als modernere REST-Plattform denkbar gewesen, die native OpenAPI-Dokumentation, Validierung und asynchrone Verarbeitung unterst\"utzt. Allerdings wurde Flask aufgrund des stabileren Lern- und Erfahrungsstands im Team sowie vorhandener Funktionalit\"at bevorzugt.

\paragraph*{Begr\"undung:}

Die Kombination aus Flask und Flasgger erm\"oglichte eine wartbare und klar dokumentierte Schnittstelle mit kurzer Entwicklungszeit und einfacher Erweiterbarkeit. Die REST-Architektur passte gut zur Anforderung, kontrolliert und gezielt Steuerinformationen zu senden.

\subsubsection*{Kommunikationsmodell: Publish/Subscribe via MQTT \"uber Topics}

F\"ur die Weiterleitung der Sollwertnachrichten an den jeweiligen Controller wird das MQTT-Protokoll mit einer thematischen Strukturierung der Topics genutzt. Jeder Controller erh\"alt ein dediziertes Topic, dessen Aufbau die eindeutige Zuordnung der Nachricht erm\"oglicht. Dies entspricht dem Prinzip der Device-Adressierung in der IoT-Kommunikation.

\paragraph*{Vergleich mit Alternativen:}

\begin{itemize}
  \item \textbf{HTTP POST:} W\"are theoretisch f\"ur Push-Verhalten geeignet, erfordert jedoch persistente Adressen und erschwert dynamische Subscriptions.
  \item \textbf{AMQP:} Komplexer, mit mehr Overhead f\"ur das Szenario der gezielten Steuerung einzelner Endpunkte.
\end{itemize}

\paragraph*{Begr\"undung:}

MQTT erm\"oglicht eine sehr leichtgewichtige \"Ubermittlung mit der Option, persistente oder volatile Nachrichten zu senden. Die Topic-Struktur bietet eine flexible Adressierung ohne zus\"atzliche Verwaltungslogik. Die vorhandene Messaging-Infrastruktur mit Solace wurde konsequent weiterverwendet, was den Integrationsaufwand gering hielt.

\subsubsection*{Nachrichtenerstellung und Serialisierung}

Die Struktur der Sollwertnachricht wurde als JSON konzipiert, um sowohl Menschenlesbarkeit als auch Systemkompatibilit\"at zu sichern. Die Erstellung erfolgt mit Hilfe von Python-Bordmitteln, wodurch externe Abh\"angigkeiten minimiert werden konnten.

\paragraph*{Alternative Formate:}
\begin{itemize}
  \item \textbf{XML:} Etabliert, aber komplexer in der Verarbeitung.
  \item \textbf{Protobuf:} Effizient, aber f\"ur kleinere Projekte \"uberdimensioniert und weniger transparent.
\end{itemize}

\paragraph*{Begr\"undung:}

JSON bietet einen ausgezeichneten Kompromiss zwischen Standardisierung, Einfachheit und Interoperabilität. Es ist gut durch Firewalls und Broker-Systeme hindurch transportierbar und in praktisch jeder Sprache verarbeitbar.

\subsection{Technologieauswahl für den Initialisierungsskript-Dienst (Solace Init)}
\subsubsection*{Zielsetzung und Kontext}
Der Solace Init-Dienst wurde als automatisierter Initialisierungsmechanismus konzipiert, um beim Start des Systems die f\"ur die Kommunikationsarchitektur erforderlichen Messaging-Objekte auf dem Solace Broker anzulegen. Dabei handelt es sich insbesondere um Queues mit spezifischen Topic-Subscriptions, deren Struktur die Grundlage f\"ur das ger\"ateindividuelle Messaging im gesamten Sensora-System bildet. Ziel war es, eine einmalige, reproduzierbare und konfigurierbare Initialisierung ohne manuelle Eingriffe zu erm\"oglichen.

\subsubsection*{Begr\"undung der Broker-Wahl: Solace PubSub+ als Messaging-Plattform}
Die Wahl des Message Brokers stellt eine der zentralen Architekturentscheidungen im Sensora-Projekt dar. Die Anforderung bestand darin, ein performantes, fehlertolerantes und hochflexibles Messaging-System zu integrieren, das sowohl klassische Publish/Subscribe-Kommunikation als auch spezifische Anforderungen an Filterung, Persistenz und Authentifizierung erf\"ullt. Nach einem Vergleich mehrerer etablierter Systeme fiel die Entscheidung auf \textbf{Solace PubSub+}.\cite{solace_overview}

\paragraph*{Vergleich mit Alternativen:}

\begin{itemize}
  \item \textbf{Apache Kafka:} Hervorragend f\"ur Event-Streaming und hohe Datenvolumina, jedoch nicht nat\"urlich auf Topic-basiertes IoT-Publish/Subscribe ausgerichtet und ohne eingebaute Message Routing Features wie Topic Wildcards.\cite{mqtt_vs_kafka}
  \item \textbf{RabbitMQ:} Solide AMQP-Implementierung mit guter Dokumentation, jedoch schw\"acher in Bezug auf native MQTT-Unterst\"utzung, dynamische Topic-Strukturen und granulare Zugriffssteuerung auf Topics.
  \item \textbf{Mosquitto:} Leichtgewichtig und speziell auf MQTT ausgelegt, aber eingeschr\"ankt in Bezug auf Sicherheitsfeatures, Persistence-Mechanismen und Administration auf Enterprise-Niveau.
\end{itemize}

\paragraph*{St\"arken von Solace:}

Solace PubSub+ verbindet als Enterprise-Grade-Plattform mehrere Vorteile, die sich direkt aus den Anforderungen des Sensora-Projekts ergeben:

\begin{itemize}
  \item \textbf{Native Unterst\"utzung mehrerer Protokolle:} Solace unterst\"utzt MQTT, REST, AMQP und WebSockets nativ auf einem Broker. Damit konnten sowohl sensornahe Kommunikation \textit{(MQTT)} als auch service-interne Schnittstellen \textit{(REST)} integriert werden, ohne unterschiedliche Middleware-L\"osungen kombinieren zu m\"ussen.
  \item \textbf{Feingranulare Topic-Strukturen und Wildcards:} F\"ur die Adressierung einzelner Controller oder Sensortypen wurden hierarchische Topics mit Wildcard-\\Unterst\"utzung genutzt, wodurch sich flexible Abonnements realisieren lie\ss{}en.
  \item \textbf{Skalierbare Persistenzmechanismen:} Solace bietet sowohl volatile als auch persistente Delivery-Modes mit garantierter Zustellung, was f\"ur zeitkritische Steuerinformationen entscheidend ist.
  \item \textbf{Zentrale Administration via SEMPv2:} Die REST-basierte Konfigurationsschnittstelle (SEMPv2) erlaubt automatisierte, containerkompatible Initialisierungsskripte wie den hier beschriebenen Init-Service.
  \item \textbf{MQTT optimiert f\"ur IoT:} Die MQTT-Implementierung von Solace ist mit Fokus auf Latenzreduktion, Lastverteilung und Delivery-Garantien implementiert und eignet sich ideal f\"ur Embedded-Ger\"atekommunikation.
  \item \textbf{Security-Features:} ACL-Management, Authentifizierung auf Benutzer- und Topic-Ebene sowie TLS-Unterst\"utzung erm\"oglichen ein differenziertes Sicherheitskonzept.
\end{itemize}

\paragraph*{Zusammenfassende Begr\"undung:}

Solace PubSub+ wurde gew\"ahlt, da es in einzigartiger Weise hohe Anspr\"uche an Zuverl\"assigkeit, Integrationstiefe und Skalierbarkeit erf\"ullt. Die native Unterst\"utzung von MQTT ist f\"ur IoT-Anwendungen essenziell, w\"ahrend die gleichzeitige Verf\"ugbarkeit von Management- und Sicherheitsfunktionen auf Enterprise-Niveau den reibungslosen Betrieb in containerisierten Umgebungen sicherstellt. Dar\"uber hinaus empfiehlt auch der Hersteller Solace selbst den Einsatz von Python f\"ur schnelle Prototypenentwicklung im Bereich IoT \cite{solace_python_doc}.

\subsubsection*{Konfigurationsbasis: JSON-Datei}

Als zentrales Format f\"ur die Definition der zu erstellenden Queues und ihrer jeweiligen Subscriptions wurde eine strukturierte JSON-Datei verwendet. Dieses Format erlaubt eine deklarative Spezifikation der gesamten Messaging-Infrastruktur und kann sowohl von Menschen editiert als auch maschinell verarbeitet werden.\cite{json_best_practices}

\paragraph*{Vergleich mit Alternativen:}

\begin{itemize}
  \item \textbf{YAML:} Ebenfalls menschenlesbar, jedoch fehleranf\"alliger bei komplexeren Strukturen und nicht nativ durch alle Python-Standardbibliotheken unterst\"utzt.
  \item \textbf{XML:} Formal stark, aber syntaktisch schwergewichtig und in der Praxis f\"ur Konfigurationszwecke oft \"uberdimensioniert.
\end{itemize}

\paragraph*{Begr\"undung:}

JSON erf\"ullt im Kontext des Projekts das optimale Gleichgewicht zwischen Lesbarkeit, Standardisierung und Softwareunterst\"utzung. Es erm\"oglicht eine flexible Erweiterung der Initialisierungskonfiguration, z.\,B. durch Hinzuf\"ugen weiterer Queues oder komplexerer Subscription-Filter, ohne strukturelle Anpassungen am Code erforderlich zu machen.

\subsubsection*{Schnittstelle zur Messaging-Infrastruktur: Solace SEMPv2 API}

Die eigentliche Initialisierung erfolgt \"uber HTTP-Requests an die SEMPv2-API von Solace PubSub+, einer offiziellen Verwaltungs- und Konfigurationsschnittstelle des Brokers. Diese REST-basierte Schnittstelle erlaubt das Anlegen, Konfigurieren und Pr\"ufen von Messaging-Komponenten wie Queues, Topics und ACLs im laufenden Betrieb.

\paragraph*{Vergleich mit Alternativen:}

\begin{itemize}
  \item \textbf{Admin GUI:} Bedienerfreundlich, aber nicht automatisierbar und nicht reproduzierbar.
  \item \textbf{CLI-Tools (solacectl):} Eher f\"ur DevOps-Prozesse geeignet, jedoch aufwendiger in der Einbettung in einen containerisierten Microservice.
\end{itemize}

\paragraph*{Begr\"undung:}

Die REST-API von Solace war f\"ur den Projektkontext besonders geeignet, da sie eine serviceinterne und vollautomatische Ansteuerung erm\"oglichte. Die Authentifizierung konnte \"uber Umgebungsvariablen geregelt werden, die in Docker Compose konfiguriert wurden, sodass die Schnittstelle sowohl sicher als auch einfach nutzbar war. Zudem erm\"oglicht SEMPv2 die Definition granulare Subscription-Filter direkt beim Queue-Erstellen, was eine exakte Abbildung der Messaging-Logik erm\"oglicht.

\subsubsection*{Fehlertoleranz und Wiederanlauf}

Das Init-Skript enth\"alt einfache Mechanismen zur Wiederanlaufbarkeit. Existierende Queues werden nicht erneut erstellt, sondern \"ubersprungen oder aktualisiert. Fehler bei der Kommunikation mit dem Solace-Broker werden protokolliert, und das Skript kann bei Bedarf mehrfach ohne Seiteneffekte ausgef\"uhrt werden.

\paragraph*{Begr\"undung:}

Diese Idempotenz ist essenziell f\"ur automatisierte Umgebungen, etwa bei der Nutzung von Docker Compose oder CI/CD-Pipelines. Durch sie wird vermieden, dass der Initialisierungsprozess bei einem Neustart ungewollt Konfigurationsfehler verursacht oder bestehende Objekte zerst\"ort.

	\newpage
	\chapter{Umsetzung}
	\section{Auth-Service: Geräteregistrierung und HMAC-Authentifizierung}
Der Auth-Service implementiert die sichere Registrierung und Anmeldung von Controllern mittels eines Challenge-Response-Verfahrens auf Basis eines vorab geteilten Geheimnisses (PSK). Dieser Dienst ist als Flask-Webanwendung realisiert und stellt HTTP-Routen für die Controller-Initialisierung, -Verifikation sowie eine Admin-Registrierung bereit. Er bildet damit das sicherheitskritische Bindeglied zwischen neuen Geräten und der Systeminfrastruktur: 
Geräte erhalten hier ihre individuellen Messaging-Zugangsdaten, sofern sie einen kryptographischen Besitznachweis erbringen. Im Folgenden werden die Architektur und der Ablauf des Auth-Service beschrieben, gefolgt von besonderen Implementierungsdetails wie der PSK-Überprüfung, Wiederanlaufbarkeit und der automatischen Broker-Konfiguration.
\subsection{Architektur und persistente Datenhaltung}
Der Auth-Service ist als Flask-Applikation mit dokumentierter REST-API (via Swagger/Flasgger) umgesetzt. Kern der Implementierung ist eine zentrale Konfigurationsdatei (auth\_config.json), die folgende Informationen persistent speichert:

\begin{enumerate}
    \item Authorized Controllers: eine Liste aller registrierten Controller mit deren Controller-ID (eindeutige Kennung), zugehörigem PSK (token) sowie einem Hash dieses Tokens (token\_hash). Zusätzlich werden Metadaten wie Modell, Besitzer (Benutzername) und Beschreibung gespeichert. Diese Datei dient als kleine lokale Datenbank, um bereits registrierte Geräte und ihre Geheimnisse nachzuhalten.
    \item Active Challenges: temporäre Herausforderungen (Challenges) im laufenden Authentifizierungsprozess.
    Für jeden angeforderten Auth-Vorgang wird ein zufälliger Challenge-String erzeugt und unter dem Schlüssel des token\_hash zwischengespeichert.
    Dies ermöglicht es, eingehende Antworten eindeutig der zuvor ausgegebenen Challenge zuzuordnen, selbst bei parallelen Anfragen.
    \item Solace Credentials: bereits für Controller erstellte Zugangs-Credentials für den Message Broker (Solace). Pro Gerät wird hier der erzeugte Broker-Benutzername und das Passwort abgelegt, um bei wiederholter Authentifizierung nicht erneut einen Broker-Account anlegen zu müssen.
\end{enumerate}
Die Konfigurationsdaten werden auf dem Container-Dateisystem unter /config/auth\_config.json verwaltet und durch Volume-Mounting persistent gehalten. Dadurch bleiben Registrierung und vergebene Credentials auch beim Neustart des Dienstes erhalten – ein wesentlicher Aspekt für Wiederanlaufbarkeit. Ergänzend hält der Auth-Service eine Datenbankverbindung (PostgreSQL via psycopg2) bereit, um neue Controller auch in der zentralen Systemdatenbank (sensora) zu registrieren. In der Tabelle sensora.controllers werden u.a. Device-ID, Modell, Besitzer und ein vom System generierter geheimer Schlüssel abgelegt. Letzterer unterscheidet sich vom PSK und dient ggf. anderen Zwecken (z.B. der Kommunikation mit externen Anwendungen), ohne das eigentliche PSK offenzulegen. Die doppelte Ablage (Datei und DB) der Registrierungsdaten mag auf den ersten Blick redundant erscheinen, ermöglicht jedoch sowohl schnelle lokale Zugriffe während des Auth-Handshakes als auch die Integration ins relationale Gesamtdatenmodell.

\subsection{Ablauf der Geräte-Registrierung und Authentifizierung}
Der Auth-Service bietet drei Haupt-Endpoints, die den Lebenszyklus eines Geräts abbilden: (a) Registrierung eines neuen Controllers (nur Admin), (b) Initialisierung des Authentifizierungs-Handshakes durch das Gerät und (c) Verifikation der Challenge-Response. 

Der Authentifizierungsablauf erfolgt in mehreren Schritten:
\begin{enumerate}
    \item Admin-Registrierung: Zunächst muss ein neuer Controller in das System aufgenommen werden. Ein Administrator (oder ein automatisierter Setup-Prozess) ruft dazu den Endpoint /api/admin/controller auf und übermittelt zumindest den gewünschten Besitzer (username) sowie optional eine vorgegebene Controller-ID und Modellbeschreibung. Der Request ist durch einen speziellen Admin-API-Key (Header X-Admin-Key) geschützt, sodass nur berechtigte Instanzen Geräte hinzufügen können. Bei Aufruf generiert der Service serverseitig ein zufälliges Token (PSK) für das Gerät (hier als UUID v4). Zusätzlich wird ein Token-Hash berechnet: mittels HMAC-SHA256 über das Token mit einem globalen Server-Geheimnis (TOKEN\_SECRET). Dieser Hash dient als öffentlicher Identifikator des Geräts, der das eigentliche Token nicht preisgibt. Anschließend werden die Gerätedaten in der auth\_config.json persistiert (authorized\_controllers) und ein Eintrag in der DB-Tabelle controllers erzeugt. Der Response an den Admin enthält Controller-ID, Token und Token-Hash, welche sicher an das Gerät für die Inbetriebnahme weitergegeben werden (z.B. manuell oder per Provisioning-App). Dieser einmalige Out-of-Band-Schritt stellt sicher, dass jedes Gerät ein individuelles geheimes Token besitzt, das dem Server bekannt ist.
    \item Challenge-Anforderung (Gerät → Auth-Service): Hat das Gerät vom Admin sein Token erhalten und wird erstmals online genommen, initiiert es den Authentifizierungsprozess über den Endpoint /api/controller/init. Dabei sendet das Gerät nur den Hash seines Tokens (token\_hash) im Request – das eigentliche Token bleibt geheim und wird nie direkt übertragen. Der Auth-Service prüft den Request und generiert eine kryptographisch zufällige Challenge (16 Byte Hex-String via secrets.token\_hex) als Antwort. Diese Challenge wird im Server-Konfigurationsspeicher unter active\_challenges zusammen mit einem Zeitstempel abgelegt, indiziert durch den token\_hash. Die Response an das Gerät enthält die Challenge als JSON. Aus Sicherheitsgründen findet an dieser Stelle noch keine Verifizierung statt – auch ein unbekannter Token-Hash erhält (vorläufig) eine Challenge. Der eigentliche Abgleich erfolgt erst im nächsten Schritt. Dieses zweistufige Verfahren erhöht die Sicherheit, da ein Angreifer ohne Besitz des PSK aus der Challenge alleine keinen Zugang erlangt. Wichtig ist, dass jede Challenge unvorhersehbar und einmalig ist, um Replay-Angriffe auszuschließen. Der Auth-Service stellt dies durch Verwendung eines Kryptografie-Moduls sicher (Python secrets \cite{pythonSecrets} bietet laut Dokumentation einen sicheren Zufallszahlengenerator für solche Token).
    \item Challenge-Response (Gerät → Auth-Service): Nach Erhalt der Challenge berechnet das Gerät die Antwort: Es verwendet sein geheimes Token und wendet darauf die gleiche Hash-Funktion an, die auch der Server kennt. Im Code wird dazu HMAC-SHA256 genutzt, wobei das Token als Schlüssel und der Challenge-String als Nachricht dient. Das Ergebnis ist die Challenge-Response (Hex-String). Diese Antwort schickt das Gerät zurück an den Auth-Service, zusammen mit seinem token\_hash (zur Identifikation der Challenge) und meist einem Benutzernamen oder Kontoinformationen des Eigentümers. Der Auth-Service schlägt nun in seiner Konfiguration den Eintrag zum token\_hash nach: Dort findet er das ursprünglich registrierte Token (PSK) und die zugehörige Controller-ID. Sollte kein Eintrag existieren, wird der Prozess abgebrochen – das Gerät war nie registriert oder der Hash unbekannt (Fehler 403). Andernfalls vergleicht der Server die vom Gerät gesendete HMAC-Antwort mit dem erwarteten Wert, den er selbst berechnet (hmac.compare\_digest() verhindert Timing-Angriffe bei der String-Bewertung). Stimmen Response und eigener Wert überein, ist bewiesen, dass das Gerät das geheime Token besitzt und somit authentisch ist. Dieses Challenge-Response-Verfahren stellt eine sichere Authentifizierung dar\cite{hmacRFC}, ohne das PSK selbst über das Netzwerk zu senden. Ein abgehörter Challenge/Response-Wert kann später nicht wiederverwendet werden, da bei der nächsten Anmeldung eine andere Challenge zum Einsatz kommt (kein statisches Passwort)
    \item Broker-Zugangsdaten erstellen: Nach erfolgreicher Verifizierung entfernt der Auth-Service die verwendete Challenge (Verbrauch der einmaligen Challenge) und fährt mit der Provisionierung des Geräts für das Messaging-System fort. Hier greift eine weitere wichtige Implementierungsentscheidung: Der Service konfiguriert den Solace-MQTT-Broker dynamisch über dessen Management-API (SEMP v2). Konkret wird geprüft, ob für den Controller bereits Broker-Zugangsdaten in solace\_credentials vorliegen. Ist dies der erste erfolgreiche Auth-Vorgang für dieses Gerät, generiert der Service mittels der Hilfsfunktion create\_solace\_user(controller\_id) einen dedizierten Broker-Account:
    \begin{enumerate}
        \item Es wird ein eindeutiger Client-Username für den Controller erstellt (z.B. controller\_ab12... gekürzt auf Basis der ID) und ein zufälliges Passwort (secrets.token\_urlsafe(32)).
        \item Für feingranulare Zugriffskontrolle richtet der Service ein eigenes ACL-Profil auf dem Broker ein. Dies geschieht über HTTP-Aufrufe an die SEMPv2-Konfigurationsschnittstelle des Solace-Brokers. Das ACL-Profil für den Controller erlaubt ausschließlich die für diesen Anwendungsfall nötigen Operationen:
        \begin{enumerate}
            \item Publish-Erlaubnis: Das Gerät darf nur auf seinem eigenen Topic-Pfad Messwerte publizieren. Im Prototoll wird das Topic-Muster sensora/v1/send/{controller\_id} verwendet. Über SEMP wird ein Topic Exception hinzugefügt, die genau dieses Topic für Publish freigibt (alle anderen werden per Default “disallow” gesetzt).
            \item Subscribe-Erlaubnis: Analog erhält das Gerät das Recht, Sollwert-Nachrichten zu abonnieren, die an sein spezifisches Target-Topic gesendet werden. Dies ist typischerweise sensora/v1/receive/{controller\_id}/targetValues. Auch hierfür wird programmgesteuert eine Ausnahme im ACL-Profil hinterlegt. Damit ist sichergestellt, dass der Controller nur Nachrichten empfängt, die explizit an ihn adressiert sind (und z.B. keine fremden Gerätedaten).
        \end{enumerate}
        \item Nachdem ACL-Profil und Berechtigungen erfolgreich angelegt wurden, erstellt der Service über die SEMP-API den Broker-Client-User und weist ihm dieses ACL-Profil zu. Sollte einer der Schritte fehlschlagen (z.B. aufgrund bereits existierender Einträge oder Verbindungsfehler zum Broker), bricht die Funktion ab und der Auth-Vorgang resultiert in einem Server-Fehler (HTTP 500). Im Erfolgsfall erhält der Auth-Service nun ein Credential-Paket bestehend aus Broker-URL, Username und Passwort für den neuen Controller.
    \end{enumerate}
    Die Nutzung der Solace Element Management Protocol API ermöglicht es, die Broker-Konfiguration zu automatisieren. Solace SEMP ist eine RESTful-Schnittstelle\cite{SolaceSEMP}, die das Anlegen von Objekten (Queues, Benutzer, ACLs etc.) per Skript erlaubt. Dadurch wird eine dynamische Inbetriebnahme neuer Geräte ohne manuelle Eingriffe möglich – ein wichtiger Vorteil im IoT-Kontext. Durch entsprechende Fehlerbehandlung (Auswerten des HTTP-Status: 200 Erfolg, 409 „Conflict“ bei bereits vorhandenen Ressourcen) ist die Funktion weitgehend wiederholbar, ohne Inkonsistenzen zu erzeugen. Beispielsweise würde ein erneuter Aufruf für denselben Controller erkennen, dass dessen Benutzerkonto schon existiert (der Auth-Service speichert dies ja auch in seiner config) und überspringt die Neuanlage. So bleibt der Prozess idempotent und ein Gerät könnte die Authentifizierung bei Bedarf erneut durchlaufen, um z.B. verlorene Credentials abzurufen.
    \item Registrierung abschließen und Antwort an Gerät: Zum Abschluss der /verify-Route führt der Auth-Service noch zwei Aktionen aus: (a) Eintrag in der Systemdatenbank: Falls noch nicht geschehen, wird der Controller endgültig in der sensora.controllers-Tabelle der DB vermerkt (inkl. Besitzverknüpfung zum Benutzerkonto, was zuvor im Request mitgesendet wurde). Außerdem kann hier optional ein Default-Sensor für den Controller in der DB angelegt werden (im Code wird bspw. ein Platzhalter-Sensor für Temperatur erstellt, um direkt Messwerte speichern zu können). (b) Secure Credential Delivery: Die vom Broker erzeugten Verbindungsdaten (Username/Passwort, Host) müssen nun dem Gerät mitgeteilt werden. Da diese Angaben sehr sensitiv sind, wurden besondere Maßnahmen getroffen, um sie vertraulich und integer zum Gerät zu übertragen. Der Server generiert zunächst einen einmaligen Session Key (eine randomisierte Byte-Sequenz mittels Fernet.generate\_key()), mit dem er die Credentials symmetrisch verschlüsselt (Fernet nutzt intern AES-128 in GCM-Modus\cite{fernetSpec} mit eingebauter HMAC für Integrität). Die so entstehende Ciphertext-Nutzlast wird als encrypted\_credentials bereitgestellt. Zusätzlich erzeugt der Server einen Credential-HMAC (credential\_key), indem er den Session Key mit dem ursprünglichen Geräte-PSK mittels HMAC-SHA256 signiert. Anschließend sendet der Auth-Service dem Gerät folgende Daten im JSON-Response:
    \begin{enumerate}
        \item session\_key: der im Base64-Format kodierte symmetrische Schlüssel,
        \item credential\_key: der HMAC (Hex-String) zur Absicherung,
        \item encrypted\_credentials: die verschlüsselten Broker-Zugangsdaten (Base64-Text).
    \end{enumerate}
    Diese Konstruktion erlaubt es dem Gerät, die erhaltenen Credentials auf Vertrauenswürdigkeit zu prüfen: Nur wenn es den gleichen HMAC über den Session Key mit seinem PSK berechnet und dieser mit dem credential\_key übereinstimmt, stammen die Daten eindeutig vom Auth-Service (der das PSK kennt). Damit ist ein Schutz gegen Man-in-the-Middle-Angriffe erreicht, selbst wenn kein vollwertiger TLS-Kanal vorhanden wäre – ein Angreifer könnte zwar den Session Key und Ciphertext stehlen, hätte aber ohne PSK keine Möglichkeit, gültige Daten vorzutäuschen. Nach erfolgreicher HMAC-Prüfung entschlüsselt das Gerät mit dem Session Key die Credentials und erhält so seinen persönlichen Broker-Login. Ab diesem Zeitpunkt kann sich der Controller am MQTT-Broker anmelden und regulär Sensordaten austauschen. Der einmalig verwendete Session Key ist nun obsolet.
\end{enumerate}
Abschließend bestätigt der Auth-Service dem Gerät die erfolgreiche Verifikation mit HTTP 200. Intern protokolliert er die erfolgreiche Authentifizierung und der Prozess ist abgeschlossen. Jegliche Fehlersituationen unterwegs (z.B. falscher Token, falsche Response, fehlende Eingabefelder) wurden mit aussagekräftigen HTTP-Codes (400 Bad Request, 403 Forbidden) und Log-Meldungen abgefangen, sodass das Gerät bzw. der Administrator direkt Rückmeldung über den Grund eines Scheiterns erhalten.


\section{Mail-Service: E-Mail-Verifikation von Benutzerkonten}
Der Mail-Service ist ein eigenständiger Webservice, der die Verifizierung von Benutzer-E-Mailadressen übernimmt. Im Gesamtsystem wird dieser Service genutzt, um nach einer Benutzerregistrierung sicherzustellen, dass die angegebene E-Mail dem Nutzer gehört und erreichbar ist – ein gängiges Verfahren, um Kontoaktivierungen durch den Nutzer selbst via Klick auf einen Bestätigungslink durchzuführen. Der Mail-Service wurde mit FastAPI (Python) umgesetzt, was die Erstellung asynchroner HTTP-Handler ermöglicht. Die Hauptaufgaben des Dienstes sind: Empfang der Verifikationsanfrage, Validierung mittels eines Pre-Shared Key (zur Absicherung interner Aufrufe), Generierung eines eindeutigen Bestätigungs-Tokens, Versand einer E-Mail mit Bestätigungslink via SMTP und abschließend die Verarbeitung des Bestätigungs-Clicks (Aktivierung des Kontos in der Datenbank).
\subsection{Architektur und Ablauf der E-Mail-Verifikation}
Der Mail-Service verfügt über zwei wesentliche Endpoints: einen POST-Endpoint /verify zum Anfordern einer Verifikationsmail und einen GET-Endpoint /confirm/{username}/{token} zum Bestätigen. Intern nutzt der Service eine PostgreSQL-Datenbankverbindung (asynchron via asyncpg), um Benutzerdatensätze zu prüfen und zu aktualisieren. Der Ablauf lässt sich wie folgt zusammenfassen:
\begin{enumerate}
    \item Anfrage zur Verifikation (POST /verify): Diese Schnittstelle wird vom übergeordneten System (z.B. dem Web-Frontend oder einem anderen Service) aufgerufen, sobald ein Benutzer eine Registrierung abgeschlossen hat oder eine E-Mail-Bestätigung angefordert wird. Der Request enthält typischerweise den Benutzernamen und die E-Mail-Adresse des Kontos. Zusätzlich erwartet der Service einen geheimen Schlüssel (key), der mitgeschickt wird. Dieser PSK (Mailservice) ist eine einfache Sicherungsmaßnahme, damit nur autorisierte Systeme (etwa das Frontend-Servermodul) den Versand von Verifizierungs-Mails auslösen können – damit wird verhindert, dass Unbefugte massenhaft Verifikations-E-Mails über die öffentliche API triggern. Der Service prüft also zuerst, ob der mitgesandte Schlüssel mit dem in den Umgebungsvariablen hinterlegten Wert (MAILSERVICE\_PSK) übereinstimmt. Ist dies nicht der Fall, wird mit HTTP 403 abgebrochen.
    Ist die Anfrage autorisiert, wird die angegebene Kombination aus username und mail in der Datenbank gesucht (SELECT * FROM sensora.users WHERE username=\%s AND mail=\%s). Nur wenn ein entsprechender Benutzeraccount existiert und noch als inaktiv markiert ist (dies wird indirekt geprüft, indem z.B. ein Feld active in der DB auf FALSE stehen sollte – im Code wird bei Nichtexistenz direkt 404 gemeldet), wird der Verifikationsprozess fortgesetzt. Im nächsten Schritt erzeugt der Service ein zufälliges Token als einmaligen Bestätigungscode. Hierzu wird Python secrets.token\_urlsafe(16) verwendet\cite{pythonSecrets}, was einen ~22 Zeichen langen kryptographisch sicheren String liefert. Das Token wird in einer in-memory Datenstruktur (tokens Dictionary) unter dem Schlüssel des Benutzernamens gespeichert. Anschließend wird ein Bestätigungslink erstellt, der die URL des Confirmation-Endpoints enthält (inkl. Pfadparameter für Username und Token). Dieser Link hat z.B. die Form: https://meinserver/confirm/alice/AbCdEfGh... – er enthält also das geheime Token.
    Nun versendet der Mail-Service eine E-Mail an die Adresse des Nutzers. Dafür wird ein SMTP-Server (hier Gmail SMTP auf Port 587) verwendet. Über Pythons smtplibwird eine TLS-geschützte Verbindung aufgebaut, der Mailaccount authentifiziert (SMTP-User und Passwort liegen in den Settings) und dann eine Textnachricht verschickt. Der E-Mail-Inhalt besteht aus einem kurzen Text mit der Aufforderung, den Link anzuklicken, um die Registrierung abzuschließen. Absender und Betreff sind entsprechend gesetzt (z.B. "Bitte bestätige deine E-Mail"). Nach erfolgreichem Versand gibt der/verify-Endpoint eine Erfolgsmeldung zurück ({"message": "Verification email sent."}mit HTTP 200). Fehlerfälle:
    Wenn die E-Mail-Adresse nicht existiert oder der DB-Zugriff fehlschlägt, wird ein HTTP 404 bzw. 500 zurückgegeben. Ein falscher PSK führt zu 403. Falls der SMTP-Versand scheitert (Exception), wird diese von FastAPI als Serverfehler zurück an den Aufrufer propagiert – in einer robusteren Version könnte man hier spezifisch mitHTTPException` antworten, doch im gegebenen Code wird auf die eingebaute Exception-Behandlung vertraut.
    \item Bestätigungsaufruf (GET /confirm/{username}/{token}): Diese Route wird aufgerufen, wenn der Benutzer den Link in der Verifikationsmail anklickt. In einem üblichen Web-Anwendungsfluss würde dieser Link z.B. zu einer Erfolgsmeldungsseite führen. Der Mail-Service übernimmt hier im Hintergrund die Aktivierung des Benutzerkontos. Er prüft zunächst, ob zum gegebenen username ein Token in seinem Zwischenspeicher vorliegt und ob es mit dem übermittelten Token übereinstimmt. Ist das Token falsch oder nicht (mehr) vorhanden, wird eine HTTP 400 Fehlermeldung erzeugt ("Invalid or expired token."). Dies deckt sowohl falsch manipulierte URLs als auch abgelaufene Tokens ab – letzteres, weil der Service das Token nach Gebrauch löscht oder nach einem Neustart vergisst (siehe weiter unten). Wenn das Token stimmt, wird mittels Datenbank-Update das Benutzerkonto aktiviert (UPDATE sensora.users SET active = TRUE WHERE username = ...). Danach entfernt der Service den genutzten Token aus seinem tokens-Dictionary (damit der Link nicht erneut verwendet werden kann, One-Time Use). Schließlich liefert der Endpoint eine einfache HTML-Antwort zurück, die dem Nutzer bestätigt, dass die E-Mail erfolgreich verifiziert wurde (im Code: Rückgabe eines kleinen <h1>-HTML mit Erfolgstext). Dieses HTML wird durch FastAPI mithilfe der HTMLResponse direkt ausgegeben – so sieht der Nutzer unmittelbar im Browser eine Bestätigung.
\end{enumerate}
Der Mail-Service arbeitet ereignisgetrieben: Nur bei Bedarf wird eine Mail erzeugt, es gibt keinen dauerhaften Hintergrundprozess außer der DB-Verbindung. Durch FastAPI’s asyncio-basierte Architektur \cite{fastAPI}kann der Service viele Anfragen gleichzeitig abwickeln, ohne dass der Versand einer Mail (der einige Sekunden dauern kann) den gesamten Server blockiert. In unserem Fall wird zwar smtplib (synchron) genutzt – was den Event Loop blockiert – doch da der zu erwartende Aufrufdurchsatz gering ist (E-Mails nur bei Registrierung, nicht ständig), wurde auf komplexere nebenläufige Auslagerung verzichtet.

\section{Database Writer: MQTT-Datenpersistierung in PostgreSQL}
Der Database Writer Service ist ein Hintergrunddienst, der eingehende Sensordaten von den Geräten entgegennimmt und diese zuverlässig in der relationalen Datenbank speichert. Er bildet damit das Bindeglied zwischen der Echtzeit-MQTT-Datenebene und der persistenten Speicherung. Aus den Anforderungen geht hervor, dass Messwerte nicht verloren gehen sollen und zeitlich historisiert abrufbar sein müssen. Daher wurde eine Lösung implementiert, die auf nachrichtenbasierten Warteschlangen und garantierter Zustellung basiert. Der Database Writer subscribiert nicht einfach flüchtig auf MQTT-Themen, sondern nutzt den Solace-Broker mit einer persistenten Queue, um eine ausfallsichere Verarbeitung zu gewährleisten.
\subsection{Architektur: Dauerhafter Queue-Consumer}
Im Gegensatz zu den zuvor beschriebenen Webservices läuft der Database Writer ohne HTTP-Schnittstelle – er startet bei Systembeginn und läuft kontinuierlich als Daemon. Implementiert wurde er in Python unter Verwendung der Solace-eigenen Python API (solace.messaging), welche eine JMS-ähnliche Schnittstelle bietet. Die Hauptkomponenten sind:
\begin{enumerate}
    \item Solace-Verbindung: Beim Start baut der Service zunächst eine Verbindung zum Solace PubSub+ Broker auf. Dafür werden die Verbindungsparameter (Host, VPN, Username, Passwort) aus Umgebungsvariablen gelesen. Im Docker-Setup zeigt z.B. SOLACE\_HOST auf den internen Broker (tcp://solace:55555 für non-SSL MQTT über das interne Solace-Protokoll). Der Code versucht bis zu 10 mal in einem Retry-Loop die Verbindung herzustellen, mit Wartezeit, da der Broker evtl. noch am Hochfahren ist. Dieser Mechanismus erhöht die Robustheit: sollte der Broker zum Zeitpunkt des Writer-Starts nicht bereit sein, gibt der Service nicht sofort auf, sondern wartet insgesamt bis zu ~50 Sekunden auf eine erfolgreiche Verbindung.
    \item Persistente Queue und Konsument: Nach Verbindungsaufbau erstellt der Service einen Consumer auf einer durablen Message-Queue namens sensor\_data. Diese Queue ist so konfiguriert, dass sie alle relevanten Sensor-MQTT-Nachrichten aufnimmt. Die Zuordnung erfolgt über Subscriptions, die der Queue im Broker zugewiesen sind (dazu später mehr im Solace-Init Teil). Damit fungiert die Queue als Pufferspeicher: eintreffende MQTT-Publishs der Geräte werden vom Broker auf dieser Warteschlange zwischengespeichert, bis der Database Writer sie abholt. Die Verwendung einer persistenten Queue garantiert, dass keine Daten verlorengehen, selbst wenn der Consumer zwischenzeitlich ausfällt oder Netzwerkprobleme auftreten – der Broker hält die Nachrichten vor. Die Queue ist im Compose-Setup als exclusive deklariert, d.h. sie wird nur von einem Consumer genutzt, was sicherstellt, dass genau ein Service-Exemplar alle Daten chronologisch verarbeitet (kein Load Balancing hier gewünscht). Der Database Writer startet einen asynchronen Empfang auf dieser Queue mittels receiver.receive\_async(MessageHandler()). Hier wird ein benutzerdefinierter MessageHandler (eine Klasse, die eine on\_message-Methode überschreibt) verwendet, was dem Entwurf eines Event-Callbacks entspricht: Jede eingehende Nachricht triggert den Aufruf von SensorMessageHandler.on\_message.
    \item Verarbeitung eingehender Nachrichten: Im on\_message-Callback wird die erhaltene Nachricht zuerst vom proprietären Format in einen String dekodiert und dann als JSON geparst. Die erwartete Struktur der Nachrichten – dies wurde im theoretischen Teil des Datenformats definiert – beinhaltet in der obersten Ebene eine Controller-Kennung (did) und eine Liste von Sensor-Datensätzen (sensors). Jede Sensorstruktur enthält eine Sensor-ID (sid), einen Status (status) und ggf. einen Array von Messwerten (values). Der Database Writer iteriert über alle Sensoren in der Nachricht und führt für jeden folgende Schritte aus:
    \begin{enumerate}
        \item Status-Update (Heartbeat): Unabhängig davon, ob Messwerte vorliegen, wird die Information genutzt, dass ein Sensor Daten gesendet hat. Über die Hilfsfunktion update\_last\_call(sensor\_id, status) wird in der Datenbank der letzte Meldungszeitpunkt (last\_call Timestamp) und der Status des Sensors aktualisiert. Dies dient dazu, die Erreichbarkeit bzw. Aktivität von Sensoren nachzuverfolgen. Im Code wird hierbei ein frischer DB-Verbindungszyklus genutzt: update\_last\_call öffnet eine DB-Verbindung, führt ein UPDATE sensora.sensors SET last\_call = NOW(), status = \%s WHERE sid = \%s, und schließt die Verbindung wieder. Der Status wird auf den vom Gerät gemeldeten Wert gesetzt (typischerweise "active" bei normaler Meldung). Damit implementiert der Service ein Heartbeat-Monitoring: jedes Gerät signalisiert durch Senden (selbst von Messwerten) seine Aktivität.
        \item Messwertspeicherung: Falls der Sensor Messwerte im JSON mitgeliefert hat (values-Array nicht leer), werden diese in der Datenbank persistiert. Hierzu ruft der Handler die Funktion save\_sensor\_data(sensor\_id, values, controller\_id) auf. Innerhalb dieser Routine findet eine detaillierte Behandlung statt:
        \begin{enumerate}
            \item Zunächst wird sichergestellt, dass der referenzierte Controller existiert (Datenintegrität). Dazu wird in der Tabelle sensora.controllers per SELECT geprüft, ob did = controller\_id vorhanden ist. Ist dies nicht der Fall, wird ein Warnhinweis geloggt und die Speicherung für diesen Sensor abgebrochen – das System ignoriert also Messdaten von unbekannten Geräten. Im Normalfall sollten alle Controller aus dem Auth-Service bekannt sein.
            \item Als nächstes wird geprüft, ob der spezifische Sensor bereits in der Datenbank angelegt ist. Die Sensoren sind in der Tabelle sensora.sensors modelliert, mit Primärschlüssel sid. Falls das SELECT ergibt, dass dieser Sensor noch nicht existiert, interpretiert der Service dies als erstmalige Meldung eines neuen Sensors an diesem Controller. In unserem Systemdesign könnten Sensoren dynamisch erkannt werden (z.B. wenn ein Controller ein neues Sensormodul bekommt). In so einem Fall legt der Database Writer automatisch einen neuen Sensor-Datensatz in der DB an. Hierfür entnimmt er der Nachricht, falls vorhanden, Meta-Informationen über den Sensor (im JSON ggf. enthalten unter "sensor\_info"). Im Code wird die erste Value-Nachricht auf sensor\_info geprüft und daraus z.B. der Sensortyp (ilk, z.B. "humidity" oder "temperature") und Einheit (unit, z.B. "\%", "°C") extrahiert. Diese werden zusammen mit der Sensor-ID und der Controller-ID in sensora.sensors eingefügt. Dadurch wird der Sensor dem System bekannt gemacht. Wichtig: Beim Insert wird plant = NULL gesetzt, da initial der Sensor noch keiner Pflanze zugeordnet ist. Nach diesem Insert wird sofort ein commit durchgeführt, damit der neue Sensor auch in weiteren Schritten verfügbar ist. Zudem loggt das System die Anlage des Sensors.
            \item Zuordnungsprüfung: Ein kritischer Aspekt ist, dass Messwerte nur gespeichert werden sollen, wenn klar ist, welcher Pflanze sie zugeordnet sind. Im Datenmodell hat jeder Sensor optional einen Fremdschlüssel auf sensora.plants. Direkt nach dem Insert (oder wenn Sensor schon existierte) wird daher plant\_id aus dem Sensor-Datensatz ausgelesen. Ist plant\_id NULL (Sensor keiner Pflanze zugeordnet), bricht die Funktion ab ohne die Messwerte zu speichern. Dieser Schritt stellt sicher, dass Daten erst dann persistiert werden, wenn die organisatorische Verknüpfung hergestellt wurde –  um zu vermeiden, dass "verwaiste" Messwerte in der Datenbank landen, die keiner Pflanze zugeordnet sind. In der Praxis würde ein Nutzer in der Applikation also zunächst einen Sensor einer Pflanze (Topf) zuweisen, bevor Werte fließen. Nicht zugeordnete Sensoren melden zwar ihren Status (wodurch last\_call aktualisiert wird), aber ihre Werte werden bis zur Zuweisung verworfen (im Code durch Log "Sensor ist keiner Pflanze zugeordnet. Werte werden nicht gespeichert." gekennzeichnet).
            
            \item Werte-Insert: Falls ein plant\_id vorhanden ist, iteriert der Service über alle übermittelten Messwerte im values Array. Jeder Eintrag enthält typischerweise einen Zeitstempel (timestamp) und einen numerischen Wert (value). Wenn kein Timestamp angegeben ist, wird im Code "CURRENT\_TIMESTAMP" als Platzhalter genutzt, was die DB veranlasst, den Einfügezeitpunkt zu nehmen. Für jeden Wert generiert der Service eine eindeutige ID (vid via UUID4) und führt ein INSERT in die Tabelle sensora.values aus. Dabei werden Wert, Timestamp, Sensor-ID und Plant-ID gespeichert. Die Verwendung einer eigenen UUID für jeden Messwert garantiert, dass Einträge eindeutig sind; alternativ hätte man ein Serien-ID der DB nutzen können – hier zeigte sich aber der Designwunsch nach verteilbar eindeutigen IDs (was in IoT-Systemen mit mehreren Quellen sinnvoll sein kann). Nach dem Schleifendurchlauf über alle Werte werden die Insertionen per conn.commit() in der DB finalisiert. Abschließend wird die DB-Verbindung geschlossen. Im Log erscheint dann pro Sensor eine Bestätigung ("Alle Werte für Sensor X gespeichert.").
            \item Fehler während der DB-Operationen werden aufgefangen und als Fehler geloggt, ohne dass der gesamte Service abstürzt. Sollte z.B. während der Inserts ein DB-Fehler auftreten, würde zwar dieser Aufruf fehlschlagen, aber der Message Handler an sich fängt die Exception und beendet nicht den Prozess (siehe nächster Punkt).
        \end{enumerate}
        \item Message Acknowledgement: Nachdem alle Sensoren einer Nachricht verarbeitet wurden, bestätigt der Database Writer dem Broker den erfolgreichen Empfang mittels receiver.ack(message). Dies ist ein wichtiger Schritt im Zusammenspiel mit der persistenten Queue: Erst durch das Acknowledge wird die Nachricht aus der Queue entfernt. Sollte der Service abstürzen oder es käme zu einem nicht behandelten Fehler vor dem Ack, würde die Nachricht in der Queue bleiben und später erneut zugestellt werden können (garantierte mindestens-einmal-Zustellung). Im implementierten Code wird das Ack nur im erfolgreichen JSON-Verarbeitungsfall aufgerufen. Bei bestimmten Fehlern, z.B. JSON-Parsing-Error oder falls essentielle Felder fehlen, wird kein Ack gesendet, was bedeutet, dass die Nachricht in der Queue verbleibt. Im Log wird ein Hinweis ausgegeben ("Nachricht ist kein gültiges JSON" oder "Ungültige Nachricht. Controller-ID fehlt."), aber ein Ack fehlt. Dieses Verhalten könnte zu einer erneuten Zustellung führen (je nach Broker-Einstellung) oder die Nachricht blockiert die Queue. Im aktuellen Setup von Solace würde eine nicht bestätigte Nachricht in einer durable Queue verbleiben; der Consumer könnte z.B. nach einem Timeout neu gestartet werden, um es erneut zu versuchen. Hier wäre eventuell eine Verbesserung, solche Nachrichten nach x Versuchen in eine Dead Message Queue zu verschieben – jedoch ist dies im Code nicht implementiert. Somit wird sich darauf verlassen, dass gut formatierte Nachrichten ankommen. Die bewusste Entscheidung, bei unverarbeitbaren Nachrichten kein Ack zu senden, spiegelt einen Anspruch auf Datenintegrität: Lieber bleibt eine fehlerhafte Nachricht liegen (und ein Admin greift ein), als dass sie fälschlich als verarbeitet markiert wird.

    \end{enumerate}
\end{enumerate}
Zusätzlich zur Callback-Verarbeitung hat der Database Writer einen Nebenprozess zur Überwachung der Sensor-Aktivität:
Timeout-Überprüfung: Mithilfe eines einfachen Endlosschleife-Timers im Hauptthread wird in regelmäßigen Abständen (jede Minute, gemäß CHECK\_INTERVAL = 60 Sekunden) die Funktion check\_sensor\_timeouts() ausgeführt. Diese öffnet eine DB-Verbindung und führt ein Update auf der Sensors-Tabelle aus, um alle Sensoren, deren last\_call älter als 5 Minuten ist, auf Status 'error' zu setzen. Damit wird ein Timeout-Mechanismus realisiert: Wenn ein Sensor 5 Minuten lang keine Daten gesendet hat (und vorher auf 'active' stand), gilt er als potenziell offline oder ausgefallen. Im Datenbankmodell wird dies durch status = 'error' kenntlich gemacht. Dieser Mechanismus ergänzt das oben erwähnte Heartbeat-Tracking. Der Zähler wird nach jedem Lauf zurückgesetzt. Durch die Schleife mit time.sleep(1) wird der CPU-Verbrauch minimal gehalten.

\section{Setpoint API: Sollwert-Vorgabe via REST und MQTT}
Die Setpoint API ermöglicht es, aus dem System heraus Steuerungswerte an die Mikrocontroller zu senden – konkret Sollwerte für bestimmte Sensoren (z.B. Feuchtigkeits-Sollwert für die Bewässerungssteuerung). Damit wird das System bidirektional: Nicht nur melden Sensoren Zustände, sondern Aktoren können angesteuert werden. Der Dienst ist als kleiner Flask-basierter Webservice realisiert, der eine REST-Endpoint bereitstellt, über den ein Sollwert gesetzt werden kann. Intern publiziert der Service diesen Sollwert dann als MQTT-Nachricht auf den Broker, sodass der Ziel-Controller ihn empfängt. Im Grunde handelt es sich also um einen Protokollübergang von HTTP zu MQTT.
\subsection{Funktionsweise und Ablauf}
Der Setpoint-Service stellt den Endpoint /sollwert (HTTP POST) bereit. Der typische Ablauf, um einen Sollwert zu setzen, ist:
\begin{enumerate}
    \item Eine externe Entität – z.B. eine Web-Frontend-Anwendung oder ein Benutzer via App – sendet einen HTTP-POST an die Setpoint API mit den Parametern:
    \begin{enumerate}
        \item controller\_id: die Kennung des Ziel-Controllers (Gerät),
        \item sensor\_id: die Kennung des Sensors (bzw. des Aktors) auf diesem Gerät, für den der Sollwert gelten soll,
        \item sollwert: der anzustrebende Wert (numerisch, z.B. Feuchtigkeit in \% oder ein Schwellwert).
    \end{enumerate}
    \item Die API prüft eingehend, ob alle nötigen Felder vorhanden und gültig sind. Falls etwas fehlt, wird mit HTTP 400 Bad Request geantwortet und ein Fehlerjson zurückgegeben.
    \item Ist die Eingabe valide, generiert der Service eine MQTT-Nachricht. Dazu wird ein JSON-Objekt erstellt, das wie folgt aussieht:
    \begin{lstlisting}
    {
        "targetValues": [
            {
                "did": "<controller_id>",
                "sid": "<sensor_id>",
                "value": <sollwert>
            }
        ]
    }
    \end{lstlisting}
    Diese Struktur ist angelehnt an das Format, das ggf. vom Gerät erwartet wird (eine Liste von Zielwerten für bestimmte Sensoren/Aktoren). Es wird also der Device-ID und Sensor-ID nochmals eingebettet, damit das Gerät die Nachricht zuordnen kann.
    \item Als nächstes bestimmt der Service das Ziel-Topic für die MQTT-Nachricht. Gemäß der in Auth-Service eingerichteten Konvention wird das Topic sensora/v1/receive/<controller\_id>/targetValues genutzt. Darauf ist der betreffende Controller (bzw. dessen MQTT-Client) berechtigt zu lauschen. Dieses Topic adressiert somit genau den gewünschten Controller.
    \item Der Service publiziert die Nachricht über den Solace-Broker: Hierzu nutzt er die vorab aufgebaute Broker-Verbindung und einen Publisher, der als persistent message publisher initialisiert wurde. Die Nachricht wird mit dem oben genannten Topic abgesendet. Der Broker sorgt dann dafür, dass – sofern der Controller online ist und das Topic abonniert hat – die Nachricht an diesen zugestellt wird. Sollte der Controller momentan nicht verbunden sein, greift je nach Broker-Einstellung die Persistierung: Entweder wurde eine Queue für solche Sollwert-Nachrichten eingerichtet (vgl. mögliches sensor\_setpoints Queue), oder im MQTT-Kontext übernimmt der Broker das Speichern bei QoS>0. Da der Publisher hier als "persistent" konfiguriert ist, lässt sich ableiten, dass die Nachricht als durable versendet wird, was in MQTT-Terminologie etwa QoS 1 entspricht (mindestens einmal zustellen). Solace bietet für MQTT-Clients mit dauerhafter Session auch an, solche Nachrichten zwischenzuspeichern, was vermutlich hier genutzt wird.
    \item Die Setpoint API gibt dem HTTP-Aufrufer eine Erfolgsmeldung zurück (HTTP 200, JSON {"status": "success"}). Damit ist der Vorgang für den Benutzer abgeschlossen. Im Hintergrund allerdings wird nun der Controller die Nachricht empfangen und z.B. seine Konfiguration anpassen (dies ist Teil der Geräte-Firmware und außerhalb dieses Service-Scopes, aber essentiell für den Regelkreis der Bewässerung).
\end{enumerate}

\section{Solace Init: Automatisierte Broker-Konfiguration}
Der Solace Init Service (bzw. Skript) wurde implementiert, um bei Start des Gesamtsystems sicherzustellen, dass der Solace-Broker über alle notwendigen Persistent Queues und Topic-Weiterleitungen verfügt. Er stellt somit eine Infrastruktur-Komponente dar, die eng mit dem Broker zusammenarbeitet. Da in einem Container-Setup der Broker beim ersten Start völlig jungfräulich ist, muss z.B. die sensor\_data Queue angelegt und mit dem entsprechenden Topic verbunden werden. Solace Init erfüllt genau diese Aufgabe über die SEMP v2 Management-API.
\subsection{Vorgehen und Konfiguration}
Solace Init ist als eigenständiges Python-Skript (init.py) konzipiert, das beim Hochfahren des Docker-Compose Stacks einmalig ausgeführt wird und sich danach beendet. Seine Aufgaben sind:
\begin{enumerate}
    \item Einlesen der gewünschten Queue-Konfiguration: Aus einer JSON-Datei (im Code queues.json) werden alle zu erstellenden Queues und ihre Eigenschaften/Subscriptions geladen. Dieses File definiert quasi deklarativ, welche Warteschlangen mit welchen Parametern und Abos existieren sollen. Ein beispielhafter Inhalt könnte so aussehen:
    \begin{lstlisting}
        
    {
      "queues": [
        {
          "queueName": "sensor_data",
          "accessType": "exclusive",
          "egressEnabled": true,
          "ingressEnabled": true,
          "subscriptions": [
            { "subscriptionTopic": "sensora/v1/send/>" }
          ]
        },
        {
          "queueName": "sensor_setpoints",
          "accessType": "exclusive",
          "subscriptions": [
            { "subscriptionTopic": "sensora/v1/receive/>" }
          ]
        }
      ]
    }
    \end{lstlisting}
    Hier würde z.B. festgelegt, dass es eine Queue sensor\_data gibt, die exklusiv ist und sowohl Ingress/Egress aktiviert hat (Standard für persistent Queues), und dass sie alle Topics unter sensora/v1/send abonniert (der > Wildcard deckt alle Unterpfade ab). Ebenso eine Queue sensor\_setpoints für alle sensora/v1/receive Nachrichten. Dieses Konstrukt deckt die zuvor diskutierten Pfade ab: Messwerte und Steuerbefehle.
    \item Verbindung zur SEMP API: Solace Init nutzt Python requests, um HTTP-POSTs an den Broker zu senden. Die notwendigen Zugangsdaten (Admin-User, Passwort, Host:Port) werden aus Env-Variablen bezogen (SOLACE\_USER, SOLACE\_PASS, SOLACE\_HOST\_SEMP). Im Docker Compose sieht man, dass solace-init Container mit depends\_on: solace gestartet wird und erst nach ~80s (wenn Broker läuft) das Skript ausführt. So ist sichergestellt, dass der Broker Management-Port 8080 erreichbar ist. SEMP v2 erfordert Basic-Auth mit dem Admin-Login\cite{solaceSEMP}, was hier als Tuple im requests.post(..., auth=(user, pass)) genutzt wird.
    \item Erstellen von Queues (SEMP /config): Für jede in der JSON gelistete Queue baut das Skript die entsprechende URL und das Payload zusammen. Beispiel: POST http://solace:8080/SEMP/v2/config/msgVpns/default/queues mit JSON 
    \begin{lstlisting}
        {"queueName": "sensor_data", "accessType": "exclusive", "egressEnabled": true, ...}
    \end{lstlisting}.
    Der Broker antwortet mit Status 201 (Created) bei Erfolg. Das Skript prüft den Statuscode:
    \begin{enumerate}
        \item 200/201 wird als Erfolg gewertet und entsprechend geloggt ("Queue erfolgreich erstellt").
        \item 409 (Conflict) bedeutet, die Queue existiert bereits – in diesem Fall loggt das Skript eine Warnung, fährt aber fort. Das ist gewollt, um Idempotenz zu erreichen: Sollte man Solace Init versehentlich zweimal ausführen oder den Broker bereits vorkonfiguriert haben, bricht es nicht ab, sondern erkennt vorhandene Entities.
        \item Andere Fehlercode führt zu einem Fehlerausdruck und Rückkehr aus der Funktion (somit würde die Subscription-Anlage für diese Queue übersprungen).
    \end{enumerate}
    \item Hinzufügen von Topic-Subscriptions: Nach dem Anlegen (oder Erkennen) der Queue iteriert das Skript über alle vorgesehenen Subscription-Themen und sendet für jedes ein POST .../queues/{queueName}/subscriptions mit {"subscriptionTopic": "<topic>"}. Hier gelten ähnliche Statuscodes:
    \begin{enumerate}
        \item 200/201: Subscription hinzugefügt (Logausgabe),
        \item 409: bereits vorhanden (Warnhinweis, aber kein Abbruch),
        \item andere: Fehler ausgeben.
    \end{enumerate}
    \item Abschluss: Nachdem alle Queues aus der Liste abgearbeitet sind, gibt das Skript eine Meldung aus, dass alle Queues und Subscriptions verarbeitet wurden, und endet.
\end{enumerate}

	\newpage
	
	% ------------- Ende Hautpteil -------------
	
	\chapter{Kritische Reflexion}
	%%Reflektiere über umsetzung der Komponenten, was hat gut funktioniert, was nicht, was war einfach, was war schwer, was war unerwartet, was war nicht so gut.
\section{IoT Device}
Das IoT Device wurde von einer Person als Hauptentwickler und mehreren Unterstützenden Entwicklern erstellt.
    \subsection{Hardware}:
    Geplant war das IoT Device als ein Gerät, das die Luftfeuchtigkeit und Temperatur misst und diese Daten an einen Server sendet.
    Für die Hardware wurde ein fertiger Bausatz von Alibaba gekauft, der alle benötigten Komponenten enthielt. Damit war es möglich, die Hardware schnell zusammenzubauen und zu testen.
    Die Sensoren sind nur mittelmäßig genau, was aber für ein Proof of Concept ausreichend ist.
    \subsection{Software}:
    Die Software wurde in C entwickelt. Dabei wurde CLion mit CMake als IDE verwendet.
    Dadurch kam es zu komplikationen im Setup, da die IDE nicht richtig konfiguriert war. Daraus resultierten verzögerungen die das Projekt belasteten
    Mehr feste Entwickler hätten hier helfen können.

Insgesamt lief die Entwicklung des IoT Devices gut, aber es gab einige Herausforderungen, die bewältigt werden mussten.
    Die Hardware war einfach zusammenzubauen, aber die Software war schwieriger zu implementieren als erwartet.
    Es gab einige Probleme mit der Kommunikation zwischen dem IoT Device und dem Server, die behoben werden mussten.
    Auch die Genauigkeit der Sensoren war nicht so hoch wie erhofft, was die Ergebnisse beeinflusste.
    Dennoch konnte das IoT Device erfolgreich entwickelt und getestet werden, und es erfüllt die Anforderungen des Projekts.
    
	\newpage 
	\chapter{Ausblick}
	%% Hier bisschen Tagträumen was in Zukunft noch so kommen könnte.
\section{IoT Device}
Das IoT Device könnte in Zukunft durch bessere Sensoren und eine bessere Software weiter verbessert werden.
Mehr Modularität könnte auch helfen, das IoT Device einfacher zu erweitern und anzupassen.
Das IoT Device könnte auch in Zukunft durch eine bessere Benutzeroberfläche und eine bessere Integration in andere Systeme weiter verbessert werden.
	\newpage
	
	%------------- Literaturverzeichnis -------------
	\newpage
	\pagestyle{fancy}
	\fancyhead[R]{LITERATUR}

	\chapter*{Literaturverzeichnis}
	\addcontentsline{toc}{chapter}{Literaturverzeichnis}
	\printbibliography
	\newpage

	%------------- Anhang -------------
	\cleardoublepage
	\fancyhead[R]{ANHANG}

	\chapter*{Anhang}
	\addcontentsline{toc}{chapter}{Anhang}

	% Hier Anhänge einfügen:
	%\input{./Anhang/sensorwerte}
	%\input{./Anhang/schaltplan}

\newpage

	
\end{document}
