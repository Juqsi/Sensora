\section{Einsatz von Tailwind CSS und Shadcn-Vue}

Moderne Webentwicklung nutzt zunehmend Utility-First-CSS-Frameworks wie Tailwind CSS und komponentenbasierte UI-Bibliotheken wie Shadcn-Vue zur effizienten Erstellung ansprechender Benutzeroberflächen. Diese Werkzeuge bieten im Kontext von Vue.js signifikante Vorteile hinsichtlich Geschwindigkeit, Wartbarkeit und Designkonsistenz \cite{TailwindCSS}, \cite{ShadcnVue2023}.

\subsection{Tailwind CSS: Utility-First-Ansatz}
Tailwind CSS ist ein Utility-First-CSS-Framework, das es Entwicklern erlaubt, direkt im HTML Code Styling zu betreiben, anstatt klassische CSS-Klassen zu definieren. Dies erleichtert die Wiederverwendbarkeit und reduziert übermäßige Stylesheet-Komplexität \cite{TailwindCSS}.

\paragraph{Vorteile:}
\begin{itemize}
	\item \textbf{Performance:} Tailwind eliminiert ungenutztes CSS im Produktionsbuild.
	\item \textbf{Responsives Design:} Vordefinierte Breakpoints erleichtern die Anpassung an verschiedene Bildschirmgrößen.
	\item \textbf{Designkonsistenz:} Einheitliche Farben, Abstände und Größen durch Konfiguration.
\end{itemize}

\paragraph{Integration in Vue.js}
Durch vordefinierte Klassen lässt sich Tailwind optimal in Vue-Komponenten einbetten. Der Code bleibt deklarativ und wartbar. Mit Tailwind können komplexe Layouts wie Grid-basierte Dashboards oder mobile Navigationsleisten ohne zusätzliche CSS-Dateien umgesetzt werden.

\subsection{Shadcn-Vue: Komponentenbibliothek f\"ur Vue 3}
Shadcn-Vue ist ein Community-getriebener Port der beliebten React-Bibliothek Shadcn/UI f\"ur Vue 3. Sie basiert auf Radix UI und Tailwind CSS, womit sie eine hohe Anpassbarkeit bei gleichzeitiger Konsistenz gew\"ahrleistet \cite{ShadcnVue2023}.

\paragraph{Modularer Aufbau}
Im Gegensatz zu anderen UI-Bibliotheken installiert Shadcn-Vue Komponenten selektiv. Dies f\"uhrt zu einer schlankeren Anwendung und erh\"oht die Kontrolle \"uber das Styling.

\paragraph{Beispiele aus dem Projekt}
Die Anwendung verwendet u.a. folgende Sha\-dcn-Kom\-po\-nen\-ten:
\begin{itemize}
	\item \texttt{Tabs}, \texttt{Cards}, \texttt{DropdownMenus} in der Pflanzen- und Sensordarstellung
	\item \texttt{Switches} und \texttt{Tooltips} in der Detailansicht und den Einstellungen
	\item \texttt{AlertDialogs} f\"ur Benutzerbest\"atigungsprozesse
\end{itemize}

\subsection{Kombination beider Technologien im Frontend-Projekt}
Tailwind CSS dient als stilistisches Fundament, w\"ahrend Shadcn-Vue darauf aufbaut und vorgefertigte interaktive Komponenten zur Verf\"ugung stellt. Das Ergebnis ist eine moderne, performante und barrierearme Benutzeroberfl\"ache, die sowohl f\"ur Desktop als auch f\"ur mobile Endger\"ate optimiert ist.

\paragraph{Fazit}
Die Kombination von Tailwind CSS und Shadcn-Vue stellt eine moderne und leistungsf\"ahige L\"osung f\"ur das Design von Vue.js-basierten Single-Page-Anwendungen dar. Sie reduziert Entwicklungsaufwand, erh\"oht die Designkonsistenz und verbessert die Nutzererfahrung signifikant \cite{Guimaraes2021}, \cite{TailwindCSS}.
