\section{State-Management mit Pinia und Persistenz}
\label{chap:pinia}

Ein zentraler Bestandteil moderner Frontend-Anwendungen ist das effiziente und wartbare Zustandsmanagement. Im Kontext der Webanwendung zur Steuerung eines smarten Bewässerungssystems kommt die State-Management-Bibliothek \texttt{Pinia} zum Einsatz. Dieser Abschnitt behandelt die Motivation, Konzeption und Persistierung des globalen Applikationszustands mittels Pinia in Vue.js.

\subsection{Einordnung von State Management in SPAs}

In Single Page Applications liegt die Verantwortung für das Daten- und UI-Management vollständig im Frontend. Daraus ergeben sich Anforderungen wie die zentrale Verwaltung globaler Zustände etwa für eingeloggte Benutzer oder verbundene Geräte, die Wiederverwendung von Daten über Komponenten hinweg, die Synchronisierung mit Backend-Endpunkten sowie die Notwendigkeit einer dauerhaften Speicherung über Seitenreloads hinaus. Pinia, als moderne und offizielle Ablösung von Vuex in der Vue-3-Welt, erfüllt diese Anforderungen durch seine modulare, typesichere und reaktive Struktur \cite{Vuex} \cite{Allotey2023}.



\subsection{Persistenz mit \texttt{pinia-plugin-persistedstate}}

Ein zentrales Merkmal der Anwendung ist die Fähigkeit, den Zustand auch bei einem Seitenreload zu bewahren. Dies wird mithilfe des Plugins \texttt{pinia-plugin-persistedstate} erreicht, das es ermöglicht, ausgewählte Stores automatisch im \texttt{localStorage} des Browsers zu speichern \cite{VueMastery2023}. Jeder Store konfiguriert explizit, ob und wie persistiert wird.



\subsection{Bewertung und Grenzen}

Im praktischen Einsatz zeigt Pinia deutliche Vorteile hinsichtlich Strukturierung und Wartbarkeit des Zustands. Die modulare Trennung erlaubt eine gute Übersicht und fördert die Wiederverwendbarkeit einzelner Logikkomponenten. Durch TypeScript wird zudem eine hohe Typsicherheit erzielt, was insbesondere in größeren Projekten die Fehleranfälligkeit reduziert. Auch das Debugging gestaltet sich dank der Integration in die Vue DevTools effizient. Die Persistenz erlaubt Offline-Szenarien und schützt vor ungewolltem Datenverlust. Eine Herausforderung besteht in der Handhabung verschachtelter Ressourcenbeziehungen. So referenziert eine Pflanze beispielsweise ihren Raum, dieser wiederum eine Gruppe. Dieses potenzielle Problem wird in der Anwendung durch gezielte Backendabfragen und das gezielte Laden zusammengehöriger Ressourcenstrukturen bei Bedarf entschärft.

\subsection*{Fazit}

Pinia mit Persistenz stellt ein robustes und wartbares State-Management für Vue-Frontends bereit. Es erfüllt die Anforderungen an Konsistenz, Wiederverwendbarkeit, Performanz und Sicherheit und bildet damit die solide Grundlage für ein reaktives und benutzerfreundliches Smart-Gardening-System.

