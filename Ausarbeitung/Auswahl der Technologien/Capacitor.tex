\section{Mobile Kompilierung mit Capacitor}
\label{chap:capacitor}

Ein zentrales Ziel der entwickelten Anwendung ist ihre Nutzbarkeit nicht nur als Web-App, sondern auch als mobile Applikation auf Android-Geräten. Dies wird durch die Integration von Capacitor ermöglicht.

\subsection{Funktionsweise von Capacitor}
Capacitor fungiert als moderne Brückentechnologie zwischen Web-Technologien und nativer Funktionalität. Die Vue-Anwendung wird dabei in eine WebView eingebettet und über ein Plugin-System mit nativen Funktionen verbunden, was eine hybride Nutzung nativer Hardware-Ressourcen und Web-Technologien erlaubt \cite{Singh2021, Giordano2024}. Der Build-Prozess beginnt mit dem Erzeugen des Produktions-Builds der Vue-App mittels \texttt{vite build}. Anschließend wird mit dem Befehl \texttt{npx cap add android} ein natives Android-Projekt erstellt. Die WebAssets der Anwendung werden mit \texttt{npx cap sync} in das Android-Projektverzeichnis \texttt{android/app/src/main/assets} überführt. Danach kann die App entweder direkt in Android Studio geöffnet oder mit der Kommandozeile gestartet werden.

\subsection{Nutzung nativer APIs}
Capacitor stellt eine Reihe nativer APIs zur Verfügung, die aus der Vue-Anwendung heraus direkt verwendet werden können. Dazu zählen unter anderem der Zugriff auf die Kamera, auf das Dateisystem sowie UI-Funktionen wie StatusBar, SplashScreen und Toast-Benachrichtigungen \cite{Singh2021}. Diese APIs werden durch offizielle \texttt{@capacitor}-Plugins bereitgestellt und erlauben den Zugriff auf Systemfunktionen, ohne dass plattformspezifischer Code geschrieben werden muss.

\subsection{Besonderheiten und Herausforderungen}
Bei der Entwicklung für mobile Geräte ergeben sich mehrere technische Herausforderungen. So muss der Vue Router im sogenannten "History-Modus" betrieben werden, da es sonst zu Problemen mit Deep-Links unter Android kommen kann \cite{CapacitorDocs}. Bei falscher Konfiguration können Routen nicht korrekt aufgelöst werden, was 404-Fehler zur Folge hat. Zudem führt die Nutzung der Bildschirmtastatur auf Mobilgeräten dazu, dass Eingabefelder teilweise verdeckt werden. Dieses Verhalten muss durch gezielte Event-Behandlung oder gestalterische Workarounds ausgeglichen werden. Auch der Zugriff auf Systemfunktionen wie Kamera oder Medien setzt unter neueren Android-Versionen explizite Berechtigungsabfragen voraus, die sorgfältig umgesetzt werden müssen \cite{Singh2021}.

\subsection{Vor- und Nachteile der Verwendung von Capacitor}
Der Einsatz von Capacitor bringt sowohl Vorteile als auch Einschränkungen mit sich. Ein wesentlicher Vorteil besteht darin, dass die Anwendung weiterhin auf Web-Technologien basiert und somit eine einheitliche Codebasis für Web- und Mobile-Plattformen verwendet werden kann. Zudem unterstützt Capacitor Hot Reloading, was die Entwicklung und das Debugging deutlich beschleunigt \cite{Singh2021}. Die Anwendung kann darüber hinaus nicht nur als native App, sondern auch als Progressive Web App (PWA) bereitgestellt werden. Auf der anderen Seite ergeben sich durch die Nutzung der WebView geringere Performancewerte im Vergleich zu vollständig nativen Anwendungen \cite{Giordano2024}. Auch ist der Zugriff auf bestimmte Systemfunktionen eingeschränkt, und die Pflege des nativen Projekts setzt Kenntnisse in Android Studio und dem Android-Entwicklungsprozess voraus.

\subsection{Bewertung für das Projekt}
Für die im Rahmen dieser Arbeit entwickelte Anwendung stellt Capacitor eine sinnvolle und praktikable Lösung dar. Da die gesamte Logik auf Web-Komponenten basiert und der Bedarf an nativer Funktionalität auf wenige Aspekte wie Kameranutzung beschränkt ist, bietet Capacitor eine effiziente Möglichkeit, eine mobile Version mit minimalem Mehraufwand bereitzustellen. Die Integration in den Entwicklungsworkflow verläuft nahtlos, wodurch die mobile Erweiterung der smarten Bewässerungssteuerung sowohl benutzerfreundlich als auch wartbar bleibt.


