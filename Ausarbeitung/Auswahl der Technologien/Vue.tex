\section{Auswahl von Vue.js für die Implementierung}
\label{sec:auswahl-vue}

Im Rahmen der Entwicklung eines webbasierten Frontends für ein intelligentes Bewässerungssystem fiel die Wahl auf Vue.js. Die Entscheidung begründet sich auf mehreren Faktoren:

\subsection{Modularität und Komponentenstruktur}
Vue ermöglicht eine klare Trennung von Funktionalität, Darstellung und Stil durch das Single-File-Component-Modell. Dies unterstützt die Wiederverwendbarkeit und die Wartbarkeit in mittelgroßen Anwendungen wie der vorliegenden \cite{VueGuide2024}. Komponenten lassen sich dabei hierarchisch strukturieren, durch Props und Events miteinander verknüpfen und flexibel wiederverwenden. Die damit verbundene Modularität ist ein zentraler Vorteil im Vergleich zu klassischen Monolith-Strukturen.

\subsection{Reaktivität und Datenbindung}
Die Composition API in Vue 3 erlaubt die strukturierte Wiederverwendung von Logik und bietet eine feinere Kontrolle über Komponentenlebenszyklen. Die Reaktivierung ist deklarativ und effizient, was zu einer reduzierten Komplexität führt \cite{VueCompositionAPI2020}. Im Gegensatz zur eher komplexen Reaktivierung in Angular oder den teils manuell zu verwaltenden Hooks in React bietet Vue ein konsistentes Modell, das einfacher zu testen und zu debuggen ist \cite{VueReactivity2016}. Insbesondere die automatische DOM-Synchronisierung bei Zustandsänderungen verringert den Entwicklungsaufwand erheblich.

\subsection{Community, Dokumentation und Lernkurve}
Im Vergleich zu Angular bietet Vue eine flachere Lernkurve und ist dennoch umfangreicher als React in seiner Grundausstattung. Besonders für kleine bis mittelgroße Teams ohne dedizierte DevOps- oder Backend-Abteilungen eignet sich Vue durch seine einfache Integration und das konsistente "{O}kosystem \cite{Allotey2023}. Die offizielle Dokumentation von Vue gilt als eine der besten im Bereich der Webframeworks und trägt wesentlich zur schnellen Produktivität bei \cite{VueGuide2024}. Hinzu kommt ein aktives Community-Umfeld mit einer Vielzahl an Open-Source-Bibliotheken und Erweiterungen.

\subsection{Flexibilität, Integration und Zukunftssicherheit}
Ein weiterer Vorteil von Vue ist seine hohe Flexibilität im Hinblick auf Tooling und Integration. Vue-Projekte können leicht mit modernen Build-Tools wie Vite kombiniert werden, welches durch schnelle Entwicklungszyklen und modulare Hot-Reloading-Mechanismen eine effiziente Frontendentwicklung ermöglicht. Durch die strikte Trennung von View- und Logikschicht lässt sich Vue problemlos mit REST-APIs, GraphQL oder WebSockets kombinieren. Zudem wird Vue kontinuierlich weiterentwickelt: Die Long-Term-Support-Strategie sowie eine klare Roadmap sprechen für eine hohe technologische Zukunftssicherheit.

\subsection{Vergleich: Options API vs. Composition API in Vue.js}
Vue.js unterstützt zwei zentrale Paradigmen zur Strukturierung von Komponenten: die klassische Options API und die moderne Composition API. Beide Modelle ermöglichen die Definition von Zuständen, Methoden, Lebenszyklus-Hooks und Reaktivität innerhalb einer \ac{SFC}, unterscheiden sich jedoch fundamental im Aufbau und in der Ausdrucksstärke.

\paragraph{Options API}
Die Options API stellt das klassische und lange Zeit dominante Paradigma zur Definition von Komponenten in Vue.js dar. Ihr zentraler Vorteil liegt in der klar strukturierten Gliederung der Komponentenlogik nach spezifischen Optionen wie data, methods, computed, watch und Lebenszyklus-Hooks. Diese Trennung erleichtert insbesondere Einsteigerinnen und Einsteigern den Zugang zur komponentenbasierten Entwicklung, da die Zuständigkeiten der einzelnen Blöcke unmittelbar nachvollziehbar sind. Die durchgängige Unterstützung in der offiziellen Vue.js-Dokumentation sowie in zahlreichen Community-Plugins trägt zusätzlich zur Zugänglichkeit und zur breiten Akzeptanz dieses Modells bei. Auch historisch bedingt ist die Options API weiterhin vollständig kompatibel und wird aktiv gepflegt, was ihre Relevanz in bestehenden Projekten unterstreicht \cite{VueGuide2024}.

\newpage
\begin{lstlisting}[caption=Beispiel Options API]
export default {
	data() {
		return {
			counter: 0
		}
	},
	methods: {
		increment() {
			this.counter++
		}
	}
}
\end{lstlisting}

Den Vorteilen stehen jedoch mehrere signifikante Einschränkungen gegenüber. Insbesondere bei wachsender Komplexität einer Komponente stößt die Options API an strukturelle Grenzen. Da Zustände und zugehörige Methoden nach Typ gruppiert und nicht funktional zusammenhängend strukturiert werden, entsteht bei umfangreicher Logik schnell eine fragmentierte Darstellung. Diese Fragmentierung erschwert nicht nur die Lesbarkeit, sondern auch die Wartbarkeit und Wiederverwendbarkeit von Code – vor allem dann, wenn sich Logik über mehrere Komponenten hinweg wiederholt. Zudem leidet die Options API unter einer eingeschränkten Typsicherheit im Umgang mit TypeScript, da Kontextinformationen wie this nicht ohne Weiteres typensicher aufgelöst werden können. Dies kann zu Laufzeitfehlern führen und behindert die statische Analyse durch TypeScript-Compiler \cite{VueGuide2024}.

\paragraph{Composition API}
Die mit Vue 3 eingeführte Composition API bietet eine moderne und hochgradig modulare Alternative zur klassischen Options API. Sie zielt insbesondere auf eine bessere Wiederverwendbarkeit und thematische Gruppierung von Logik ab. Ein zentrales Merkmal ist, dass Zustände, Methoden und Nebenwirkungen innerhalb der \texttt{setup()}-Funktion definiert werden. Dadurch lassen sich zusammenhängende Funktionsblöcke logisch gruppieren und als sogenannte Composables wiederverwenden. Dies erhöht die Wartbarkeit bei wachsender Komponentenkomplexität erheblich \cite{CompositionAPIFAQ}.

Ein weiterer Fortschritt besteht in der Einführung des sogenannten \texttt{<script setup>}-Blocks, der in Vue 3 als syntaktischer Zucker (syntactic sugar) über der Composition API liegt. Im Gegensatz zur expliziten Verwendung von \texttt{setup()} in einem klassischen \texttt{<script>}-Block vereinfacht \texttt{<script setup>} die Struktur, reduziert Boilerplate-Code und macht die Komponenten deklarativer und kompakter. Dabei werden alle im \texttt{<script setup>} definierten Variablen automatisch im Template verfügbar gemacht, ohne dass ein \texttt{return} erforderlich ist \cite{ScriptSetup}.
\newpage
\begin{lstlisting}[caption=Beispiel für die Composition API mit \texttt{<script setup>},numbers=left,label=lst:scriptsetup,language=html]
	<script setup lang="ts"> 
		import { ref } from 'vue' 
		const counter = ref(0) 
		const increment = () => { counter.value++ } 
	</script>
	\end{lstlisting}

Hier wird ein reaktiver Zustand \texttt{counter} über die Funktion \texttt{ref()} erstellt, welcher automatisch mit dem DOM synchronisiert wird. Die Funktion \texttt{increment} verändert diesen Zustand und steht im Template zur Verfügung.

Die Vorteile dieses Ansatzes liegen auf der Hand: Logik ist gruppiert, leicht extrahierbar und testbar, insbesondere durch Composables. Darüber hinaus bietet die Composition API eine exzellente Typsicherheit, insbesondere im Zusammenspiel mit TypeScript \cite{VueCompositionAPI2020}.

Allerdings ergeben sich auch gewisse Herausforderungen. Für Neulinge kann der reduzierte strukturelle Rahmen der \texttt{setup()}-Funktion zunächst weniger Orientierung bieten als die Options API. Zudem besteht bei unstrukturierter Nutzung die Gefahr einer unübersichtlichen, flachen Anordnung vieler Logikelemente ohne klare Gruppierung, was die Lesbarkeit und Wartbarkeit negativ beeinflussen kann \cite{CompositionAPIFAQ}.

Während die Options API ihre Stärken in der Klarheit und dem geringeren Einstieg hat, bietet die Composition API insbesondere in Kombination mit \texttt{<script setup>}, eine moderne, typsichere und wiederverwendbare Struktur für Vue-Komponenten - insbesondere für mittlere bis große Anwendungen mit komplexer Zustandslogik \cite{CompositionAPIFAQ}.

\paragraph{Einordnung im Projektkontext}
In der vorliegenden Anwendung wurde bewusst die Composition API eingesetzt, da sie sich durch eine deklarative, modulare und testfreundliche Struktur auszeichnet. Besonders in Kombination mit Composables, wie etwa für API-Zugriffe oder Formularvalidierungen, ließ sich dadurch eine bessere Trennung von Logik und Darstellung erreichen. Die Integration mit dem State-Management-Tool Pinia ist ebenfalls eng an die Composition API gekoppelt, was die Konsistenz im Projekt stärkt.


\subsection{Einschränkungen und Gegenmaßnahmen}
Vue bringt zwar Einschränkungen hinsichtlich \ac{SEO} mit sich, insbesondere ohne \ac{SSR}. Diese lassen sich jedoch durch Techniken wie Prerendering oder den Einsatz von Frameworks wie Nuxt.js abmildern. Für komplexes State-Management stehen mit Pinia und Vuex leistungsfähige Bibliotheken zur Verfügung \cite{VueMastery2023}.

\bigskip
Insgesamt stellt Vue.js einen geeigneten Kompromiss zwischen Einfachheit, Leistung und Erweiterbarkeit dar und erweist sich als besonders geeignet für domänenspezifische Anwendungen mit moderater Komplexität. Die Balance zwischen Einstiegstauglichkeit und technischer Tiefe macht das Framework sowohl für Lernzwecke als auch für professionelle Webentwicklung attraktiv.

