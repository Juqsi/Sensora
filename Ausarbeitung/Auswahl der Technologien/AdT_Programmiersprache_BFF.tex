
\section{Wahl der Programmiersprache}
In einem großen Projekt wie Sensora ist es notwendig, einheitliche Richtlinien und Guidelines zu etablieren. Dies sorgt dafür, dass einige grundlegende Punkte von allen gleich gehandhabt werden, was wiederum die schnelle Entwicklung, Weiterentwicklung und Wartung von Software fördert und die Benutzbarkeit von Software oder Softwaremodulen erhöht.

Einer der zentralen Aspekte, die unter diesen Richtlinien fallen, ist die Wahl der Programmiersprache. Innerhalb des Sensora-Teams gibt es mannigfaltige Fähigkeiten. Eine Bestandsaufnahme der Kenntnisse ergab, dass Python von allen Entwicklern geschrieben und gelesen werden kann.
Für stark fragmentierte Module sollte jedoch eine performante Lösung geschaffen werden, die große Lasten leichter und somit kostengünstiger trägt.

Die Bestandsaufnahme lässt schließen, dass Go die dafür am stärksten vertretene Sprache unter den Entwicklern ist. Benchmarks haben jedoch ergeben, dass Rust die deutlich performantere Sprache ist. \cite{Prokopiou2021} Nachfolgend werden die technischen Besonderheiten beider Sprachen diskutiert.

\subsection{Ein Einblick in Go}
Go, auch bekannt als Golang, ist eine moderne Programmiersprache, die in den letzten Jahren stetig an Popularität gewonnen hat. \cite{JetBrains2023} Sie wurde von Google entwickelt und zielt darauf ab, eine klare und prägnante Syntax zu bieten, die gleichzeitig hohe Leistung und Skalierbarkeit ermöglicht. Mit den komplexen Anforderungen an Softwareprodukte und den vielen kooperierenden Teams war die Notwendigkeit einer effizienten und gut skalierbaren Sprache entscheidend. Go sollte insbesondere das Problem der langen Kompilierungszeiten und der Schwierigkeit, parallelen Code zu schreiben, lösen.

\subsubsection{Syntax und Struktur}
Go zeichnet sich durch eine minimalistische und leicht verständliche Syntax aus. Die Grundstruktur eines Go-Programms ist einfach und übersichtlich, was die Einarbeitung für Entwickler erleichtert. Ein einfaches Hello-World-Programm in Go sieht wie folgt aus:
\begin{lstlisting}[language=Golang]
package main

import "fmt"

func main() {
    fmt.Println("Hello, World!")
}
\end{lstlisting}
Dieser Code illustriert einige grundlegende Konzepte von Go: die Verwendung von Paketen, die explizite Importierung von Bibliotheken und die Definition der main-Funktion als Einstiegspunkt des Programms.

\subsubsection{Der Go-Compiler}
Ein wesentlicher Grund für die Effizienz von Go ist der Go-Compiler. Go verwendet einen statischen Typenchecker, was bedeutet, dass Typen zur Kompilierungszeit überprüft werden. Dies trägt zur frühzeitigen Fehlererkennung bei und verbessert die Zuverlässigkeit des Codes. Der Compiler selbst ist äußerst schnell und ermöglicht es Entwicklern, ihre Programme in kürzester Zeit zu kompilieren. Diese Schnelligkeit fördert eine agile Entwicklungsweise, da Änderungen schnell getestet werden können.

\subsubsection{Goroutines und Concurrency}

Ein herausragendes Merkmal von Go ist die native Unterstützung für Concurrency, also die Fähigkeit, mehrere Aufgaben gleichzeitig auszuführen. Go erreicht dies durch sogenannte Goroutines, die mit dem Schlüsselwort go vor einem Funktionsaufruf gestartet werden:

\begin{lstlisting}[language=Golang]
go func() {
    fmt.Println("Concurrent task")
}()
\end{lstlisting}

Goroutines sind wesentlich ressourcenschonender als herkömmliche Threads und ermöglichen es, viele davon in einer Anwendung zu betreiben, ohne den Speicher zu überlasten. Die Kommunikation zwischen Goroutines erfolgt über Channels, ein weiteres einzigartiges Konzept von Go, das die Synchronisation und den Datenaustausch vereinfacht. \cite{Kuree}

\subsubsection{Was Go ausmacht}
Go wurde entwickelt, um sowohl einfach zu erlernen als auch effizient in der Anwendung zu sein. Die Sprache verzichtet bewusst auf komplexe Features wie Vererbung, die in anderen Programmiersprachen oft zu einer steilen Lernkurve führen können. Stattdessen setzt Go auf Komposition, was zu einer besseren Lesbarkeit und Wartbarkeit des Codes führt.

Ein weiterer Vorteil von Go ist die umfangreiche Standardbibliothek, die viele der gängigen Aufgaben der Softwareentwicklung abdeckt, von der Dateiverarbeitung über Netzwerkkommunikation bis hin zur Kryptographie. Diese Standardbibliothek trägt zur Konsistenz und Zuverlässigkeit bei, da Entwickler nicht auf externe Bibliotheken angewiesen sind, die möglicherweise weniger gut gewartet werden.

Go bietet zudem eine hervorragende Cross-Platform-Kompatibilität. Der Go-Compiler erzeugt ausführbare Dateien, die auf verschiedenen Betriebssystemen laufen können, ohne dass Änderungen am Quellcode erforderlich sind. Diese Fähigkeit, plattformübergreifende Anwendungen zu erstellen, ist ein entscheidender Vorteil in einer Umgebung, in der Anwendungen auf verschiedenen Systemen bereitgestellt werden müssen. \cite{Pike}

\subsubsection{Anwendungsbereiche von Go}
Go wird in verschiedenen Anwendungsbereichen eingesetzt, die Leistung, Parallelität und Skalierbarkeit priorisieren. Typische Einsatzgebiete umfassen:
\begin{description}
    \item[Backend-Entwicklung:] Go ist ideal für die Entwicklung von Serveranwendungen, insbesondere für Web- und API-Server. Seine Fähigkeit, viele parallele Anfragen zu verarbeiten, macht es zu einer hervorragenden Wahl für hoch skalierbare Systeme.
    \item[Cloud-Computing:] Dank seiner Effizienz und Einfachheit ist Go eine bevorzugte Sprache für Cloud-native Anwendungen und Microservices.
    \item[Netzwerkdienste:] Go eignet sich hervorragend für die Entwicklung von Netzwerkdiensten wie Proxies, Gateways und Load Balancers, da es leistungsstarke Netzwerkbibliotheken und native Unterstützung für Concurrency bietet.
\end{description}

Es gibt jedoch auch Anwendungsbereiche, in denen Go weniger geeignet ist:
\begin{description}
    \item[High-Performance-Grafik und Spieleentwicklung:] Obwohl Go schnell ist, gibt es speziellere Sprachen wie C++ oder Rust, die besser für die extremen Leistungsanforderungen der Grafik- und Spieleentwicklung optimiert sind, eine breitere Palette an spezifischen Bibliotheken bieten und besser von Game-Engines unterstützt werden.
    \item[Rapid Prototyping und Skripting:] Für schnelle Prototypen oder einfache Skripte sind dynamische Sprachen wie Python oder JavaScript aufgrund ihrer Flexibilität und der großen Anzahl an verfügbaren Bibliotheken oft die bessere Wahl.
    \item[Grafische Benutzeroberflächen:] Go bietet keine einfache Möglichkeit, eine \ac{gui} zu erstellen. Hier sind Sprachen wie C\# im .NET-Frame\-work oder Java geeigneter.
\end{description}
\cite{Merrick2023}

% TODO: Rust und Vergleich
