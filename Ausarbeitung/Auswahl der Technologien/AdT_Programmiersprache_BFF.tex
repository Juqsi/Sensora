
\section{Wahl der Programmiersprache}
In einem großen Projekt wie Sensora ist es notwendig, einheitliche Richtlinien und Guidelines zu etablieren. Dies sorgt dafür, dass einige grundlegende Punkte von allen gleich gehandhabt werden, was wiederum die schnelle Entwicklung, Weiterentwicklung und Wartung von Software fördert und die Benutzbarkeit von Software oder Softwaremodulen erhöht.

Einer der zentralen Aspekte, die unter diesen Richtlinien fallen, ist die Wahl der Programmiersprache. Innerhalb des Sensora-Teams gibt es mannigfaltige Fähigkeiten. Eine Bestandsaufnahme der Kenntnisse ergab, dass Python von allen Entwicklern geschrieben und gelesen werden kann.
Für stark fragmentierte Module sollte jedoch eine performante Lösung geschaffen werden, die große Lasten leichter und somit kostengünstiger trägt.

Die Bestandsaufnahme lässt schließen, dass Go die dafür am stärksten vertretene Sprache unter den Entwicklern ist. Benchmarks haben jedoch ergeben, dass Rust die deutlich performantere Sprache ist. \cite{Prokopiou2021} Nachfolgend werden die technischen Besonderheiten beider Sprachen diskutiert.

\subsection{Ein Einblick in Go}
Go, auch bekannt als Golang, ist eine moderne Programmiersprache, die in den letzten Jahren stetig an Popularität gewonnen hat. \cite{JetBrains2023} Sie wurde von Google entwickelt und zielt darauf ab, eine klare und prägnante Syntax zu bieten, die gleichzeitig hohe Leistung und Skalierbarkeit ermöglicht. Mit den komplexen Anforderungen an Softwareprodukte und den vielen kooperierenden Teams war die Notwendigkeit einer effizienten und gut skalierbaren Sprache entscheidend. Go sollte insbesondere das Problem der langen Kompilierungszeiten und der Schwierigkeit, parallelen Code zu schreiben, lösen.

\subsubsection{Syntax und Struktur}
Go zeichnet sich durch eine minimalistische und leicht verständliche Syntax aus. Die Grundstruktur eines Go-Programms ist einfach und übersichtlich, was die Einarbeitung für Entwickler erleichtert. Ein einfaches Hello-World-Programm in Go sieht wie folgt aus:
\begin{lstlisting}[language=Golang]
package main

import "fmt"

func main() {
    fmt.Println("Hello, World!")
}
\end{lstlisting}
Dieser Code illustriert einige grundlegende Konzepte von Go: die Verwendung von Paketen, die explizite Importierung von Bibliotheken und die Definition der main-Funktion als Einstiegspunkt des Programms.

\subsubsection{Der Go-Compiler}
Ein wesentlicher Grund für die Effizienz von Go ist der Go-Compiler. Go verwendet einen statischen Typenchecker, was bedeutet, dass Typen zur Kompilierungszeit überprüft werden. Dies trägt zur frühzeitigen Fehlererkennung bei und verbessert die Zuverlässigkeit des Codes. Der Compiler selbst ist äußerst schnell und ermöglicht es Entwicklern, ihre Programme in kürzester Zeit zu kompilieren. Diese Schnelligkeit fördert eine agile Entwicklungsweise, da Änderungen schnell getestet werden können.

\subsubsection{Goroutines und Concurrency}

Ein herausragendes Merkmal von Go ist die native Unterstützung für Concurrency, also die Fähigkeit, mehrere Aufgaben gleichzeitig auszuführen. Go erreicht dies durch sogenannte Goroutines, die mit dem Schlüsselwort go vor einem Funktionsaufruf gestartet werden:

\begin{lstlisting}[language=Golang]
go func() {
    fmt.Println("Concurrent task")
}()
\end{lstlisting}

Goroutines sind wesentlich ressourcenschonender als herkömmliche Threads und ermöglichen es, viele davon in einer Anwendung zu betreiben, ohne den Speicher zu überlasten. Die Kommunikation zwischen Goroutines erfolgt über Channels, ein weiteres einzigartiges Konzept von Go, das die Synchronisation und den Datenaustausch vereinfacht. \cite{Kuree}

\subsubsection{Was Go ausmacht}
Go wurde entwickelt, um sowohl einfach zu erlernen als auch effizient in der Anwendung zu sein. Die Sprache verzichtet bewusst auf komplexe Features wie Vererbung, die in anderen Programmiersprachen oft zu einer steilen Lernkurve führen können. Stattdessen setzt Go auf Komposition, was zu einer besseren Lesbarkeit und Wartbarkeit des Codes führt.

Ein weiterer Vorteil von Go ist die umfangreiche Standardbibliothek, die viele der gängigen Aufgaben der Softwareentwicklung abdeckt, von der Dateiverarbeitung über Netzwerkkommunikation bis hin zur Kryptographie. Diese Standardbibliothek trägt zur Konsistenz und Zuverlässigkeit bei, da Entwickler nicht auf externe Bibliotheken angewiesen sind, die möglicherweise weniger gut gewartet werden.

Go bietet zudem eine hervorragende Cross-Platform-Kompatibilität. Der Go-Compiler erzeugt ausführbare Dateien, die auf verschiedenen Betriebssystemen laufen können, ohne dass Änderungen am Quellcode erforderlich sind. Diese Fähigkeit, plattformübergreifende Anwendungen zu erstellen, ist ein entscheidender Vorteil in einer Umgebung, in der Anwendungen auf verschiedenen Systemen bereitgestellt werden müssen. \cite{Pike}

\subsubsection{Anwendungsbereiche von Go}
Go wird in verschiedenen Anwendungsbereichen eingesetzt, die Leistung, Parallelität und Skalierbarkeit priorisieren. Typische Einsatzgebiete umfassen:
\begin{description}
    \item[Backend-Entwicklung:] Go ist ideal für die Entwicklung von Serveranwendungen, insbesondere für Web- und API-Server. Seine Fähigkeit, viele parallele Anfragen zu verarbeiten, macht es zu einer hervorragenden Wahl für hoch skalierbare Systeme.
    \item[Cloud-Computing:] Dank seiner Effizienz und Einfachheit ist Go eine bevorzugte Sprache für Cloud-native Anwendungen und Microservices.
    \item[Netzwerkdienste:] Go eignet sich hervorragend für die Entwicklung von Netzwerkdiensten wie Proxies, Gateways und Load Balancers, da es leistungsstarke Netzwerkbibliotheken und native Unterstützung für Concurrency bietet.
\end{description}

Es gibt jedoch auch Anwendungsbereiche, in denen Go weniger geeignet ist:
\begin{description}
    \item[High-Performance-Grafik und Spieleentwicklung:] Obwohl Go schnell ist, gibt es speziellere Sprachen wie C++ oder Rust, die besser für die extremen Leistungsanforderungen der Grafik- und Spieleentwicklung optimiert sind, eine breitere Palette an spezifischen Bibliotheken bieten und besser von Game-Engines unterstützt werden.
    \item[Rapid Prototyping und Skripting:] Für schnelle Prototypen oder einfache Skripte sind dynamische Sprachen wie Python oder JavaScript aufgrund ihrer Flexibilität und der großen Anzahl an verfügbaren Bibliotheken oft die bessere Wahl.
    \item[Grafische Benutzeroberflächen:] Go bietet keine einfache Möglichkeit, eine \ac{gui} zu erstellen. Hier sind Sprachen wie C\# im .NET-Frame\-work oder Java geeigneter.
\end{description}
\cite{Merrick2023}

% TODO: Rust und Vergleich
\subsection{Einblicke in Rust}
Rust ist eine moderne, systemnahe Programmiersprache, die mit dem Ziel entwickelt wurde, Speichersicherheit, hohe Performance und nebenläufige Programmierung ohne Laufzeit-Overhead zu ermöglichen. Sie wurde ursprünglich von Mozilla Research initiiert und 2015 in Version 1.0 veröffentlicht. Im Zentrum steht ein Ownership-Modell, das Speicherfehler wie Null-Pointer, Use-after-free oder Datenrennen zur Compile-Zeit verhindert – ohne Garbage Collector. Damit schließt Rust eine Lücke zwischen sicherem Hochsprachenkomfort und der Kontrolle traditioneller Low-Level-Sprachen. In der heutigen Softwarelandschaft wird Rust zunehmend in Bereichen wie WebAssembly, Embedded Systems, Systemtools und Backend-Entwicklung eingesetzt. Als Alternative zu C/C++ und Go vereint Rust Systems Programming mit modernen Sprachfeatures und rückt dadurch ins Zentrum aktueller Softwareentwicklung.

\subsection{Syntax und Struktur}
Rust kombiniert eine moderne, präzise Syntax mit klarer Struktur. Die Sprache ist ausdrucksstark, stark typisiert und legt großen Wert auf Lesbarkeit und Sicherheit. Funktionen, Kontrollstrukturen und Fehlerbehandlung orientieren sich an funktionalen und systemnahen Paradigmen. Die Standardstruktur eines Rust-Programms besteht aus Funktionen, Modulen und typstarken Definitionen. Dabei ist der Einstieg einfach und direkt.

\begin{lstlisting}[language=Rust]
fn main() {
    println!("Hello, world!");
}
\end{lstlisting}
Der Code ist kompakt, typsicher und führt ohne Boilerplate direkt zu einer lauffähigen Applikation.

Diese minimalistische Struktur zeigt Rusts Ziel, Klarheit mit technischer Kontrolle zu verbinden – vom kleinen Programm bis zur komplexen Anwendung.
\subsubsection{Grundprinzipien und Sprachdesign von Rust}

\paragraph{Kein Garbage Collector: Warum und wie das funktioniert}
Rust garantiert Speichersicherheit ohne \ac{gc}. Statt automatischer Laufzeitüberwachung verwaltet der Compiler den Speicher zur Compile-Zeit mithilfe des Ownership-Systems. Jede Variable besitzt genau einen Eigentümer; beim Transfer von Ownership wird sichergestellt, dass es keine doppelten Freigaben oder Dangling Pointers gibt. Referenzen unterliegen Borrowing-Regeln, die eine gleichzeitige, aber kontrollierte Nutzung ermöglichen. Diese statische Analyse erfolgt im sogenannten Borrow Checker, der Teil der rustc-Pipeline ist \cite{mark-i-m2025}. Dadurch wird Speicher deterministisch freigegeben, ohne Garbage Collection oder manuelles free.

\paragraph{Performance durch Systemnähe und zero-cost abstractions}
Rust wird direkt zu Maschinencode kompiliert und verzichtet auf eine virtuelle Maschine oder Interpretationsschicht. Die Sprache erlaubt es, Low-Level-Operationen präzise zu steuern (z.B. Speicherlayout, Alignment, Inline-Assembler), ohne Sicherheitsgarantien zu opfern. Gleichzeitig ermöglichen Sprachfeatures wie Traits, Pattern Matching und Iteratoren hochabstrakte Konstrukte, die durch Monomorphisierung zur Compile-Zeit in effizienten Code übersetzt werden. Diese sogenannten zero-cost abstractions erzeugen keinen Overhead – was insbesondere für sicherheitskritische oder performance-sensitive Anwendungen relevant ist \cite{SteveKlabnik2024}.

\paragraph{Typensicherheit und Fehlervermeidung zur Compile-Zeit}
Rust setzt auf ein starkes, statisches Typsystem mit expliziten Typannotationen, generischen Typen und algebraischen Datentypen (enum, Option, Result). Dadurch werden viele Fehler – wie Null-Zugriffe, Typverwechslungen oder unbehandelte Rückgabewerte – bereits beim Kompilieren erkannt. Der Compiler prüft nicht nur Typverträglichkeit, sondern auch die Einhaltung von Lifetime-Regeln und Borrowing-Konflikten. Seit der Einführung \ac{nll} kann rustc auch kontextabhängige Lebensdauern dynamischer erkennen und erlaubt damit flexiblere, aber weiterhin sichere Referenznutzung \cite{Matsakis2022}. Das Ergebnis ist robustere Software mit weniger Laufzeitfehlern.

\subsubsection{Speicher- und Referenzmanagement}

\paragraph{Das Ownership-Modell}
Rusts Ownership-Modell ist zentral für die speichersichere Programmierung ohne \ac{gc}. Jeder Wert hat genau einen Besitzer. Wird ein Wert einer anderen Variablen zugewiesen oder an eine Funktion übergeben, geht das Eigentum über, und der ursprüngliche Besitzer verliert den Zugriff. Dies verhindert doppelte Freigaben und Dangling Pointers. Beim Verlassen des Gültigkeitsbereichs wird der Wert automatisch freigegeben, was durch das Drop-Trait ermöglicht wird. \cite{wiki2024rust}

\paragraph{Mutable vs. Immutable Borrowing}

Neben dem Eigentum erlaubt Rust das Ausleihen von Werten durch Referenzen. Es gibt zwei Arten:
\begin{description}
    \item[Immutable Borrowing \code{\&T}:] Erlaubt beliebig viele gleichzeitige, aber nur lesende Zugriffe.
    \item[Mutable Borrowing \code{\&mut T}:] Erlaubt genau einen schreibenden Zugriff, aber keine weiteren gleichzeitigen Referenzen.
\end{description}

Diese Regeln verhindern Datenrennen und garantieren, dass keine Referenz während einer Mutation auf veraltete Daten zeigt. Der Compiler erzwingt diese Regeln strikt zur Compile-Zeit.

\paragraph{Der Borrow Checker: Statische Analyse zur Compile-Zeit}
Der Borrow Checker ist ein zentrales Werkzeug des Rust-Compilers, das sicherstellt, dass die Regeln des Ownership- und Borrowing-Modells eingehalten werden. Er analysiert den Code während der Kompilierung und verhindert beispielsweise, dass ein Wert nach einem Move weiterhin verwendet wird oder dass gleichzeitig eine mutable und eine immutable Referenz existieren. Diese statische Analyse eliminiert eine Vielzahl von Speicherfehlern bereits vor der Ausführung des Programms.
Medium

\paragraph{Lifetimes und Lebensdauern von Referenzen}
Lifetimes sind ein Mechanismus in Rust, um die Gültigkeitsdauer von Referenzen zu verfolgen und sicherzustellen, dass sie nicht auf ungültigen Speicher zeigen. Der Compiler kann in vielen Fällen die Lifetimes automatisch ableiten, aber in komplexeren Szenarien müssen sie explizit angegeben werden. Seit der Einführung der \ac{nll} ist der Compiler in der Lage, die Lebensdauer von Referenzen flexibler und präziser zu bestimmen, was die Programmierung erleichtert und die Sicherheit erhöht.

\subsubsection{Webentwicklung mit actix-web}
Actix-web ist eines der populärsten Web-Frameworks in Rust und bekannt für seine hohe Performance, geringe Latenz und speichersichere Architektur. Es basiert auf dem Actor-Modell (über die actix-Laufzeit) und bietet eine asynchrone, eventgesteuerte Serverstruktur. Das Framework nutzt Rusts async/await-Syntax vollständig aus und integriert sich nahtlos mit Tokio oder anderen async-Runtimes.

Die API von actix-web ist modular aufgebaut und erlaubt eine saubere Trennung von Routen, Middleware und Handlern. Dank des Typsystems können viele typische Webfehler (z.B. ungültige Parameter) bereits zur Compile-Zeit verhindert werden. actix-web unterstützt \ac{rest}, WebSockets, \ac{tls}, Body-Parsing, \ac{cors} und viele weitere Funktionen „out of the box“.

\subsection{Vergleich: Rust vs. Go}

\subsubsection{Garbage Collection vs. Ownership}
Go verwendet eine automatische \ac{gc}, die Speicher zur Laufzeit freigibt, sobald Objekte nicht mehr referenziert werden. Dies vereinfacht die Entwicklung, kann jedoch zu unvorhersehbaren Pausen führen, insbesondere bei speicherintensiven Anwendungen. \cite{Bhavya2021}

Rust hingegen verzichtet vollständig auf eine \ac{gc}. Stattdessen setzt es auf ein Ownership-Modell mit statischer Analyse zur Compile-Zeit. Jeder Wert hat einen eindeutigen Besitzer, und der Compiler stellt sicher, dass keine ungültigen Referenzen oder Datenrennen entstehen. Dies ermöglicht eine deterministische Speicherfreigabe ohne Laufzeit-Overhead. \cite{puri2025}

\subsubsection{Unterschiede in Sicherheit, Performance und Parallelismus}
\begin{description}
    \item[Sicherheit:] Rust bietet durch sein Ownership- und Borrowing-System eine hohe Speichersicherheit. Der Compiler verhindert viele Fehler bereits zur Compile-Zeit, wie z.B. Null-Pointer-Dereferenzen oder Datenrennen. Go bietet ebenfalls Mechanismen zur Speichersicherheit, jedoch werden bestimmte Fehler erst zur Laufzeit erkannt.
    \item [Performance:] Rust erzielt in Benchmarks häufig bessere Laufzeiten und geringeren Speicherverbrauch als Go, insbesondere bei CPU-intensiven Aufgaben. \cite{Proxify2021} Dies liegt an der fehlenden \ac{gc} und den Zero-Cost-Abstraktionen von Rust.\cite{Abhinav2025}
    \item[Parallelismus:] Go erleichtert die nebenläufige Programmierung durch Goroutines und Channels, was eine einfache Handhabung von Parallelität ermöglicht. \cite{dami2024} Allerdings liegt die Verantwortung für die Vermeidung von Datenrennen beim Entwickler. Rust hingegen erzwingt durch sein Typsystem und den Borrow Checker sichere Parallelität, indem es Datenrennen bereits zur Compile-Zeit verhindert. \cite{Obregon2023}
\end{description}
