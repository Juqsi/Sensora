\section{Ausblick Frontend}

Im weiteren Verlauf der Entwicklung bestehen vielfältige Potenziale zur funktionalen und gestalterischen Erweiterung des Frontends. Ein naheliegender Ansatzpunkt ist die Vertiefung der Gamification-Strategien. Erste Ideen, etwa visuelle Wetter-Overlays wie eine Sonne bei Trockenheit oder eine Regenwolke bei Staunässe oberhalb des Pflanzen-Avatars, wurden bereits erprobt, konnten jedoch aus Zeitgründen nicht final implementiert werden. Diese Erweiterung würde die emotionale Bindung zur virtuellen Pflanze weiter steigern und eine intuitivere Wahrnehmung ihres Zustandes ermöglichen.

Auch im Bereich der Datenvisualisierung besteht weiteres Entwicklungspotenzial. Die aktuelle Version der Anwendung zeigt Messwerte innerhalb eines Zeitfensters von 24 Stunden an. Die zugrundeliegenden Methoden und Datenmodelle erlauben jedoch eine verlängerte Betrachtung, beispielsweise über Tage oder Wochen hinweg. Eine benutzerfreundliche Auswahlkomponente für den gewünschten Zeitraum könnte hier eine wertvolle Ergänzung darstellen, um Langzeitveränderungen besser nachvollziehen zu können.

Ein weiterer spannender Entwicklungsschritt betrifft die Integration aktiver Steuerungsmechanismen für die Pflanzenbewässerung. Mit entsprechender Konfiguration und der Anbindung wäre es möglich, die Applikation um eine Live-Bewässerungsfunktion zu erweitern. Auch automatisierte Abläufe, etwa zur Simulation von Tageszyklen oder zur reaktiven Anpassung an Umweltdaten, könnten in einem erweiterten System realisiert werden.

Diese möglichen Erweiterungen unterstreichen das technische und konzeptionelle Potenzial der aktuellen Frontend-Architektur und liefern wertvolle Ansatzpunkte für eine fortlaufende Weiterentwicklung.

