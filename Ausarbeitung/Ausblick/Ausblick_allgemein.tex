Mit dem erfolgreichen Abschluss des vorliegenden Projekts konnte ein funktionierender Prototyp für ein automatisiertes Bewässerungssystem entwickelt werden, das zentrale Anforderungen hinsichtlich Datenerfassung, Steuerung und Kommunikation erfüllt. Gleichzeitig bietet das System vielfältige Ansatzpunkte für eine Weiterentwicklung und Professionalisierung in technischer, organisatorischer und konzeptioneller Hinsicht.

\subsection*{Potenziale für Systemerweiterungen}

Die modulare Architektur des Systems ermöglicht eine schrittweise Erweiterung und Anpassung. Denkbar sind beispielsweise die Einbindung zusätzlicher Sensorik (z.\,B. Boden-pH, CO\textsubscript{2}, Regen), die Ansteuerung mehrerer Pumpen in unterschiedlichen Zonen oder die Einbindung von Aktoren wie Ventile oder Ventilatoren. Auch eine Integration in bestehende Gartenmanagement- oder Smart-Home-Systeme eröffnet neue Anwendungsszenarien.

\subsection*{Optimierung der Systemintegration}

Eine zentrale Herausforderung in der aktuellen Projektphase bestand in der Koordination zwischen Microcontroller, Backend und mobiler App. Künftig sollte die Systemintegration stärker auf standardisierte Protokolle und Datenformate setzen, um die Wartbarkeit und Austauschbarkeit einzelner Komponenten zu erhöhen. Insbesondere die Etablierung klar definierter Schnittstellen sowie eine kontinuierliche Integration (\textit{CI}) könnten die Zusammenarbeit und Weiterentwicklung erleichtern.

\subsection*{Produktisierung und Praxistauglichkeit}

Die Überführung des entwickelten Prototyps in ein marktfähiges Produkt würde eine Reihe zusätzlicher Anforderungen mit sich bringen, etwa hinsichtlich Gehäusedesign, Stromversorgung (z.\,B. Akku, Solarpanel), Energieoptimierung und Benutzerfreundlichkeit. Auch die Robustheit gegenüber Umwelteinflüssen und die Einhaltung regulatorischer Anforderungen müssten bei einer Produktisierung geprüft und berücksichtigt werden.

\subsection*{Nutzerzentrierung und Usability}

Während der Fokus des vorliegenden Projekts primär auf der technischen Realisierung lag, wäre eine stärkere Einbindung der Perspektive potenzieller Nutzer in weiteren Projektphasen wünschenswert. Dazu zählen beispielsweise eine intuitive Benutzeroberfläche in der App, eine transparente Darstellung der Messwerte sowie einfache Möglichkeiten zur Konfiguration und Fehlerdiagnose. Auch Aspekte wie Mehrsprachigkeit, Barrierefreiheit oder eine smarte Benachrichtigungslogik könnten den Anwendungskomfort erheblich steigern.

\subsection*{Langfristige Perspektiven und Forschungspotenzial}

Das Projekt bietet über den unmittelbaren Anwendungsfall hinaus auch spannende Anknüpfungspunkte für weiterführende Forschung. So könnten beispielsweise adaptive Systeme zur Ressourcenschonung in der Landwirtschaft, autonome Sensornetzwerke oder KI-gestützte Umweltanalysen auf der vorliegenden Arbeit aufbauen. Auch die Verbindung mit externen Wetterdiensten, Satellitendaten oder cloudbasierten Agrarsystemen bietet interessante Perspektiven.

