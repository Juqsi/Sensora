\section{Scrum und das Agile Manifest}
Scrum basiert auf den Grundsätzen des Agilen Manifests. Das Agile Manifest ist eine Sammlung von Priorisierungsprinzipien, die im Jahr 2001 von 17 Experten und Vertretern unterschiedlicher agiler Vorgehensweisen in Snowbird, USA, entwickelt wurde. Diese Gruppe, oft als die \glqq Snowbird 17\grqq{} bezeichnet, erkannte frühzeitig die Notwendigkeit einer neuen Ära der Softwareentwicklung. \cite{Drumond} Das Agile Manifest umfasst im Original nur 68 Wörter, aber aus diesem knappen Text wurden allgemeingültige Prinzipien abgeleitet, die bis heute die Grundlage für agile Methoden wie Scrum bilden.

Das Manifest lautet sinngemäß ins Deutsche übersetzt: 

\vspace{1em}
\glqq Wir erschließen bessere Wege, Software zu entwickeln, indem wir es selbst tun und anderen dabei helfen. Durch diese Tätigkeit haben wir diese Werte zu schätzen gelernt:

\noindent \textbf{Individuen und Interaktionen}
\vspace{-1em}
\begin{flushright}
mehr als Prozesse und Werkzeuge,
\end{flushright}
\vspace{-1em}
\textbf{Funktionierende Software}
\vspace{-1em}
\begin{flushright}
mehr als umfassende Dokumentation,
\end{flushright}
\vspace{-1em}
\textbf{Zusammenarbeit mit dem Kunden}
\vspace{-1em}
\begin{flushright}
mehr als Vertragsverhandlung,
\end{flushright}
\vspace{-1em}
\textbf{Reagieren auf Veränderung}
\vspace{-1em}
\begin{flushright}
mehr als das Befolgen eines Plans.
\end{flushright}

Das heißt, obwohl wir die Werte auf der rechten Seite wichtig finden, schätzen wir die Werte auf der linken Seite höher ein.\grqq{} \cite{Snowbird2001}

\subsection{Interpretation des Agilen Manifests in der Praxis}
Das Agile Manifest betont die Wichtigkeit von Praxisnähe in der agilen Entwicklung. Es reicht nicht aus, nur theoretische Konzepte zu entwickeln; vielmehr müssen praktische Erfahrungen gesammelt und in den Entwicklungsprozess eingebracht werden. Ein zentrales Prinzip ist, dass die Individuen und ihre Interaktionen im Vordergrund stehen. Dies bedeutet, dass ein Vorgehen gefunden werden muss, das eine effektive Kommunikation und Interaktion aller Beteiligten ermöglicht. Prozesse und Werkzeuge sollten an die Bedürfnisse der Menschen angepasst werden, nicht umgekehrt.

Darüber hinaus wird die Bedeutung funktionierender Software hervorgehoben. Der Fortschritt eines Projekts wird anhand der tatsächlich funktionierenden Software gemessen, nicht anhand umfangreicher Dokumentationen. Es ist entscheidend, dass die entwickelte Software in regelmäßigen Abständen gezeigt und von den Anwendern beurteilt wird. Dies stellt sicher, dass das Projekt auf dem richtigen Weg bleibt und die Anforderungen der Nutzer erfüllt.

Da Softwareentwicklung ein dynamischer Prozess ist, entstehen oft neue Herausforderungen oder Anforderungen, selbst bei sorgfältiger Planung. Daher ist es unerlässlich, flexibel auf Veränderungen zu reagieren. Der Erfolg eines Projekts wird daran gemessen, wie gut es sich an neue Gegebenheiten anpasst und ob daraus ein Lerneffekt resultiert, der das Projekt voranbringt.

Auch wenn das Manifest die Bedeutung von Prozessen, Dokumentation, Verträgen und Plänen anerkennt, stellt es klar, dass diese Aspekte im Vergleich zu den übergeordneten Prinzipien von geringerer Priorität sind. Sie haben jedoch weiterhin ihre Daseinsberechtigung und müssen in einem angemessenen Maße berücksichtigt werden. \cite{Wolf2011}

\subsection{Zusammenhang zwischen Scrum und dem Agilen Manifest}
Scrum operationalisiert die Prinzipien des Agilen Manifests in einem strukturierten Rahmenwerk. Die regelmäßigen Sprints und die damit verbundenen Meetings – wie das Sprint Planning, Daily Stand-ups, Sprint Reviews und Retrospektiven – fördern die Kommunikation und die Interaktion zwischen den Teammitgliedern und den Stakeholdern. Durch die iterative Natur von Scrum wird sichergestellt, dass funktionierende Software frühzeitig und kontinuierlich geliefert wird, wodurch die Kundenzufriedenheit gesteigert wird.

Die enge Zusammenarbeit mit dem Kunden, die in Scrum durch die Rolle des Product Owners verkörpert wird, gewährleistet, dass das Entwicklungsteam ständig auf die sich ändernden Anforderungen reagieren kann. Diese Flexibilität ist ein direkter Ausdruck des Wertes \glqq Reagieren auf Veränderung mehr als das Befolgen eines Plans\grqq, der im Agilen Manifest verankert ist.


\subsection{Anwendung von Scrum im Sensora-Projekt}
Aus den Grundsätzen des Agilen Manifests ergeben sich spezifische Herangehensweisen für das Sensora-Projekt:
\begin{description}
    \item[Zwischenergebnisse:] Es wird eine hohe Frequenz bei der Präsentation von Zwischenergebnissen angestrebt. Diese regelmäßigen Präsentationen bieten eine hervorragende Gelegenheit, um mit den Stakeholdern in den Dialog zu treten, Feedback zu sammeln und Verbesserungen zu identifizieren. Zudem fungieren diese Präsentationen als Indikatoren für den Projektfortschritt, wodurch erkennbar wird, ob das Projekt planmäßig voranschreitet oder ob Maßnahmen zur Kurskorrektur erforderlich sind.
    \item[Kommunikation:] Alle technischen Entscheidungen werden in enger Abstimmung mit allen Entwicklern getroffen. Dabei werden die Meinungen und Bedenken der beteiligten Personen berücksichtigt, um sicherzustellen, dass realistische Lösungen verfolgt werden. Durch diesen intensiven Austausch wird verhindert, dass Zeit und Ressourcen in ineffiziente oder unpraktikable Lösungen investiert werden. Gleichzeitig wird sichergestellt, dass das kollektive Wissen genutzt wird und potenzielle Probleme frühzeitig erkannt werden.
    \item[Lessons Learned:] Im Verlauf der Entwicklung entstehen neue Erkenntnisse, die zu neuen Möglichkeiten führen. Diese Lernfortschritte, sowohl auf fachlicher als auch auf technischer Ebene, werden genutzt, um das Projekt kontinuierlich weiterzuentwickeln und anzupassen. Die Fähigkeit, aus Erfahrungen zu lernen und diese in den Entwicklungsprozess einzubeziehen, ist ein zentraler Bestandteil des agilen Vorgehens.
\end{description}

Zusammengefasst projiziert Scrum die Werte und Prinzipien des Agilen Manifests auf einen praxisnahen und strukturierten Entwicklungsprozess, der es Teams ermöglicht, effizient und flexibel auf die Herausforderungen der Softwareentwicklung zu reagieren. Durch die Integration dieser Prinzipien in das Sensora-Projekt wird sichergestellt, dass das Produkt den Anforderungen gerecht wird und gleichzeitig eine hohe Qualität und Benutzerfreundlichkeit erreicht.
