
\section{Designprinzipien und -muster}
Die Entwicklung von Sensora basiert auf einer klar definierten und bewährten Architektur, die die spezifischen Anforderungen an eine flexible, skalierbare und sichere Benachrichtigungsplattform erfüllt. Im Rahmen dieser Entwicklung wurden verschiedene Designprinzipien und -muster berücksichtigt, die sicherstellen, dass das System nicht nur leistungsfähig, sondern auch leicht wartbar und zukunftssicher ist.

\subsection{Prinzipiengeleitetes Design}
Sensora wurde unter strikter Beachtung der Coding-Guidlines entwickelt, einer Reihe von spezifischen Richtlinien, auf die sich die Entwickler geeinigt haben. Diese Guidlines werden an oberster Stelle während des gesamten Entwicklungsprozesses befolgt, um eine einheitlich hohe Qualität in der Architektur und im Code zu gewährleisten. Die \ac{api} wurde gemäß den Best Practices gestaltet, um eine nahtlose Integration in die Softwarelandschaft des Gesamtprodukts zu gewährleisten. Diese Integration wurde durch die strikte Einhaltung der Prinzipien zur \ac{api}-Entwicklung, einschließlich der Nutzung von \ac{rest} für synchrone und asynchronen Kommunikation, ermöglicht. Zudem wurden Sicherheitsprinzipien, insbesondere das Zero Trust-Modell, konsequent umgesetzt, wodurch jede Anfrage eines Clients neu authentifiziert wird.

\subsection{Coding-Guidlines}
Die Coding-Guidlines stellen ein umfassendes Regelwerk dar, das die Grundlage für die Softwareentwicklung bildet. Diese Prinzipien sind mehr als nur Richtlinien; sie sind integraler Bestandteil der Projektkultur und stellen sicher, dass alle Entwicklungsprojekte nach denselben hohen Standards durchgeführt werden. Die Coding-Guidlines decken eine Vielzahl von Aspekten ab, von der Architektur über die Server-Nutzung bis hin zur Sicherheit und Best Practices für spezifische Programmiersprachen wie Python oder Rust.

Ein zentrales Prinzip ist das der losen Kopplung. Es fordert, dass Anwendungen nur über sprachunabhängige Protokolle miteinander kommunizieren. Dies fördert die Modularität und erleichtert die Wartung sowie die Integration neuer Systeme. In Sensora wird dies durch die strikte Trennung der Module und die Nutzung von klar definierten Schnittstellen erreicht. Jedes Modul ist eigenständig und kommuniziert über definierte \acp{api}, was die Austauschbarkeit und Wiederverwendbarkeit der Komponenten sicherstellt.

Das Zero Trust-Sicherheitsmodell ist ein weiteres kritisches Element der Coding-Guidlines. Es besagt, dass keine Entität – weder Benutzer noch Gerät oder Netzwerk – automatisch vertraut wird, unabhängig davon, ob sie sich innerhalb oder außerhalb des Netzwerks befindet. Stattdessen wird jede Anfrage verifiziert. Authentifizierung und Autorisierung erfolgen kontinuierlich. Zugriffe werden auf das Minimum beschränkt, basierend auf dem Prinzip der geringsten Privilegien. Diese Sicherheitsanforderung wurde in Sensora durch die Integration von \acp{jwt} umgesetzt, sodass jede Anfrage an einen Service strengen Authentifizierung unterliegt.

Abgerundet werden die Coding-Guidlines durch eine starke Fokussierung auf Best Practices und Clean Code. Diese umfassen spezifische Regeln für die Nutzung von Rust, wie beispielsweise die Bevorzugung von etablierten Bibliotheken wie \code{acti\_web}, die Verwendung von \code{match}-Ausdrücken statt verschachtelter \code{if}-Bedingungen und die Anwendung des \code{?}-Operators zur Verbesserung der Lesbarkeit und Sicherheit des Codes. Diese Praktiken tragen dazu bei, dass der Code von Sensora nicht nur funktional, sondern auch wartbar und erweiterbar ist.

Durch die strikte Beachtung der Coding-Guidlines bei der Entwicklung von Sensora konnte ein System geschaffen werden, das nicht nur den hohen technischen Anforderungen entspricht, sondern auch die langfristigen Ziele in Bezug auf Nachhaltigkeit, Sicherheit und Effizienz unterstützt. Diese Prinzipien sind somit ein wesentlicher Bestandteil der Architektur und des Designs von Sensora und bilden das Fundament für alle getroffenen Entscheidungen während der Entwicklung.

\subsection{Event-Driven Architecture}
Ein zentrales architektonisches Muster, das bei der Entwicklung von Sensora bewusst gewählt wurde, ist die Event-Driven Architecture. Dieses Muster eignet sich besonders gut für die Verarbeitung von Messungen, die asynchron generiert und verteilt werden müssen. Die Entscheidung für eine Event-Driven Architecture ermöglicht es, auf Ereignisse in Echtzeit zu reagieren und Daten effizient zu verteilen, ohne die Performance des Systems zu beeinträchtigen. Solace, das als Messaging-System innerhalb von Sensora eingesetzt wird, spielt hierbei eine entscheidende Rolle, indem es stabile und zuverlässige asynchrone Kommunikation sicherstellt.

\subsection{Herausforderungen und Lösungen}
Eine der zentralen Herausforderungen bei der Implementierung der Event-Driven Architecture war die asynchrone Benachrichtigung der Clients. Diese Herausforderung wurde erfolgreich durch die Integration von Solace gemeistert, das als stabiles und zuverlässiges Messaging-System fungiert. Ein weiteres potenzielles Problem, nämlich die Handhabung von Zugriffskollisionen bei gleichzeitigen Datenbankzugriffen, wurde durch die Verwendung von PostgreSQL adressiert. PostgreSQL bietet ein fortschrittliches Session-Management, das automatisch Kollisionen bei gleichzeitigen Zugriffen verwaltet, sodass Sensora selbst keine zusätzlichen Mechanismen zur Kollisionsvermeidung implementieren musste.

\subsection{Zusammenfassung}
Die Entwicklung von Sensora basiert auf einer durchdachten Kombination aus bewährten Designprinzipien und modernen Architekturmustern. Durch die konsequente Anwendung der Coding-Guidlines und die Nutzung einer Event-Driven Architecture wurde ein System geschaffen, das sowohl leistungsfähig als auch flexibel ist. Die Berücksichtigung von Best Practices und die Umsetzung eines strengen Sicherheitsmodells gewährleisten, dass Sensora nicht nur den aktuellen Anforderungen gerecht wird, sondern auch zukunftssicher und erweiterbar bleibt.
