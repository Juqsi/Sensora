\section{Benutzerzentriertes Design und UI/UX im Frontend}
\label{chap:uiux}

Im Rahmen der theoretischen Grundlagen ist die Gestaltung einer benutzerfreundlichen und intuitiven Benutzeroberfläche ein zentraler Aspekt der Frontend-Entwicklung. Gerade bei \acp{SPA} ist die Nutzerinteraktion entscheidend für den Erfolg der Anwendung, da technische Funktionalität und Design eng miteinander verzahnt sind. In diesem Kapitel werden daher zentrale Konzepte wie User-Centered Design (UCD), Usability-Heuristiken, Responsive Design sowie prototypische Evaluationsmethoden behandelt und anhand des smarten Bewässerungssystems konkretisiert.

\subsection{User-Centered Design (UCD)}

\ac{UCD} bezeichnet einen iterativen Gestaltungsprozess, der die Bedürfnisse, Anforderungen und Einschränkungen der EndnutzerInnen systematisch in den Mittelpunkt stellt \cite{IxDF_UCD2025}. Ziel ist es, Systeme zu entwickeln, die für die intendierte Zielgruppe funktional, zugänglich und zufriedenstellend nutzbar sind. Die methodische Umsetzung umfasst typischerweise vier Phasen: Nutzerforschung, Anforderungsdefinition, Prototyping und Evaluation.

\subsection{Usability-Heuristiken nach Nielsen}

Die zehn Usability-Heuristiken von Jakob Nielsen gelten als grundlegende Prinzipien zur Bewertung der Gebrauchstauglichkeit grafischer Benutzungsschnittstellen \cite{Guimaraes2021}. Diese lauten:

\begin{enumerate}
	\item \textbf{Sichtbarkeit des Systemstatus}: NutzerInnen sollten stets darüber informiert sein, was im System vorgeht. Dies erfolgt durch visuelles oder auditives Feedback in angemessener Zeit.
	\item \textbf{Entsprechung zwischen System und realer Welt}: Das System sollte die Sprache der NutzerInnen sprechen, mit vertrauten Begriffen, Konzepten und logischen Abläufen.
	\item \textbf{Benutzerkontrolle und -freiheit}: NutzerInnen sollten stets die Möglichkeit haben, unbeabsichtigte Aktionen rückgängig zu machen oder zu verlassen (z.\,B. durch eine \enquote{Zurück}-Funktion).
	\item \textbf{Konsistenz und Standards}: Einheitliche Begriffe, Symbole und Designmuster sollten in der gesamten Anwendung verwendet werden, um Lernaufwand zu minimieren.
	\item \textbf{Fehlervorbeugung}: Systeme sollten so gestaltet sein, dass Fehler bereits im Vorfeld vermieden werden (z.\,B. durch Eingabebeschränkungen oder Bestätigungsdialoge).
	\item \textbf{Wiedererkennung statt Erinnerung}: Die Benutzeroberfläche sollte Informationen sichtbar und abrufbar machen, anstatt sich auf das Gedächtnis der NutzerInnen zu verlassen.
	\item \textbf{Flexibilität und Effizienz der Nutzung}: Die Anwendung sollte sowohl AnfängerInnen als auch fortgeschrittene NutzerInnen durch Tastenkombinationen oder Shortcuts unterstützen.
	\item \textbf{Ästhetisches und minimalistisches Design}: Die Darstellung sollte sich auf relevante Informationen beschränken und keine irrelevanten oder selten gebrauchten Inhalte enthalten.
	\item \textbf{Hilfe beim Erkennen, Beheben und Vermeiden von Fehlern}: Fehlermeldungen sollten klar formuliert, Ursachen aufzeigen und konstruktive Lösungsvorschläge machen.
	\item \textbf{Hilfe und Dokumentation}: Auch wenn ein System ohne externe Hilfe bedienbar sein sollte, ist dokumentierte Unterstützung für komplexere Aufgaben hilfreich.
\end{enumerate}

Diese Prinzipien bieten ein umfassendes Rahmenwerk zur Gestaltung und Evaluation benutzerfreundlicher \acp{UI}.

\subsection{Gamification im UI/UX-Kontext}

Ein zunehmend relevanter Gestaltungsansatz im Bereich benutzerzentrierter Systeme ist die Gamification, also die Anwendung spieltypischer Elemente in nicht-spielerischen Kontexten, um Motivation, Engagement und Nutzerbindung zu steigern. Im Rahmen von UI/UX verfolgt Gamification das Ziel, Interaktionen mit digitalen Produkten emotional aufzuladen und ein immersives Nutzungserlebnis zu schaffen \cite{Deterding2011}.

Typische Gamification-Elemente sind unter anderem personalisierte visuelle Repräsentationen in Form von Avataren, die NutzerInnen eine Identifikation mit dem System erleichtern. Ebenso zählen Punkte, Levels und Fortschrittsanzeigen dazu, die als quantitative Rückmeldungen den NutzerInnen ein Gefühl von Zielerreichung und Fortschritt vermitteln. Darüber hinaus spielen Herausforderungen und Belohnungen eine wichtige Rolle, indem sie Aufgaben oder Ziele durch Belohnungssysteme motivierend verstärken. Ergänzt wird dies durch Feedback-Mechanismen, welche sofortige Reaktionen auf Nutzeraktionen bieten, etwa durch Animationen, akustische Signale oder farbliche Hervorhebungen\cite{Werbach2012, Deterding2011}.

Gamification stellt somit eine strategische Erweiterung klassischer UX-Designprinzipien dar, die insbesondere in Anwendungen mit repetitiven Aufgaben oder hohem Interaktionspotenzial zur Steigerung der Motivation und der Verhaltensver"anderung beitragen kann \cite{Deterding2011}.