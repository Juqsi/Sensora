\section{Anforderungen an die entwickelten Schnittstellen-Services im Projekt Sensora}

Die Hauptaufgabe der Schnittstellen-Services lag in der Konzeption und Realisierung mehrerer Dienste, die als Vermittlungskomponenten zwischen Sensor-Controllern, Benutzerschnittstellen, dem zentralen Datenspeicher sowie einem Nachrichtenübertragungsmechanismus fungieren. Diese Services sind nicht Bestandteil der Steuerlogik auf Hardwareebene, sondern unterstützen den bidirektionalen Informationsaustausch und die sichere Verwaltung verteilter Geräteinstanzen.

Ziel dieses Kapitels ist es, die funktionalen und nicht-funktionalen Anforderungen an diese Schnittstellenkomponenten technologieoffen zu definieren. Die konkrete Wahl der eingesetzten Technologien sowie deren theoretische Fundierung erfolgt erst in den nachfolgenden Kapiteln.

\subsection{Allgemeine Anforderungen an alle Schnittstellen-Services}

Unabhängig von ihrer konkreten Aufgabe müssen sämtliche entwickelten Services folgende generelle Anforderungen erfüllen, die sich aus dem Aufbau des Gesamtsystems sowie den Entwicklungsprinzipien eines modernen verteilten Softwaresystems ergeben:

\begin{itemize}
  \item \textbf{Modularität:} Die Komponenten sollen entkoppelt und unabhängig voneinander betreibbar sein, sodass sie einzeln aktualisiert, getestet und ersetzt werden können.

  \item \textbf{Plattformunabhängigkeit:} Die Services müssen innerhalb einer containerisierten Umgebung lauffähig sein und dürfen keine plattformspezifischen Abhängigkeiten voraussetzen.

  \item \textbf{Fehlertoleranz:} Die Komponenten müssen so gestaltet sein, dass bei Ausfall abhängiger Systeme (z.\,B. Netzwerk, Datenbank) keine kritischen Fehler entstehen. Entsprechende Retry-Mechanismen und Fehlerprotokollierung sind vorzusehen.

  \item \textbf{Datensicherheit:} Sensible Informationen dürfen zu keinem Zeitpunkt im Klartext übertragen oder ungeschützt gespeichert werden. Eine sichere Authentifizierung und Zugriffsbeschränkung ist auf allen öffentlich erreichbaren Schnittstellen erforderlich.

  \item \textbf{Skalierbarkeit und Erweiterbarkeit:} Die Architektur der Services soll so beschaffen sein, dass zusätzliche Sensoren, Controller oder Benutzer ohne grundlegende Systemänderungen hinzugefügt werden können.

  \item \textbf{Zuverlässigkeit bei der Kommunikation:} Die Kommunikation zwischen Services sowie mit externen Geräten soll gegen Nachrichtenverlust abgesichert sein, insbesondere bei systemkritischen Operationen wie Steuerbefehlen oder Datenspeicherung.
\end{itemize}

\section{Anforderungsanalyse für die entwickelten Services}

Im Folgenden werden die spezifischen Anforderungen an die einzelnen Services beschrieben, wobei sowohl funktionale Anforderungen (Was soll der Service leisten?) als auch systemische Rahmenbedingungen betrachtet werden.

\subsection{Registrierungs- und Authentifizierungsservice (Auth-Service)}

Der Auth-Service bildet die sicherheitsrelevante Schnittstelle zur Einbindung verteilter Steuergeräte (Controller) in das System. Er ist verantwortlich für die Validierung, Registrierung und individuelle Konfiguration dieser Geräte.

\subsubsection{Funktionale Anforderungen}
\begin{itemize}
  \item Der Service muss es ermöglichen, neue Geräteinstanzen kontrolliert durch autorisierte Verwaltungsprozesse zu registrieren. Eine unbeaufsichtigte Selbstregistrierung ist auszuschließen.
  
  \item Für jede registrierte Geräteinstanz ist ein eindeutiger Identifikator sowie ein zugehöriges Zugriffsprofil zu erzeugen. Dieses Profil muss zur differenzierten Rechtevergabe geeignet sein.
  
  \item Die Authentifizierung eines Geräts gegenüber dem System muss über ein sicheres Verfahren erfolgen, das keine langfristige Speicherung von Klartextgeheimnissen erfordert und gleichzeitig eine manipulationssichere Verifikation ermöglicht.

  \item Nach erfolgreicher Authentifizierung soll eine Kommunikationsfähigkeit zwischen Gerät und System ermöglicht werden, die sich explizit auf festgelegte Kommunikationskanäle beschränkt.

  \item Der Service muss relevante Metadaten persistieren, sodass die zugehörige Gerätelogik (z.\,B. Sensor- oder Aktorzuordnung) systemweit nachvollzogen werden kann.
\end{itemize}

\subsubsection{Nicht-funktionale Anforderungen}
\begin{itemize}
  \item Die Registrierungslogik darf nur durch explizit autorisierte Systeme oder Nutzer aufrufbar sein.
  \item Die Kommunikation mit dem Service muss verschlüsselt erfolgen.
  \item Eine versehentliche Mehrfachregistrierung desselben Geräts ist zu erkennen und abzuweisen.
\end{itemize}

\subsection{E-Mail-Verifikationsservice (Mail-Service)}

Der Mail-Service übernimmt die Kommunikation mit Nutzenden zur Verifizierung neu angelegter Benutzerkonten. Seine primäre Aufgabe ist es, die Zustellung von zeitlich begrenzten Bestätigungslinks zu ermöglichen.

\subsubsection{Funktionale Anforderungen}
\begin{itemize}
  \item Der Service muss über eine Schnittstelle ansprechbar sein, über die Verifizierungsanfragen gestellt werden können.
  \item Nach Validierung der Anfrage ist ein einmalig nutzbarer Link zu generieren und über einen gängigen E-Mail-Dienst an den Empfänger zu übermitteln.
  \item Bei Aufruf des Verifikationslinks muss der Status des entsprechenden Benutzerkontos im System aktualisiert werden.
\end{itemize}

\subsubsection{Nicht-funktionale Anforderungen}
\begin{itemize}
  \item Nur autorisierte Systeme dürfen Anfragen zur Mailverifikation stellen.
  \item Die Verifizierung darf nur erfolgen, wenn die Kombination aus Benutzername und E-Mail-Adresse im System bekannt ist.
  \item Ein nicht eingelöster Verifizierungslink muss nach Ablauf einer definierten Zeitspanne seine Gültigkeit verlieren.
\end{itemize}

\subsection{Datenschreibdienst für Messdaten (Database Writer)}

Dieser Dienst verarbeitet eingehende Datenströme von Sensoren und persistiert die extrahierten Informationen strukturiert in einem relationalen Datensystem.

\subsubsection{Funktionale Anforderungen}
\begin{itemize}
  \item Der Dienst muss kontinuierlich eingehende Daten von Messgeräten empfangen, analysieren und in geeigneter Form speichern.
  \item Die Identität der messenden Einheit muss eindeutig ermittelbar sein. Ist die Einheit im System nicht bekannt, so muss diese bei Bedarf dynamisch registriert werden können.
  \item Die Speicherung darf nur erfolgen, wenn eine vollständige logische Zuordnung der Sensordaten zu einer Pflanze bzw. zum zugehörigen Anwendungsfall gegeben ist.
  \item Es müssen regelmäßig alle bekannten Sensoren auf Aktivität geprüft werden. Wird über einen definierten Zeitraum keine neue Messung empfangen, ist dies systemintern als Fehlverhalten zu markieren.
\end{itemize}

\subsubsection{Nicht-funktionale Anforderungen}
\begin{itemize}
  \item Der Dienst muss bei Netzwerkausfällen oder vorübergehenden Störungen in der Verbindung zur Datenbank stabil bleiben.
  \item Wiederholte Nachrichtenübertragungen dürfen nicht zu doppelten Datensätzen führen.
  \item Die Datenverarbeitung muss nachvollziehbar protokolliert werden, um Fehler oder Auffälligkeiten zu diagnostizieren.
\end{itemize}

\subsection{Service zur Steuerung von Zielwerten (Setpoint API)}

Über diesen Dienst können Zielwerte (Sollwerte) für bestimmte Sensoren oder Aktoren von außen gesetzt werden. Ziel ist es, die Regelgrößen im System dynamisch anpassen zu können.

\subsubsection{Funktionale Anforderungen}
\begin{itemize}
  \item Der Dienst muss eine externe Schnittstelle bereitstellen, über die Zielwerte für bestimmte Komponenten adressiert werden können.
  \item Der Dienst muss in der Lage sein, die übermittelten Steuerinformationen an die richtige Geräteinstanz weiterzuleiten.
  \item Die übertragenen Datenpakete müssen die Zuordnung zur Zielkomponente sowie den anvisierten Wert enthalten.
  \item Eine Dokumentation der Schnittstelle muss vorliegen, um die Integration in andere Komponenten zu ermöglichen.
\end{itemize}

\subsubsection{Nicht-funktionale Anforderungen}
\begin{itemize}
  \item Der Dienst darf keine gespeicherten Zustände über Zielwerte vorhalten.
  \item Die Kommunikation muss so gestaltet sein, dass sie eine zuverlässige Zustellung ermöglicht.
  \item Fehlkonfigurationen oder fehlerhafte Eingaben müssen zu validierbaren Fehlermeldungen führen.
\end{itemize}

\subsection{Konfigurationsdienst für Kommunikationsinfrastruktur (Solace Init)}

Dieser Service ist verantwortlich für die initiale Einrichtung der Kommunikationsinfrastruktur im System, insbesondere im Hinblick auf Messaging-Komponenten wie Queues oder themenbasierte Weiterleitungspfade.

\subsubsection{Funktionale Anforderungen}
\begin{itemize}
  \item Der Dienst muss eine maschinenlesbare Konfiguration interpretieren können, in der Kommunikationskanäle, Routingregeln und Zugriffspfade definiert sind.
  \item Auf Basis dieser Konfiguration muss die zugrunde liegende Infrastruktur um definierte Ressourcen ergänzt werden.
  \item Bereits existierende Konfigurationselemente dürfen dabei nicht überschrieben oder dupliziert werden.
\end{itemize}

\subsubsection{Nicht-funktionale Anforderungen}
\begin{itemize}
  \item Der Dienst muss so gestaltet sein, dass er mehrfach ausgeführt werden kann, ohne die Konsistenz der Kommunikationsstruktur zu gefährden.
  \item Fehlerhafte Konfigurationseinträge sind zu erkennen, protokollieren und dürfen die Verarbeitung nicht abbrechen.
\end{itemize}
