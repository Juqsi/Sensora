\section{Anforderungen an das Backend}
Das Backend bildet die zentrale Kommunikationsschnittstelle zwischen Clients, internen Services und der Datenbank. Entsprechend hoch sind die Anforderungen an Struktur, Sicherheit und Stabilität.

\subsection{Architektur und Technologien}
\begin{itemize}
    \item Das Backend ist als \ac{rest}-Service zu implementieren.
    \item Es muss eine modulare, wartbare Architektur aufweisen, die dem Prinzip der Trennung von Zuständigkeiten (Separation of Concerns) folgt.
    \item Als Technologie-Stack wird eine moderne, performante Sprache wie Rust mit einem Webframework wie actix\_web empfohlen.
\end{itemize}

\subsection{Authentifizierung und Autorisierung}
\begin{itemize}
    \item Alle Zugriffe auf geschützte Ressourcen müssen durch ein Authentifizierungsverfahren abgesichert werden (z.B. \ac{jwt} oder OAuth2).
    \item Autorisierungen auf Zeit wie z.B. Sessions sind nicht erlaubt. Alternativen zu Passwörtern (z.B. Tokens) müssen begrenzt gültig sein.
\end{itemize}

\subsection{Fehlerbehandlung und Logging}
\begin{itemize}
    \item Fehlerzustände müssen konsistent behandelt und in einem strukturierten Format an den Client kommuniziert werden.
    \item Es ist ein mehrstufiges Logging-System zu implementieren, das zwischen Info, Warnung und Fehler unterscheidet.
    \item Sensible Informationen dürfen in Logs nicht gespeichert werden. Logs müssen zentral gesammelt und gegen Manipulation abgesichert werden.
\end{itemize}

\subsection{Skalierbarkeit und Performance}
\begin{itemize}
    \item Das Backend ist zustandslos zu gestalten, um horizontale Skalierung über Load-Balancing zu ermöglichen.
    \item Die Antwortzeiten für einfache \ac{crud} Operationen sollen im Normalbetrieb unter 100ms liegen.
    \item Die Architektur soll auf Lastspitzen vorbereitet sein (z.B. durch Queues oder Caching).
\end{itemize}

\subsection{Sicherheit}
\begin{itemize}
    \item Gängige Sicherheitslücken (z.B. SQL-Injection, XSS, CSRF) sind durch geeignete Mechanismen zu verhindern.
    \item Eingaben vom Client sind streng zu validieren und zu sanitieren.
\end{itemize}

\subsection{API-Dokumentation}
\begin{itemize}
    \item Die Schnittstellen müssen vollständig dokumentiert werden.
    \item Eine maschinenlesbare API-Spezifikation (z.B. OpenAPI 3.0) ist zu pflegen.
    \item Optional kann eine interaktive API-Oberfläche für Entwickler bereitgestellt werden (z.B. Swagger UI).
\end{itemize}

\section{Anforderungen an die Datenbank}
Die Datenbank dient als persistente Grundlage für alle im System gespeicherten Informationen. Sie muss sowohl performant als auch sicher und konsistent arbeiten.

\subsection{Modellierung und Struktur}
\begin{itemize}
    \item Das Datenbankschema ist klar zu dokumentieren und mindestens in der 3. Normalform zu entwerfen, sofern nicht durch Performance-Aspekte begründet anders.
    \item Entitäten und ihre Beziehungen müssen nachvollziehbar und versionierbar abgebildet werden.
\end{itemize}

\subsection{Datensicherheit und Integrität}
\begin{itemize}
    \item Sensible Daten (z.B. Passwörter, Tokens) müssen verschlüsselt gespeichert werden.
    \item Datenbankeigene Mechanismen wie Constraints, Foreign Keys und ggf. Trigger sind zur Sicherstellung der Datenintegrität zu verwenden.
    \item Referentielle Integrität ist in allen relevanten Tabellen durchzusetzen.
\end{itemize}

\subsection{Zugriffskontrolle}
\begin{itemize}
    \item Der Datenbankzugriff erfolgt ausschließlich über definierte Rollen mit minimalen Rechten.
    \item Es muss zwischen Administrations-, Lese- und Schreibzugriff differenziert werden.
    \item Externe Dienste erhalten nur selektiven Zugriff auf erforderliche Tabellen.
\end{itemize}

\subsection{Backups und Wiederherstellung}
\begin{itemize}
    \item Es ist ein automatisiertes Backup-Konzept zu implementieren, welches tägliche Snapshots sowie inkrementelle Sicherungen vorsieht.
    \item Wiederherstellungsroutinen müssen dokumentiert und regelmäßig geprüft werden.
\end{itemize}

\subsection{Performance und Skalierung}
\begin{itemize}
    \item Für häufig genutzte Felder sind geeignete Indizes zu definieren.
    \item Bei wachsendem Datenvolumen sollen Mechanismen wie Read-Replicas, Sharding oder Partitionierung zum Einsatz kommen.
    \item Abfragen müssen gezielt optimiert und auf lange Laufzeiten geprüft werden.
\end{itemize}

\subsection{Technologischer Rahmen}
\begin{itemize}
    \item Die Datenbanklösung muss Open Source, stabil, transaktionssicher und für hohe Datenmengen geeignet sein.
    \item Es wird der Einsatz von PostgreSQL empfohlen.
\end{itemize}