\section{Motivation}
\label{sec:Motivation}

Pflanzen gehören in Deutschland und Europa fest zum Alltag in Wohnung und Garten. Laut einer repräsentativen Umfrage aus dem Jahr 2020 besitzen rund drei Viertel der Bundesbürger:innen (74\,\%) Zimmerpflanzen in ihrem Zuhause; auch auf Balkonen (35\,\%), Terrassen (30\,\%) und Fensterbänken (21\,\%) grünt es, während nur etwa 10\,\% ganz ohne Pflanzen leben. \autocite{pflanzenbesitz_de} Dieses „grüne Zuhause“ liegt im Trend und gewann insbesondere während der COVID-19-Pandemie an Bedeutung – viele Menschen entdeckten 2020 im Home-Office ihre Liebe zu Haus- und Gartenpflanzen neu. \autocite{gabot_artikel} Entsprechend stieg der Absatz: Der deutsche Markt für Blumen und Zierpflanzen erreichte nach Jahren der Stagnation 2020 ein Rekordvolumen von 9{,}4\,Mrd.\,€. \autocite{gabot_artikel} 
\\
Ähnlich hohe Werte zeigen sich europaweit, wo Pflanzen als wichtiger Teil der Wohn- und Lebensqualität gelten. Neben dekorativen Aspekten werden Zimmer- und Gartenpflanzen aufgrund positiver Effekte wie besserer Luftqualität und Stressreduktion geschätzt. \autocite{pflanzenbesitz_de} Die hohe Verbreitung und Wertschätzung von Pflanzen in Privathaushalten bildet den Ausgangspunkt für die Betrachtung, wie ihre Pflege im Alltag unterstützt werden kann.
\\
Allerdings stehen viele Pflanzenbesitzer:innen vor praktischen Herausforderungen bei der Pflege ihres „grünen Mitbewohners“. Im hektischen Alltag wird das Gießen leicht vergessen oder unregelmäßig vorgenommen; umgekehrt gießen unerfahrene Halter oft zu viel aus Sorge um die Pflanze. Studien bestätigen, dass Überwässerung der häufigste Grund für das Eingehen von Zimmerpflanzen ist. \autocite{pflanzenpflege_fehler} 
\\
Generell erfordert jede Pflanzenart spezifische Kenntnisse zu Wasser- und Nährstoffbedarf, Lichtverhältnissen etc., über die im privaten Umfeld nicht immer ausreichend Wissen vorhanden ist. So gaben in einer Umfrage lediglich 37\,\% der befragten Frauen und 20\,\% der Männer an, einen „grünen Daumen“ zu haben \autocite{pflanzenbesitz_de} – die Mehrheit traut sich die optimale Pflanzenpflege also eher nicht zu. Hinzu kommt, dass während Urlaubs- oder Abwesenheitszeiten oft keine Betreuung für die heimischen Gewächse sichergestellt ist. Tatsächlich vermissten in einer Befragung 26\,\% der Pflanzenhalter:innen ihre Zimmerpflanzen im Urlaub sogar mehr als die Kolleg:innen \autocite{pflanzenbesitz_de}, was die emotionale Bindung und zugleich das Problem der Versorgung in dieser Zeit verdeutlicht. Diese Pflegeherausforderungen führen dazu, dass viele privat gehaltene Pflanzen Schäden nehmen oder vorzeitig absterben.
\\
Die Folgen von falscher oder unregelmäßiger Pflege sind in Zahlen beträchtlich. Hochrechnungen zufolge überlebt ein erheblicher Teil der gekauften Zierpflanzen nicht lange: Viele Pflanzen gehen bereits in der Lieferkette zugrunde, und rund 35\,\% der gekauften Zimmerpflanzen sterben später in den Wohnungen der Kundschaft. \autocite{pflanzensterben_statistik} Mit anderen Worten wird etwa ein Drittel aller gekauften Haus- und Gartenpflanzen letztlich aufgrund suboptimaler Bedingungen oder Pflegefehler nicht dauerhaft erhalten. Auch Verbraucherumfragen deuten auf dieses Problem hin. Beispielsweise gab über ein Drittel der Hobbygärtner in einer aktuellen Erhebung an, jedes Jahr ein bis zwei Zimmerpflanzen zu verlieren. \autocite{pflanzensterben_statistik_2} 
\\
Solche Verluste sind nicht nur emotional enttäuschend für Pflanzenliebhaber, sondern bedeuten auch Ressourcenverschwendung – insbesondere von Wasser, Zeit und Geld. Schätzungen aus den USA zeigen etwa, dass jüngere „Plant Parents“ im Durchschnitt schon Sieben ihrer erworbenen Pflanzen unbeabsichtigt zum Eingehen gebracht haben.\autocite{pflanzensterben_statistik_3} Diese Zahlen unterstreichen die Notwendigkeit, neue Wege zu finden, um häufige Pflegefehler zu vermeiden und die Lebensdauer der Pflanzen zu verlängern.
\\
Technologische Lösungen im Sinne von \textit{Smart Gardening} setzen hier an und versprechen Abhilfe. Insbesondere automatische Bewässerungssysteme für den Heimgebrauch bieten die Möglichkeit, den Gießvorgang zu optimieren und zu automatisieren. Solche Systeme kombinieren oft Sensoren (etwa für Bodenfeuchte oder Licht) mit internetfähigen Steuerungen, um den Pflanzen exakt bei Bedarf und in der richtigen Menge Wasser zuzuführen. 
\\
Erste Ansätze sind bereits auf dem Markt verfügbar – von App-gesteuerten Bewässerungscomputern bis hin zu smarten Pflanzentöpfen mit Selbstbewässerungs-Funktion. Die Akzeptanz solcher \textit{Smart-Home}-Technologien im Garten- und Pflanzenbereich steigt kontinuierlich. Laut dem STIHL-Gartenbarometer 2022 nutzen bereits rund 7\,\% der deutschen Gartenbesitzer smarte Garden-Lösungen, und etwa 30\,\% wünschen sich zukünftig solche automatisierten Helfer. \autocite{smart_gardening} Dabei stehen Bewässerungsautomationen an erster Stelle der Wunschliste: 83\,\% der Befragten mit Smart-Gardening-Interesse nennen ein automatisches Bewässerungssystem als besonders gefragte Lösung. Diese Nachfrage spiegelt sich auch in anderen Ländern wider. Beispielsweise glauben in Österreich über 60\,\% der Gartenbesitzer, dass sich der Wasserverbrauch durch automatisierte Bewässerungsanlagen deutlich optimieren lässt. \autocite{gardena} 
\\
Moderne Systeme können Wetterdaten oder Bodensensoren einbeziehen, um nur dann zu wässern, wenn die Pflanze es wirklich benötigt – eine Technik, die den Pflanzenstress reduziert und zugleich Wasserverschwendung vorbeugt. Aktuelle Untersuchungen zeigen denn auch, dass intelligente Bewässerungssteuerungen den Wasserverbrauch im Garten gegenüber herkömmlichen Timern erheblich senken können (um etwa 20–40\,\% je nach System). \autocite{water_savings}
\\
Mehrere übergeordnete Trends begünstigen die Verbreitung von smarten Pflanzenpflege-Systemen. Zum einen führt die Urbanisierung dazu, dass immer mehr Menschen auf kleinem Raum in Städten leben – in Deutschland etwa 78\,\% der Bevölkerung \autocite{urbanisierung_de} – und sich dennoch nach Natur im eigenen Umfeld sehnen.  Insbesondere Stadtbewohner ohne Garten kultivieren vermehrt Zimmerpflanzen oder Balkongrün, sind aber beruflich oft stark eingebunden. Eine automatische Bewässerung kann hier den Pflegeaufwand mindern und sicherstellen, dass Pflanzen trotz hektischem Alltag oder Abwesenheiten ausreichend versorgt werden. 
\\
Zum anderen rückt Nachhaltigkeit in den Fokus: Wassermanagement und effiziente Ressourcennutzung gewinnen an Bedeutung, da die Auswirkungen des Klimawandels – etwa häufigere Sommerdürreperioden – auch private Gärten und Balkone betreffen. In Umfragen äußern fast zwei Drittel der Befragten die Erwartung, dass digitale Technologien im Garten helfen können, den Klimawandel abzuschwächen, und nennen den schonenden Umgang mit Wasser als oberste Priorität. \autocite{gardena} Smart-Bewässerungssysteme erfüllen genau diesen Zweck, indem sie bedarfsgerecht gießen und Überwässerung verhindern. 
\\
Schließlich trägt auch die allgemeine Verbreitung von \textit{Internet of Things}-Anwendungen im Haushalt dazu bei, dass vernetzte Lösungen immer selbstverständlicher werden. Der europäische Smart-Home-Markt verzeichnet hohe Wachstumsraten und wird 2024 bereits auf über 22\,Mrd.\,US-\$ geschätzt \autocite{iot_trend}. Vernetzte, per App oder Sprache steuerbare Geräte – vom Thermostat bis zur Lichtsteuerung – gehören zunehmend zum Alltag. Diese Entwicklung macht auch vor dem Bereich der Pflanzenpflege nicht Halt: Die Nutzerakzeptanz für digitale Helfer im Haushalt schafft ein günstiges Umfeld für \textit{Smart Gardening}-Innovationen.
\\
Insgesamt ist die Einführung eines smarten Bewässerungssystems im heimischen Umfeld vor dem Hintergrund dieser Fakten sowohl technisch zeitgemäß als auch gesellschaftlich sinnvoll. Die weit verbreitete Haltung von Zimmer- und Gartenpflanzen einerseits und die häufig auftretenden Pflegeprobleme andererseits schaffen ein deutliches Bedürfnis nach Unterstützung. Automatisierte Bewässerungslösungen können hier einen doppelten Nutzen stiften: Sie helfen Pflanzenbesitzer:innen, ihre grünen Schützlinge zuverlässig und fachgerecht zu versorgen, und tragen zugleich zu Nachhaltigkeit und Komfort bei. Indem ein smartes Bewässerungssystem Wasser bedarfsgerecht dosiert und den Pflegeprozess vereinfacht, steigert es die Überlebensrate und Vitalität der Pflanzen und entlastet den Menschen von Routineaufgaben. 
\\
Die vorliegenden Studien, Statistiken und Trends untermauern somit die Notwendigkeit und den Nutzen eines solchen Systems, das im Folgenden technisch konzipiert und beschrieben wird.
