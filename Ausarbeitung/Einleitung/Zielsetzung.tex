\section{Zielsetzung}
\label{sec:Zielsetzung}

Ziel dieser Studienarbeit ist die Konzeption, Entwicklung und prototypische Umsetzung eines automatisierten Systems zur Bewässerung von Zimmerpflanzen im privaten Wohnumfeld. Es soll eine lauffähige Gesamtlösung entstehen, die aus einem Mikrocontroller als zentrale Steuereinheit, einer Backend-Infrastruktur zur Datenverarbeitung und -persistierung sowie einem benutzerfreundlichen Frontend zur Visualisierung und Steuerung besteht. Die Realisierung erfolgt im Rahmen eines \textit{Proof of Concept}, der die technische Machbarkeit sowie die Integration der Systemkomponenten demonstriert.
\\
Das zu entwickelnde System erfasst über geeignete Sensorik (z.\,B. Bodenfeuchte, Temperatur, Luftfeuchtigkeit, Lichtintensität) kontinuierlich relevante Umgebungsdaten. Diese Messwerte dienen entweder als Entscheidungsgrundlage für den Nutzer, um über die Benutzeroberfläche manuell eine Bewässerung anzustoßen, oder sie werden vom Mikrocontroller automatisch verarbeitet. Im letzteren Fall wird anhand zuvor definierter Sollwerte eine autonome Steuerung der Bewässerungseinheit realisiert. Vorrangiges Ziel ist die technische Umsetzung des automatischen Betriebsmodus. Die Konzeption und Entwicklung des manuellen Modus sowie die Integration beider Steuerungsarten in das Gesamtsystem erfolgen nachrangig und abhängig von den im Projektverlauf verfügbaren Entwicklungskapazitäten.
\\
Die Bewässerungslösung ist primär für den Einsatz in Innenräumen konzipiert. Dies umfasst insbesondere Haushalte mit Zimmerpflanzen, bei denen typische Pflegeprobleme wie unregelmäßiges Gießen oder Unsicherheit bezüglich des Wasserbedarfs adressiert werden sollen.
\\
Der konkrete Funktionsumfang des Systems wird im Verlauf des Projekts iterativ entwickelt. Eine detaillierte Beschreibung der funktionalen und nicht-funktionalen Anforderungen sowie der Zielsystemeigenschaften erfolgt in Kapitel~\ref{chap:Anforderungen}. Dabei wird angestrebt, etablierte \textit{Best Practices} der Software- und Systementwicklung zu berücksichtigen und – wo sinnvoll und realisierbar – aktuelle Technologien gemäß dem Stand der Technik (\textit{State of the Art}) zu verwenden. Gleichzeitig wird die technische Umsetzung unter Berücksichtigung der konzeptionellen Natur als \textit{Proof of Concept} gewichtet, sodass pragmatische Abwägungen hinsichtlich Komplexität, Aufwand und Ressourcen erfolgen.
\\
Insgesamt dient die Arbeit dem Ziel, ein funktional überzeugendes Demonstrationssystem zu realisieren, das eine fundierte Grundlage für weiterführende Entwicklungen, Evaluationen oder mögliche Produktivsetzungen bietet.
